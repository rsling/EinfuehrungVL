%\documentclass[handout,aspectratio=1610]{beamer}
\documentclass[aspectratio=1610]{beamer}

%\usepackage[T1]{fontenc}
\usepackage[ngerman]{babel}
\usepackage{color}
\usepackage{colortbl}
\usepackage{textcomp}
\usepackage{multirow}
\usepackage{nicefrac}
\usepackage{multicol}
\usepackage{gb4e-}
\usepackage{verbatim}
\usepackage{cancel}
\usepackage{graphicx}
\usepackage{hyperref}
\usepackage{verbatim}
\usepackage{boxedminipage}
\usepackage{rotating}
\usepackage{booktabs}
\usepackage{bbding}
\usepackage{pifont}
\usepackage{multicol}
\usepackage{stmaryrd}

\usepackage{tikz}
\usetikzlibrary{positioning,arrows,cd}
\tikzset{>=latex}

\usepackage[linguistics]{forest}

\usepackage{FiraSans}

\usepackage[maxbibnames=99,
  maxcitenames=2,
  uniquelist=false,
  backend=biber,
  doi=false,
  url=false,
  isbn=false,
  bibstyle=biblatex-sp-unified,
  citestyle=sp-authoryear-comp]{biblatex}
\addbibresource{rs.bib}

\forestset{
  Ephr/.style={draw, ellipse, thick, inner sep=2pt},
  Eobl/.style={draw, rounded corners, inner sep=5pt},
  Eopt/.style={draw, rounded corners, densely dashed, inner sep=5pt},
  Erec/.style={draw, rounded corners, double, inner sep=5pt},
  Eoptrec/.style={draw, rounded corners, densely dashed, double, inner sep=5pt},
  Ehd/.style={rounded corners, fill=gray, inner sep=5pt,
    delay={content=\whyte{##1}}
  },
  Emult/.style={for children={no edge}, for tree={l sep=0pt}},
  phrasenschema/.style={for tree={l sep=2em, s sep=2em}},
  decide/.style={draw, chamfered rectangle, inner sep=2pt},
  finall/.style={rounded corners, fill=gray, text=white},
  intrme/.style={draw, rounded corners},
  yes/.style={edge label={node[near end, above, sloped, font=\scriptsize]{Ja}}},
  no/.style={edge label={node[near end, above, sloped, font=\scriptsize]{Nein}}},
  sake/.style={tier=preterminal},
  ake/.style={
    tier=preterminal
    },
}


\tikzset{
    invisible/.style={opacity=0,text opacity=0},
    visible on/.style={alt=#1{}{invisible}},
    alt/.code args={<#1>#2#3}{%
      \alt<#1>{\pgfkeysalso{#2}}{\pgfkeysalso{#3}} % \pgfkeysalso doesn't change the path
    },
}
\forestset{
  visible on/.style={
    for tree={
      /tikz/visible on={#1},
      edge+={/tikz/visible on={#1}}}}}

\forestset{
  narroof/.style={roof, inner xsep=-0.25em, rounded corners},
}


\definecolor{lg}{rgb}{.8,.8,.8}
\newcommand{\Dim}{\cellcolor{lg}}


\newcommand{\Sub}[1]{\ensuremath{_{\text{#1}}}}
\newcommand{\Up}[1]{\ensuremath{^{\text{#1}}}}
\newcommand{\UpSub}[2]{\ensuremath{^{\text{#1}}_{\text{#2}}}}
 
\newcommand{\Spur}[1]{t\Sub{#1}}
\newcommand{\Ti}{\Spur{1}}
\newcommand{\Tii}{\Spur{2}}
\newcommand{\Tiii}{\Spur{3}}
\newcommand{\Tiv}{\Spur{4}}

\newcommand{\Ck}{\CheckmarkBold}
\newcommand{\Fl}{\XSolidBrush}

\usetheme[hideothersubsections]{Goettingen}

%\newcommand{\rot}[1]{{\color[rgb]{0.4,0.2,0}#1}}
%\newcommand{\blau}[1]{{\color[rgb]{0,0,.9} #1}}
\newcommand{\gruen}[1]{{\color[rgb]{0,0.4,0}#1}}
\newcommand{\blaw}[1]{{\color[rgb]{0,0,.9}#1}}

\newcommand{\rot}[1]{{\color[rgb]{0.6,0.2,0.0}#1}}
\newcommand{\blau}[1]{{\color[rgb]{0.0,0.0,0.9}#1}}
\newcommand{\orongsch}[1]{{\color[RGB]{255,165,0}#1}}
\newcommand{\grau}[1]{{\color[rgb]{0.5,0.5,0.5}#1}}

\renewcommand<>{\rot}[1]{%
  \alt#2{\beameroriginal{\rot}{#1}}{#1}%
}
\renewcommand<>{\blau}[1]{%
  \alt#2{\beameroriginal{\blau}{#1}}{#1}%
}
\renewcommand<>{\orongsch}[1]{%
  \alt#2{\beameroriginal{\orongsch}{#1}}{#1}%
}

\definecolor{trueblue}{rgb}{0,0.0,0.7}
\setbeamercolor{alerted text}{fg=trueblue}

\newcommand{\xxx}{\hspaceThis{[}}
\newcommand{\zB}{z.\,B.\ }
\newcommand{\down}[1]{\ensuremath{\mathrm{#1}}}

\newcommand{\Zeile}{\vspace{\baselineskip}}
\newcommand{\Halbzeile}{\vspace{0.5\baselineskip}}
\newcommand{\Viertelzeile}{\vspace{0.25\baselineskip}}

\newcommand{\whyte}[1]{\textcolor{white}{#1}}

\newcommand{\KTArr}[1]{\ding{226}~\textit{#1}~\ding{226}}
\newcommand{\Ast}{*}

\newcommand{\SL}{\ensuremath{\llbracket}}
\newcommand{\SR}{\ensuremath{\rrbracket}}



\addtobeamertemplate{navigation symbols}{}{%
    \usebeamerfont{footline}%
    \usebeamercolor[fg]{footline}%
    \hspace{1em}%
    \insertframenumber/\inserttotalframenumber
}

\newcounter{lastpagemainpart}

\resetcounteronoverlays{exx}

\AtBeginSection[]{
  \begin{frame}
  \vfill
  \centering
  \begin{beamercolorbox}[sep=8pt,center,shadow=true,rounded=true]{title}
    \usebeamerfont{title}\insertsectionhead\par%
  \end{beamercolorbox}
  \vfill
  \end{frame}
}

\title{Einführung in die Sprachwissenschaft\\
12.~Graphematik und Phonologie}
\author{Roland Schäfer}
\institute{Deutsche und niederländische Philologie\\Freie Universität Berlin}
\date{Wintersemester 2018/2019\\29.~Januar 2019}

\begin{document}

\frame{\titlepage}

\section{Rückblick}

\begin{frame}
  {Rückblick: Syntaktische Relationen}
  \pause
  \begin{itemize}[<+->]
    \item semantische Rollen: Syntax-Semantik-Schnittstelle für Verben
      \Halbzeile
    \item Satzprädikat: entweder "`finites Verb"' oder \rot{undefiniert}
    \item andere "`prädikative"' Konstituenten: \alert{Kopula-Test}
      \Halbzeile
    \item \alert{Valenzänderungen und Valenzanreicherungen}
      \begin{itemize}[<+->]
        \item Vorgangspassiv (\textit{werden}, \orongsch{Nom\_Ag→\textit{von}-PP}, \alert{ggf.\ Akk→Nom})
        \item Rezipientenpassiv (\textit{bekommen}, \orongsch{Nom\_Ag→\textit{von}-PP}, \rot{Dat→Nom})
        \item "`freie Dative"': Valenzerweiterung (bis auf Bewertungsdativ)
      \end{itemize}
      \Halbzeile
    \item Ergänzungen und Angaben:
      \begin{itemize}[<+->]
        \item Subjekt: regierter und mit Verb kongruierender \alert{Nom}\\
           (oder Satz an dessen Stelle)
         \item dir.\ Objekt: verbregierter (ggf.\ vom Vorgangspassiv betroffener) \alert{Akk}\\
          (oder Satz an dessen Stelle)
        \item indir.\ Objekt: verbregierter (vom Rezipientenpassiv betroffener) \alert{Dat}
        \item \alert{Rollenbindung ans Verb} oder nicht
        \item bei PPs: Auskopplungstest (aber problematisch)
      \end{itemize}
  \end{itemize}
\end{frame}


\section{Überblick}

\begin{frame}
  {Graphematik: Segmentschreibungen}
  \pause
  \begin{itemize}[<+->]
    \item Graphematik als Teil der Grammatik\slash Linguistik
      \Halbzeile
    \item \alert{phonologisches Schreibprinzip}:\\
      zugrundeliegende Form $\Leftrightarrow$ Buchstabe
    \item große Ausnahme davon bei den Vokalen 
      \Halbzeile
    \item Nicht-Prinzip der Dehnungsschreibung (unsystematisch)
    \item \alert{Prinzip der Gelenkschreibung} ("`Schärfungsschreibung"')
      \Halbzeile
    \item Eszett und die Eliminierung des zugrundeliegenden /s/
    \item Grenz-\textit{h}
      \Halbzeile
    \item nicht gesondert behandelt: \alert{Orthographie} (Norm)\\
      vs.\ \alert{Graphematik} (linguistische Analyse der Schreibprinzipien)
    \item idealerweise: Orthographie folgt (verzögert) der Graphematik\\
      (Prinzip: Norm als Beschreibung und vorsichtige Standardisierung)
  \end{itemize}
\end{frame}

\begin{frame}
  {Bedeutung für Erwerb und Lehre der Schriftsprache}
  \pause
  \begin{itemize}[<+->]
    \item Das müssen wir nicht besonders betonen, oder?
      \Halbzeile
    \item extreme Aufgabe für Lerner*innen ab JGS 1:
      \begin{itemize}[<+->]
        \item Erwerb der Buchstaben\ldots\ naja, kein Problem
        \item aber: Schreibprinzipien mit allen grammatischen Ebenen verbunden
        \item \rot{explizites Erlernen für (Grund-)Schulkinder nahezu unmöglich}
      \end{itemize}
      \Halbzeile
    \item Aufgaben der Lehrpersonen im weitgehend impliziten Lernprozess:
      \begin{itemize}[<+->]
        \item \alert{korrekten und \textbf{geschriebenen} Input auswählen}\\
          (vgl.\ Anlaut-\slash Auslautreihen oder das Prinzip \alert{Kern vor Peripherie})
        \item \alert{Produktionsprobleme richtig klassifizieren, richtig helfen}
        \item notgedrungen: \alert{Aussprache des Standards parallel vermitteln}
      \end{itemize}
      \Halbzeile
    \item Viele Dinge sind so einfach\ldots\ Bitte:
      \begin{itemize}[<+->]
        \item \rot{\textbf{nicht}} sofort zur Lese-\slash Schreibförderung schicken,\\
          denn das heißt zu \rot{kapitulieren}, \rot{brandmarken} und \rot{demotivieren}
        \item \rot{\textbf{niemals}} \rot{Hinhörschreibungen} lehren: \alert{immer und}\\
          \alert{von Anfang an den korrekten geschriebenen Input geben}
        \item folglich: \rot{\textbf{niemals} "`Ausprobierschreibungen"' zulassen}  
      \end{itemize}
  \end{itemize}
\end{frame}


\section{Graphematik als Teil der Grammatik?}

\begin{frame}
  {Was ist hier falsch?}
  \pause
  Alle diese Schreibungen sind mögliche Schreibungen,\\
  kodieren aber etwas Anderes als im Kontext grammatisch nötig.\\
  \Halbzeile
  \pause
  \begin{exe}
    \ex\label{ex:graphematikalsteildergrammatik001}
    \begin{xlist}
      \ex[*]{\label{ex:graphematikalsteildergrammatik002} Fine findet, \rot{das} die Schuhe gut aussehen.}
      \pause
      \ex[*]{\label{ex:graphematikalsteildergrammatik003} Wenn ich Geld hätte, \rot{nehme} ich den Kopfhörer mit.}
      \pause
      \ex[*]{\label{ex:graphematikalsteildergrammatik004} Um voranzukommen, nimmt Fine an der Fortbildung \rot{Teil}.}
      \pause
      \ex[*]{\label{ex:graphematikalsteildergrammatik005} \rot{Zurückbleibt} der Schreibtisch nur, wenn der LKW randvoll ist.}
    \end{xlist}
  \end{exe}
  \pause
  \begin{itemize}[<+->]
    \item falsche lexikalische Schreibung → Wort existiert,\\
      \alert{hier falsche Wortklasse}
    \item falsche Segmentschreibung → Form möglich, \alert{hier falsche Flexionsform}
    \item falsche Wort(klassen)schreibung → Wort existiert,\\
      \alert{hier falscher morphosyntaktischer Status}
    \item falsche Wortschreibung (Spatium) → \textit{zurückbleibt} anderswo möglich\\
      \alert{hier durch Bewegungssyntax ausgeschlossen}
  \end{itemize}
\end{frame}

\begin{frame}
  {Einordnung und andere Meinungen I}
  \pause
  \begin{itemize}[<+->]
    \item Graphematik als eins der \alert{Kodierungssysteme der Grammatik}
    \item Relevanzunterschied zu Phonetik (= anderes Medium)? --- \alert{Keiner!}
    \item Und \alert{Gebärdensprache}?
    \Halbzeile
    \item \alert{Natürlich gehört die Graphematik zur Grammatik\slash Linguistik.}
    \Halbzeile
    \item \rot{Aber viele Sprachen haben keine Schriftsysteme!}
      \begin{itemize}[<+->]
        \item \alert{Ja und? Viele haben eins, \zB das Deutsche.}
      \end{itemize}
      \Viertelzeile
    \item \rot{Aber es gibt Sprachen ohne Schrift und keine Schrift ohne Sprache!}
      \begin{itemize}[<+->]
        \item \alert{Ja und? Im Gegenteil: In \textit{Kulturen}, die Jahrhunderte oder -tausende lang\\
        verschriften, gibt es erhebliche Rückkopplungen\\
        zwischen Gesprochenem und Geschriebenem, \zB\ im Deutschen.}
      \end{itemize}
      \Viertelzeile
    \item \rot{Aber die Schrift haben sich Leute ausgedacht!}\\
      (soll heißen: Die Schreibung hat sich nicht natürlich entwickelt.)
      \begin{itemize}[<+->]
        \item \alert{Ach? Schonmal die Entwicklung der deutschen Schreibung angesehen?}
      \end{itemize}
  \end{itemize}
\end{frame}

\begin{frame}
  {Einordnung und andere Meinungen II}
  \pause
  \begin{itemize}[<+->]
    \item \rot{Aber die Schriftsprache ist nicht spontan, daher uninteressant\\
      für Linguistik (= Erforschung unbewusster kognitiver Vorgänge)!}
      \begin{itemize}[<+->]
        \item \alert{Ach? Sagen Linguist*innen, die glauben, dass sie selber (oder andere)\\
          durch Introspektion an ihre interne Grammatik rankommen!}
        \item Bildungssprache tendiert generell zur reflektierten \alert{Überformung},\\
          das Medium spielt dafür nur tendentiell eine Rolle.
      \end{itemize}
      \Viertelzeile
    \item \rot{Aber Kinder lernen zuerst Sprechen, ohne Schrift!}
      \begin{itemize}[<+->]
        \item \alert{Ja und? Wir beschreiben beide Kodierungssysteme ja auch getrennt.\\
          Niemand sagt, dass das dasselbe ist.}
        \item Das akustische Medium hat meist aus praktischen Gründen Vorrang\\
          (aber vgl.\ \zB gehörlose Kinder).
      \end{itemize}
      \Viertelzeile
    \item \rot{Aber aus diesen } (falschen) \rot{Gründen, hält die gesprochene Sprache\\
      in der Linguistik traditionell das Primat über die geschriebene!}
      \begin{itemize}[<+->]
        \item \alert{Blanker Unsinn. Die meisten Linguist*innen, die sowas behaupten,\\
          haben keinerlei Ahnung von gesprochener Sprache.}
        \item \grau{Vgl.\ \citet{Schwitalla2011} zur Einführung in gesprochene Sprache.}
      \end{itemize}
  \end{itemize}
\end{frame}

\begin{frame}
  {Erinnerung: der Kernwortschatz}
  \pause
  Was war nochmal der Kernwortschatz?\\
  \Halbzeile
  \pause
  \begin{itemize}[<+->]
    \item Wörter, für die \alert{die weitreichenden Generalisierungen gelten}
    \item = Wörter und Wortklassen mit \alert{hoher Typenhäufigkeit}
    \item \rot{nicht} die "`häufigen Wörter"' (= Tokenhäufigkeit)
    \item \rot{nicht} die Erbwörter (aber Erbwörter meistens im Kern)
      \Halbzeile
    \item Kern-Substantive: Einsilbler (im Plural Trochäus) oder Trochäus
    \item warum gerade Substantive so zentral?\\
      \alert{mit Abstand die mächtigste Wortklasse}
      \Halbzeile
    \item \rot{Missverständnis}: Kern\slash Peripherie klar abgegrenzt
    \item je höher die Typenhäufigkeit, desto kerniger
    \item periphere Wörter, Konstruktionen usw.\ \alert{nicht weniger grammatisch}
      \Halbzeile
    \item Egal, was man Ihnen erzählt: \rot{Die Definition ist nicht zirkulär!}
  \end{itemize}
\end{frame}

\section{Segment\-schreibungen}

\begin{frame}
  {Ordnung total: die Konsonantenzeichen}
  \pause
  \centering
  \resizebox{0.48\textwidth}{!}{
    \begin{tabular}{lll}
      \toprule
      \textbf{Segment} & \textbf{Buchstabe(n)} & \textbf{Beispielwörter} \\
      \midrule
     p & p & \textit{Plan} \\
     b & b & \textit{Baum}, \textit{Trab} \\
     p͡f & pf & \textit{Pfad} \\
     f & f & \textit{Fahrt} \\
     v & w & \textit{Wand} \\
     m & m & \textit{Mus} \\
     t & t & \textit{Tau} \\
     d & d & \textit{Dach}, \textit{Bild}\\
     t͡s & z & \textit{Zeit} \\
     \rot{s} & \rot{s} & \textit{Los} \\
     \rot{z} & \rot{s} & \textit{Sau} \\
     ʃ & sch & \textit{Schiff} \\
     n & n & \textit{Not}, \textit{Klang} \\
     l & l & \textit{Lob} \\
     ç & ch & \textit{Blech}, \textit{Wacht} \\
     ʝ & j & \textit{Jahr} \\
     k & k & \textit{Kiel} \\
     g & g & \textit{Gans}, \textit{Weg}, \textit{König} \\
     ʁ & r & \textit{Ritt}, \textit{Tür} \\
     h & h & \textit{Herz} \\
      \bottomrule
    \end{tabular}
  }
\end{frame}

\begin{frame}
  {Invarianz der Konsonantenzeichen}
  \pause
  \alert{Wir schreiben, wie unsere zugrundeliegenden Formen aussehen.}\\
  \pause
  \Zeile
  \centering
  \resizebox{0.9\textwidth}{!}{
    \begin{tabular}{lp{0.15cm}lp{0.15cm}llp{0.15cm}llp{0.15cm}l}
      \toprule
      \textbf{zugr.} && \textbf{Buch-} && \multicolumn{2}{l}{\textbf{phonetische}}    && \multicolumn{2}{l}{\textbf{phonologische}} && \textbf{phonetische} \\
      \textbf{Segm.} && \textbf{stabe(n)} && \multicolumn{2}{l}{\textbf{Realisierungen}} && \multicolumn{2}{l}{\textbf{Schreibungen}}  && \textbf{Schreibung} \\
      \midrule
      b && b && ba͡ɔm & loːp && \textit{Baum} & \textit{Lob} && *\textit{Lop} \\
      d && d && daχ & ʁɪnt && \textit{Dach} & \textit{Rind} && *\textit{Rint} \\
      n && n && naχt & klaŋ && \textit{Nacht} & \textit{Klang} && *\textit{Klaŋ} \\
      ç && ch && lɪçt & vaχt && \textit{Licht} & \textit{Wacht} && *\textit{Waχt} \\
      g && g && gans & køːnɪç && \textit{Gans} & \textit{König} && *\textit{Könich} \\
      ʁ && r && ʁuːm & to͡ɐ && \textit{Ruhm} & \textit{Tor} && *\textit{Toe} \\
      \bottomrule
    \end{tabular}
  }
  \Zeile
  \pause
  \begin{itemize}[<+->]
    \item einige Substitutionsphänome (anlautendes /kv/ als \textit{qu} usw.)
    \item \alert{Das Problem mit den \textit{s}-Schreibungen wird noch gelöst!}
  \end{itemize}
\end{frame}

\begin{frame}
  {Ordnung naja: Vokalzeichen}
  \pause
  \centering
  \scalebox{0.8}{%
    \begin{tabular}{lp{0.5cm}llp{0.25cm}ll}
      \toprule
      \multirow{2}{*}{\textbf{Buchstabe}} && \multicolumn{2}{l}{\textbf{Segment}} && \multicolumn{2}{l}{\textbf{Segment}} \\
       && \textbf{gespannt} & \textbf{Beispiel} && \textbf{ungespannt} & \textbf{Beispiel} \\
      \midrule
      i  && i  & \textit{Igel} && ɪ & \textit{Licht} \\
      ü  && y  & \textit{Rübe} && ʏ & \textit{Rücken} \\
      u  && u  & \textit{Mut} && ʊ & \textit{Butter} \\
      e  && e  & \textit{Mehl} && ɛ̆ & \textit{Bett} \\
      ö  && ø & \textit{Höhle} && œ & \textit{Löffel} \\
      o  && o  & \textit{Ofen} && ɔ & \textit{Motte} \\
      ä  && ɛ  & \textit{Gräte} && ɛ̆ & \textit{Säcke} \\
      a  && a  & \textit{Wal} && ă & \textit{Wall} \\
      \bottomrule
    \end{tabular}
  }
    \Zeile
    \pause
    \begin{itemize}[<+->]
      \item \alert{für gespannte\slash ungespannte Vokalpaare nur je ein Zeichen}
      \item außerdem \textit{e}→/ɛ̆/ und \textit{ä}→/ɛ̆/
      \item "`speter"'-Dialekte zusätzlich \textit{e}→/e/ und \textit{ä}→/e/
        \Halbzeile
      \item \alert{Diphthonge} brechen zusätzlich das phonematische Prinzip (s.\ Buch)
    \end{itemize}
\end{frame}

\begin{frame}
  {Gründe für das System der Vokalzeichen}
  \pause
  \begin{itemize}[<+->]
    \item im Kern: \alert{starke Kopplung von Gespanntheit, Länge und Betonung}
    \item nahe an \alert{einer zugrundeliegenden Form} für Gespanntheitspaare
    \item zusammen mit \alert{Silbengelenkschreibung} (s.\,u.) daher\\
      kaum Bedarf an graphematischer Differenzierung
      \Halbzeile
    \item außerdem Entwicklung von \alert{Dehnungsschreibungen}\\
      zur Desambiguierung
    \item \ldots weil \alert{Länge + Akzent → Gespanntheit}
      \Halbzeile
    \item trotzdem suboptimal
  \end{itemize}
\end{frame}


\section{Dehnung und Schärfung}

\begin{frame}
  {Das Kreuz mit der Dehnungsschreibung}
  \pause
  \begin{itemize}[<+->]
    \item Dehnungs-\textit{h} (\textit{Reh}, \textit{Pfahl}) oder Dehnungs-Doppelvokal (\textit{Saat}, \textit{Boot})
    \item speziell bei \textit{i} (dort fast immer): Dehnungs-\textit{e} (\textit{Knie}, \textit{Dieb})
      \Halbzeile
    \item \alert{weitgehend redundant} (erst recht im Kern)
    \item \alert{unsystematisch} (\textit{Lid}, \textit{Lied} usw.)
      \Halbzeile
    \item mangels Systematik: \alert{oft Erwerbsprobleme}
    \item \ldots denen kaum systematisch zu begenen ist
  \end{itemize}
\end{frame}

\newcommand{\LocStrutGrph}{\hspace{0.1\textwidth}}
\newcommand{\Nono}{---}

\begin{frame}
  {Das Faszinosum der Schärfungsschreibung}
  \pause
  Dehnungs-\slash Schärfungsschreibungen (Einsilbler\slash trochäischer Zweisilbler)\\
  \Zeile
  \pause
  \centering
  \resizebox{0.85\textwidth}{!}{
    \begin{tabular}{lllllllll}
      \toprule
      & & & \textbf{ɪ} & \textbf{ʊ} & \multicolumn{2}{l}{\LocStrutGrph\textbf{ɛ̆}} & \textbf{ɔ} & \textbf{ă} \\
      \midrule

      \multirow{4}{*}{\rotatebox{90}{\textbf{ungespannt}}}

      & \multirow{2}{*}{\rotatebox{90}{\textbf{offen}}}
      & \textbf{einsilb.}  & \textit{\Nono}  & \textit{\Nono}           & \multicolumn{2}{l}{\LocStrutGrph\textit{\Nono}}         & \textit{\Nono}        & \textit{\Nono}           \\
      && \textbf{zweisilb.}  & \textit{Li.\alert{pp}e} & \textit{Fu.\alert{tt}er}         & \multicolumn{2}{l}{\LocStrutGrph\textit{We.\alert{ck}e}}        & \textit{o.\alert{ff}en}       & \textit{wa.\alert{ck}er}         \\
        & \multirow{2}{*}{\rotatebox{90}{\textbf{gesch.}}}
        & \textbf{einsilb.}  & \textit{Ki\rot{nn}}   & \textit{Schu\rot{tt}}    & \multicolumn{2}{l}{\LocStrutGrph\textit{Be\rot{tt}}}           & \textit{Ro\rot{ck}}         & \textit{Wa\rot{tt}}            \\
        && \textbf{zweisilb.}  & \textit{Rin.de} & \textit{Wun.der}        & \multicolumn{2}{l}{\LocStrutGrph\textit{Wen.de}}        & \textit{pol.ter}      & \textit{Tan.te}          \\

      \midrule

      \multirow{4}{*}{\rotatebox{90}{\textbf{gespannt}}}

      & \multirow{2}{*}{\rotatebox{90}{\textbf{offen}}}
        & \textbf{einsilb.}  & \textit{Knie}   & \textit{Schuh}       & \textit{Schnee, Reh}  & \textit{zäh}          & \textit{roh}          & (\textit{da})            \\
      && \textbf{zweisilb.}  & \textit{Bie.ne} & \textit{Kuh.le, Schu.le} & \textit{we.nig}       & \textit{Äh.re, rä.kel} & \textit{oh.ne, O.fen} & \textit{Fah.ne, Spa.ten} \\

      & \multirow{2}{*}{\rotatebox{90}{\textbf{gesch.}}}
        & \textbf{einsilb.}  & \textit{lieb}  & \textit{Ruhm, Glut}      & \textit{Weg}          & \textit{spät}           & \textit{rot}          & \textit{Tat}             \\
      && \textbf{zweisilb.}  & (\textit{lieb.lich}) & (\textit{lug.te})   & (\textit{red.lich})   & (\textit{wähl.te})     & (\textit{brot.los})   & (\textit{rat.los})       \\

      \midrule
      & & & \textbf{i} & \textbf{u} & \textbf{e} & \textbf{ε} & \textbf{o} & \textbf{a} \\

      \bottomrule
    \end{tabular}
  }
  \Halbzeile\pause
  \begin{itemize}[<+->]
    \item \alert{Schärfungsschreibung im Trochäus nur nach ungespanntem Vokal\\
      in offener Silbe, wenn Anfangsrand der Zweitsilbe konsonantisch}
    \item (\ldots und im geschlossenen Einsilbler mit ungespannten Vokal)
  \end{itemize}
\end{frame}

\begin{frame}
  {Details und oft Übersehenes}
  \pause
  \begin{itemize}[<+->]
    \item \alert{Schärfungsschreibung = Silbengelenkschreibung}
    \item Aber warum dann im Einsilbler (\textit{Kinn}, \textit{Bett}, \textit{Rock})?
      \begin{itemize}[<+->]
        \item Siehe nächste Woche!
      \end{itemize}
      \Halbzeile
    \item Merke: Silbengelenkschreibung nur da, wo auch Silbengelenk:
      \begin{itemize}[<+->]
        \item \alert{zwischen Erst- und Zweitsilbe des Trochäus}
        \item \alert{nach ungespanntem} (=kurzem) \alert{Vokal}
      \end{itemize}
      \Halbzeile
    \item \alert{keine Schärfungsschreibung bei Di- und Trigraphen}
      \begin{itemize}[<+->]
        \item \textit{Esche} [ɛʃ̣ə], \textit{zischen} [t͡sɪʃ̣ən]
        \item \textit{Kachel} [kaχ̣əl], \textit{Zeche} [t͡sɛç̣ə]
        \item \textit{Kringel} [kʁɪŋ̣əl], \textit{Zunge} [t͡sʊŋ̣ə]
      \end{itemize}
      \Halbzeile
    \item \rot{Warum sind stimmhaften Obstruenten im Silbengelenk unmöglich?}
      \begin{itemize}[<+->]
        \item Obstruent auch im Endrand der Erstsilbe: \alert{Endrand-Desonorisierung}
        \item \textit{Kladde}, \textit{Robbe}, \textit{Bagger}, ?\textit{prasseln} [pʁazəln], *\textit{quivveln}
        \item \ldots \rot{nicht Kern} (fünf oder sechs Typen, alle niederdeutsch)
      \end{itemize}
  \end{itemize}
\end{frame}

\begin{frame}
  {Eszett: Warum ist mir das wichtig, und worum gehts?}
  \pause
  \begin{itemize}[<+->]
    \item Problem für manche Schreiber*innen
    \item herrliches Beispiel für reduktionistische Methode
    \item theorieinterne deduktive Argumentation (= Wissenschaft)
    \item Eliminierung des zugrundeliegenden /s/
      \Halbzeile
    \item immerhin: erhebliche \alert{Systemstraffung} durch Orthographiereform!
      \Halbzeile
    \item Erinnerung: Verteilung von /s/ und /z/
      \begin{itemize}[<+->]
        \item Wortanfang: nur /z/ (\textit{Sog} [zoːk], niemals *[soːk])
        \item Wortauslaut: nur /s/ (\textit{Mus} [muːs], niemals *[muːz])
        \item \alert{im Wortinneren nach ungespanntem Vokal: nur /s/ (\textit{Masse} [maṣə])}
        \item \rot{im Wortinneren nach gespanntem Vokal:\\
            /s/ (\textit{Straße} [ʃtʁaːsə]) und /z/ (\textit{Hase} [haːzə])}
      \end{itemize}
  \end{itemize}
\end{frame}


\newcommand{\phopro}{\ensuremath{\Rightarrow}}

\begin{frame}
  {Analyse des Eszett}
  \pause
  \begin{itemize}[<+->]
    \item \alert{Alle Positionen bis auf die \textit{ß}-Umgebung sind herleitbar:}
      \begin{itemize}[<+->]
        \item Wortanlaut (\textit{Sog} [zoːk]): zugrundeliegendes /z/ bleibt [z]
        \item Wortauslaut (\textit{Mus} [muːs]): zugrundeliegendes /z/ würde sowieso [s]\\
          wegen Endrand-Desonorisierung
        \item Wortinneren nach ungespanntem Vokal (\textit{Masse} [maṣə]): \alert{Silbengelenk}\\
          immer stimmlos wegen Endranddesonorisierung (/măzə/ denkbar)
      \end{itemize}
      \Halbzeile
    \item \alert{Bis hierhin brauchen wir noch kein zugrundeliegendes /s/!}
      \Halbzeile
    \item zugrundeliegendes /s/ \rot{nur für das Wortinnere nach gespanntem Vokal}\\
      \textit{Straße} [ʃtʁaːsə] gegenüber \textit{Hase} [haːzə]
    \item \alert{Und wenn statt /s/ einfach /zz/ zugrundeliegt?}
    \item \alert{Und wenn /zz/ nach gespanntem Vokal mit \textit{ß} geschrieben wird?}
    \item also: \textit{Bußen} als /buzzən/ \phopro [buːssən]
  \end{itemize}
\end{frame}

\begin{frame}
  {Eszett-Silben und die anderen \textit{s}}
  \pause
  \centering
  {\footnotesize\textit{Busen}:}\hspace{1em}\scalebox{0.6}{%
    \begin{forest}
      for tree={s sep+=1em}
      [Phonologisches Wort, calign=first
        [Silbe, calign=last
          [Ar., ake
            [b]
          ]
          [Reim
            [Kern, ake
              [uː]
            ]
          ]
        ]
        [Silbe, calign=last
          [Ar., ake, baseline
            [z]
          ]
          [Reim, calign=first
            [Kern, ake
              [ə]
            ]
            [Er., ake
              [n]
            ]
          ]
        ]
      ]
    \end{forest}
  }~\pause\hspace{1.5em}{\footnotesize\textit{Bussen}:}\hspace{1em}\scalebox{0.6}{%
    \begin{forest}
      for tree={s sep+=1em}
      [Phonologisches Wort, calign=first
        [Silbe, calign=last
          [Ar., ake
            [b]
          ]
          [Reim, calign=first
            [Kern, ake
              [ʊ]
            ]
            [Er., ake, name=BusenEr]
          ]
        ]
        [Silbe, calign=last
          [Ar., ake, baseline
            [s]
            {\draw[-] (.north) -- (BusenEr.south);}
          ]
          [Reim, calign=first
            [Kern, ake
              [ə]
            ]
            [Er., ake
              [n]
            ]
          ]
        ]
      ]
    \end{forest}
  }\\
  \pause
  \Zeile
  {\footnotesize\textit{Bußen} mit \alert{Auslautverhärtung} und \orongsch{Assimilation}:}\hspace{1em}\scalebox{0.6}{%  
    \begin{forest}
      for tree={s sep+=1em}
      [Phonologisches Wort, calign=first
        [Silbe, calign=last
          [Ar., ake
            [b]
          ]
          [Reim, calign=first
            [Kern, ake
              [uː]
            ]
            [Er., ake
              [\alert{\textbf{s}}]
            ]
          ]
        ]
        [Silbe, calign=last
          [Ar., ake, baseline
            [\orongsch{\textbf{s}}]
          ]
          [Reim, calign=first
            [Kern, ake
              [ə]
            ]
            [Er., ake
              [n]
            ]
          ]
        ]
      ]
    \end{forest}
  }
\end{frame}


\begin{frame}
  {Schritt für Schritt}
  \pause
  \begin{enumerate}[<+->]
    \item zugrundeliegende Form: \alert{/buzzən/}
    \item Silbifizierung \phopro \{buz\orongsch{.}zən\}
    \item Längung gespannter Vokale \phopro \{b\orongsch{uː}z.zən\}
    \item Endranddesonorisierung \phopro \{buː\orongsch{s}.zən\}
    \item Assimilation des Anfangsrands \phopro \alert{[buːs.}\orongsch{s}\alert{ən]}
  \end{enumerate}
  \pause
  \begin{itemize}[<+->]
    \item Ist die Assimilation ein Taschenspielertrick?
    \item Nein, denn sie findet auch in anderen Fällen statt!
  \end{itemize}
  \pause
  \begin{exe}
    \ex\label{ex:dehnungsundschaerfungsschreibungen024}
    \begin{xlist}
      \ex{\label{ex:dehnungsundschaerfungsschreibungen025} /ɛ̆k\alert{z}ə/ \phopro\ [ʔɛk.\orongsch{s}ə] (\textit{Echse})}
      \pause
      \ex{\label{ex:dehnungsundschaerfungsschreibungen026} /ɛ̆ʁb\alert{z}e/ \phopro\ [ʔɛ͡əp.\orongsch{s}ə] (\textit{Erbse})}
    \end{xlist}
  \end{exe}
  \pause
  \begin{itemize}[<+->]
    \item Also ist das Konsonantenzeichen \textit{s} \rot{nicht} doppelt belegt.
    \item \alert{Es gibt zugrundeliegend nur /z/.}
  \end{itemize}
\end{frame}


\begin{frame}
  {Achtung: Grenz-\textit{h}: weder Dehnung noch Segment}
  \pause
  \begin{exe}
    \ex wehe /veə/
    \pause
    \ex Ruhe /ʁuə/
    \pause
    \ex fliehe /fliə/
    \pause
    \ex Krähe /kʁɛə/
  \end{exe}
  \pause
  \begin{itemize}[<+->]
    \item keine Dehnungsschreibung, siehe \textit{fliehe}
    \item \alert{Silbengrenzenanzeiger} zwischen Vokalen
      \Halbzeile
    \item Ausnahme: nach Diphthong steht Grenz-\textit{h} nicht\\
      (\textit{Reue}, \textit{Kleie}, \textit{Schreie}, \textit{Säue})
    \item bis auf Ausnahmen (\textit{verzeihen}, \textit{leihen}, \textit{Reihe}, \textit{Weiher})
  \end{itemize}
\end{frame}


\section{Vorschau}

\begin{frame}
  {Wortschreibungen}
  \pause
  \begin{itemize}[<+->]
    \item Prinzip der Spatienschreibung
    \item Prinzip der positionsabhängigen Majuskelschreibung
    \item Prinzip der \alert{Konstantschreibung}
      \Halbzeile
    \item kurz zu den Interpunktionszeichen
      \Halbzeile
    \item Da bleibt noch Zeit\ldots
    \item Mal sehen, wofür die genutzt wird.
  \end{itemize}
  \pause
  \Halbzeile
  \begin{center}
    Bitte lesen Sie bis nächste Woche:\\
    \alert{Kapitel 16 (S.~495--515)}
  \end{center}
\end{frame}

\begin{frame}<handout:0>[noframenumbering]
  {Exklusiv nur in der VL und wenn Sie dageblieben sind\ldots}
  \pause
  Teil 1 der Klausurtipps:\\
  \Halbzeile
  \pause
  \rot{Sie sollten sich folgende Themen dringend genauer ansehen!}
  \Halbzeile
  \pause
  \begin{itemize}
    \item<4> phonologische Prozesse (grob)
    \item<5> \alert{Strukturbedingungen der Silbifizierung}
    \item<6> die wichtigen Wortklassenfilter (je höher im Baum, desto wichtiger),\\
      \zB Nomen vs.\ Verb, Partikel vs.\ Adverb
    \item<7> Komposition und Derivation
    \item<8> Flexionssysteme der verschiedenen Nomina (Subst, Art, Pron, Adj)
    \item<9> Phrasentypen (besondere, nicht ausschließliche Beachtung: NP, VP)
    \item<10> Form von (Neben-)Satztypen und Bewegung
    \item<11> \alert{Valenzänderungen aller Art} (auch Dative)
    \item<12> Ergänzungen, Angaben und Valenz (auch Kapitel 2!)
    \item<13> Schreibprinzipien für Segmentschreibungen
  \end{itemize}
\end{frame}


\begin{frame}<handout:0>[noframenumbering]
  {Wortschreibungen}
  \begin{itemize}
    \item Prinzip der Spatienschreibung
    \item Prinzip der positionsabhängigen Majuskelschreibung
    \item Prinzip der \alert{Konstantschreibung}
      \Halbzeile
    \item kurz zu den Interpunktionszeichen
      \Halbzeile
    \item Das bleibt noch Zeit\ldots
    \item Mal sehen, wofür die genutzt wird.
  \end{itemize}
  \Halbzeile
  \begin{center}
    Bitte lesen Sie bis nächste Woche:\\
    \alert{Kapitel 16 (S.~495--515)}
  \end{center}
\end{frame}

\makeatletter
\setcounter{lastpagemainpart}{\the\c@framenumber}
\makeatother

\appendix

\begin{frame}
  {Literatur}
  \renewcommand*{\bibfont}{\footnotesize}
  \setbeamertemplate{bibliography item}{}
  \printbibliography
\end{frame}

\mode<beamer>{\setcounter{framenumber}{\thelastpagemainpart}}

\end{document}
