\documentclass[handout,aspectratio=1610]{beamer}
%\documentclass[aspectratio=1610]{beamer}

%\usepackage[T1]{fontenc}
\usepackage[ngerman]{babel}
\usepackage{color}
\usepackage{colortbl}
\usepackage{textcomp}
\usepackage{multirow}
\usepackage{nicefrac}
\usepackage{multicol}
\usepackage{gb4e-}
\usepackage{verbatim}
\usepackage{cancel}
\usepackage{graphicx}
\usepackage{hyperref}
\usepackage{verbatim}
\usepackage{boxedminipage}
\usepackage{rotating}
\usepackage{booktabs}
\usepackage{bbding}

\usepackage{tikz}
\usetikzlibrary{positioning,arrows,cd}
\tikzset{>=latex}

\usepackage[linguistics]{forest}

\usepackage{FiraSans}

\usepackage[maxbibnames=99,
  maxcitenames=2,
  uniquelist=false,
  backend=biber,
  doi=false,
  url=false,
  isbn=false,
  bibstyle=biblatex-sp-unified,
  citestyle=sp-authoryear-comp]{biblatex}
\addbibresource{rs.bib}

\forestset{
  Ephr/.style={draw, ellipse, thick, inner sep=2pt},
  Eobl/.style={draw, rounded corners, inner sep=5pt},
  Eopt/.style={draw, rounded corners, densely dashed, inner sep=5pt},
  Erec/.style={draw, rounded corners, double, inner sep=5pt},
  Eoptrec/.style={draw, rounded corners, densely dashed, double, inner sep=5pt},
  Ehd/.style={rounded corners, fill=gray, inner sep=5pt,
    delay={content=\whyte{##1}}
  },
  Emult/.style={for children={no edge}, for tree={l sep=0pt}},
  phrasenschema/.style={for tree={l sep=2em, s sep=2em}},
  decide/.style={draw, chamfered rectangle, inner sep=2pt},
  finall/.style={rounded corners, fill=gray, text=white},
  intrme/.style={draw, rounded corners},
  yes/.style={edge label={node[near end, above, sloped, font=\scriptsize]{Ja}}},
  no/.style={edge label={node[near end, above, sloped, font=\scriptsize]{Nein}}},
  sake/.style={tier=preterminal},
  ake/.style={
    tier=preterminal
    },
}


\tikzset{
    invisible/.style={opacity=0,text opacity=0},
    visible on/.style={alt=#1{}{invisible}},
    alt/.code args={<#1>#2#3}{%
      \alt<#1>{\pgfkeysalso{#2}}{\pgfkeysalso{#3}} % \pgfkeysalso doesn't change the path
    },
}
\forestset{
  visible on/.style={
    for tree={
      /tikz/visible on={#1},
      edge+={/tikz/visible on={#1}}}}}

\definecolor{lg}{rgb}{.8,.8,.8}
\newcommand{\Dim}{\cellcolor{lg}}


\newcommand{\Sub}[1]{\ensuremath{_{\text{#1}}}}
\newcommand{\Up}[1]{\ensuremath{^{\text{#1}}}}

\newcommand{\Ck}{\CheckmarkBold}
\newcommand{\Fl}{\XSolidBrush}

\usetheme[hideothersubsections]{Goettingen}

%\newcommand{\rot}[1]{{\color[rgb]{0.4,0.2,0}#1}}
%\newcommand{\blau}[1]{{\color[rgb]{0,0,.9} #1}}
\newcommand{\gruen}[1]{{\color[rgb]{0,0.4,0}#1}}

\newcommand{\rot}[1]{{\color[rgb]{0.6,0.2,0.0}#1}}
\newcommand{\blau}[1]{{\color[rgb]{0.0,0.0,0.9}#1}}
\newcommand{\orongsch}[1]{{\color[RGB]{255,165,0}#1}}
\newcommand{\grau}[1]{{\color[rgb]{0.5,0.5,0.5}#1}}

\renewcommand<>{\rot}[1]{
  \alt#2{\beameroriginal{\rot}{#1}}{#1}%
}
\renewcommand<>{\blau}[1]{
  \alt#2{\beameroriginal{\blau}{#1}}{#1}%
}
\renewcommand<>{\orongsch}[1]{
  \alt#2{\beameroriginal{\orongsch}{#1}}{#1}%
}

\definecolor{trueblue}{rgb}{0,0.0,0.7}
\setbeamercolor{alerted text}{fg=trueblue}

\newcommand{\xxx}{\hspaceThis{[}}
\newcommand{\zB}{z.\,B.\ }
\newcommand{\down}[1]{\ensuremath{\mathrm{#1}}}

\newcommand{\Zeile}{\vspace{\baselineskip}}
\newcommand{\Halbzeile}{\vspace{0.5\baselineskip}}
\newcommand{\Viertelzeile}{\vspace{0.25\baselineskip}}

\resetcounteronoverlays{exx}


\title{Einführung in die Sprachwissenschaft\\
5.~Wortklassen}
\author{Roland Schäfer}
\institute{Deutsche und niederländische Philologie\\Freie Universität Berlin}
\date{Wintersemester 2018/2019\\14.~und 15.~November 2018}

\begin{document}

\frame{\titlepage}

\section{Rückblick}

\begin{frame}
  {Silbenphonologie}
  \pause
  \begin{itemize}[<+->]
    \item Silben sind nicht lexikalisch\slash zugrundeliegend.
    \item Sonorität: Öffnen und Schließen des Vokaltrakts
    \item Sonoritätskontur: Anstieg zum Vokal, dann Abfall
    \item Anfangsrand, Kern, Endrand; Reim
    \item \alert{extrasilbische} Sonoritätsverletzungen: /ʃ/, /s/, /t/
    \item prototypischer komplexe Anfangsrand: \alert{Obstruent + Liquid}
    \item prototypischer komplexe Endrand: \alert{Liquid + Obstruent}
    \item Wichtig: \alert{Das gilt für betonte Silben im Kernwortschatz.}
    \item Silbengewicht in \alert{Moren} (Vokal: eine\slash zwei, Kons.: je eine)
    \item normale Silben: zwei- oder dreimorig; Schwa-Silben: einmorig
    \item Überschwere: verhindert durch Extrasilbizität
    \item Silbengelenk: geteilter Konsonant statt überleichter Silbe im Trochäus
    \item Anfangsrandmaximierung bei Zweifelsfällen der Silbifizierung
  \end{itemize}
\end{frame}

\begin{frame}
  {Warum Reim?}
  \pause
  \begin{itemize}[<+->]
    \item Reim = Kern und Endrand
    \item \alert{Für das Silbengewicht zählt nur der Reim!}
    \item Prinzip: eigene Regularität → eigene Struktur
      \Zeile
    \item außerdem: literarischer \alert{Endreim}: \alert{Anfangsrand egal}
    \item und: literarischer \alert{Anfangsreim} (Stabreim): \alert{Silbenreim egal} 
  \end{itemize}
\end{frame}

\begin{frame}
  {Alfred Lichtenstein: Die Dämmerung}
  \pause
  Ein dicker Junge spielt mit einem \textbf{\rot<3>{T}\alert<3>{eich}}.\\
  Der Wind hat sich in einem Baum ge\textbf{\rot<4>{f}\alert<4>{an}}/gen.\\
  Der Himmel sieht verbummelt aus und \textbf{\rot<3>{bl}\alert<3>{eich}},\\
  Als wäre ihm die Schminke ausge\textbf{\rot<4>{g}\alert<4>{an}}/gen.\\
  \Zeile
  Auf lange Krücken schief herabge\textbf{\rot<5>{b}\alert<5>{ückt}}\\
  Und schwatzend kriechen auf dem Feld zwei \textbf{\rot<6>{L}\alert<6>{ah}}|me.\\
  Ein blonder Dichter wird vielleicht ver\textbf{\rot<5>{r}\alert<5>{ückt}}.\\
  Ein Pferdchen stolpert über eine \textbf{\rot<6>{D}\alert<6>{a}}|me.\\
  \Zeile
  An einem Fenster klebt ein fetter \textbf{\rot<7>{M}\alert<7>{ann}}.\\
  Ein Jüngling will ein weiches Weib be\textbf{\rot<8>{s}\alert<8>{u}}|chen.\\
  Ein grauer Clown zieht sich die Stiefel \textbf{\alert<7>{an}}.\\
  Ein Kinderwagen schreit und Hunde \textbf{\rot<8>{fl}\alert<8>{u}}|chen.\\
  \Zeile
  \footnotesize
  Aus: Pinthus, Kurt (Hrsg.). 1920. \textit{Menschheitsdämmerung}. Berlin: Rowohlt. S.~11.\\
  Mit | sind normale Silbengrenzen und mit / Silbengelenke markiert.
\end{frame}

\begin{frame}
  {Besser als die Klatschmethode}
  \pause
  \begin{itemize}[<+->]
    \item eigentlich nicht meine Aufgabe, aber\dots
      \Zeile
    \item \alert{Bewusstsein für Vokallänge und Silbengewicht}
    \item Bewusstsein für \alert{Vokallänge je nach Position}
    \item kurz\slash leicht vor Konsonat im Trochäus $\Rightarrow$ Silbengelenk, Gelenkschreibung
      \Zeile
    \item \rot{\textbf{geschriebene(!)} Formenreihen als Ausgangsbasis}, \alert{nur Kernwortschatz}
    \item Anfang mit dem \alert{Einsilbler} (möglichst ohne Dehnungsschreibung)
    \item weiter mit dem \alert{trochäischen Zweisilbler ohne Silbengelenk}
    \item schließlich \alert{Zweisilbler mit Silbengelenk}
    \item Silbengelenke \alert{ohne Doppelschreibung} zuletzt (\textit{ng}, \textit{tsch} usw.)
    \item[!] Da sind schon zwei Konsonanten!
    \item für Anfangsrandmaximierung: Parallele zum \alert{Einsilbler}
  \end{itemize}
\end{frame}

\section{Überblick}

\begin{frame}
  {Überblick}
  \pause
  \begin{itemize}[<+->]
    \item Was sind Wörter?
    \item Lexikalisches vs.\ syntaktisches Wort
      \Zeile
    \item Wozu Wortklassen?
    \item \rot{Bedeutungsklassen} und Wortklassen
    \item \alert{Morphologie} von Wortklassen
    \item \alert{Syntax} von Wortklassen
      \Zeile
    \item wichtige Wortklassen
      \begin{itemize}[<+->]
        \item Nomen
        \item Verb
        \item Präposition
        \item Komplementierer
        \item Adverb
        \item Partikel
      \end{itemize}
  \end{itemize}
\end{frame}

\begin{frame}
  {Wortklassen und Bildungssprache\slash Lehramt}
  \pause
  \begin{itemize}[<+->]
    \item direkter Einfluss von Wortklassenwissen auf\\
      bildungssprachliche Fähigkeiten: \rot{keiner}
      \Zeile
    \item Sprachbetrachtung (Woche 1):
      \begin{itemize}[<+->]
        \item \rot{Form} → \alert{Funktion}
        \item \rot{systematisch}, also basierend auf \alert{Generalisierungen}
        \item essentiell für formale Generalisierungen: \alert{Wortklassen}
      \end{itemize}
      \Zeile
    \item Normfragen und Wortklassenbezug
      \begin{itemize}
        \item \alert{Substantivgroßschreibung}
        \item Nebensätze: Komplementierer, Pronomina, Kommas
        \item Flexion (Problemfälle: Konjunktiv, Adjektive usw.)
        \item \dots alles nicht ohne Wortklassen beschreibbar
      \end{itemize}
  \end{itemize}
\end{frame}

\section{Wörter}

\begin{frame}
  {Ebenen und Einheiten}
  \pause
  \begin{itemize}[<+->]
    \item Wortakzent: \textit{\alert{\textbf{Sie}}ges\alert{säu}le}\\
      $\rightarrow$ phonologisches\slash prosodisches Wort
      \Zeile
    \item Eigenschaften von Wörtern jenseits der Phonologie?
  \end{itemize}
  \Zeile
  \pause
  \begin{exe}
    \ex
    \begin{xlist}
      \ex[]{Staat-es}
      \pause
      \ex[*]{Tür-es}
    \end{xlist}
    \pause
    \Zeile
    \ex
    \begin{xlist}
      \ex[]{Der Satz ist eine grammatische Einheit.}
      \pause
      \ex[*]{Die Satz ist eine grammatische Einheit.}
    \end{xlist}
  \end{exe}
\end{frame}

\begin{frame}
  {Wört haben eine Bedeutung?}
  \pause
  \begin{exe}
    \ex \alert{Es} \alert{wird} schon wieder früh dunkel.
    \pause
    \ex Kristine denkt, \alert{dass} \alert{es} bald regnen \alert{wird}.
    \pause
    \ex Adrianna \alert{hat} gestern \alert{den} Keller inspiziert.
    \pause
    \ex Camilla \alert{und} Emma sehen \alert{sich} \alert{die} Fotos \alert{an}.
  \end{exe}
  \Zeile
  \pause
  \large
  \alert{Bedeutungstragende Wörter und Funktionswörter}
\end{frame}

\begin{frame}
  {Morphologie und Syntax}
  \pause
  \begin{itemize}[<+->]
    \item Kombinatorik für \alert{Wortbestandteile}: Morphologie
      \begin{itemize}[<+->]
        \item Wortbestandteile \zB mit \alert{Umlaut}: \textit{rot} -- \textit{röter}
        \item oder \alert{Ablaut}: \textit{heben} -- \textit{hob}
      \end{itemize}
    \item Kombinatorik für \alert{Wörter}: Syntax
      \Zeile
    \item \alert{Zirkuläre oder leere Definitionen?}
    \item \rot{Nein!} Prinzip: eigene Regularität → eigene Struktur
      \Zeile
    \item Wortbestandteile \alert{nicht trennbar}:
      \begin{itemize}
        \item \textit{heb-t}\\
          *\textit{heb mit Mühe t}
        \item \textit{Ge-hob-en-heit} \\
          *\textit{Gehoben anspruchsvolle heit}
        \item \textit{Sie geht schnell heim.}\\
          \textit{Schnell geht sie heim.}
      \end{itemize}
  \end{itemize}
\end{frame}

\begin{frame}
  {Wort und Wortform}
  \pause
  \begin{exe}
    \ex
    \begin{xlist}
      \ex (der) Tisch
      \pause
      \ex (den) Tisch
      \pause
      \ex (dem) Tisch\alert{e}
      \pause
      \ex (des) Tisch\alert{es}
      \pause
      \ex (die) Tisch\alert{e}
      \pause
      \ex (den) Tisch\alert{en}
    \end{xlist}
  \end{exe}
  \pause
  \begin{exe}
    \ex
    \begin{xlist}
      \ex Der \_\_\_\ ist voll hässlich.
      \pause
      \ex Ich kaufe den \_\_\_ nicht.
      \pause
      \ex Wir speisten am \_\_\_\ des Bundespräsidenten.
      \pause
      \ex Der Preis des \_\_\_\ ist eine Unverschämtheit.
      \pause
      \ex Die \_\_\_\ kosten nur noch die Hälfte.
      \pause
      \ex Mit den \_\_\_\ können wir nichts mehr anfangen.
    \end{xlist}
  \end{exe}
\end{frame}

% \begin{frame}
%   {Wort und Wortform II}
%   \pause
%   \large
%   Eine \alert{Wortform} ist eine in syntaktischen Strukturen auftretende und in diesen Strukturen nicht weiter zu unterteilende Einheit.
%   Die Werte der Merkmale von Wortformen sind gemäß ihrem Paradigma vollständig spezifiziert.\\
%   \Zeile
%   \pause
%   Das (\alert{lexikalische}) \alert{Wort} ist eine Repräsentation von paradigmatisch zusammengehörenden Wortformen.
%   Für das lexikalische Wort sind die Werte nur für diejenigen Merkmale spezifiziert, die in allen Wortformen des Paradigmas dieselben Werte haben.
%   Die restlichen Werte werden gemäß der Position im Paradigma bei den konkret vorkommenden Wortformen des Wortes gesetzt.
% \end{frame}

\section{Methode}

\begin{frame}
  {Klassische Grundschul-Wortarten.}
  \pause
  \begin{itemize}[<+->]
    \item Dingwort
    \item Tuwort, Tätigkeitswort
    \item Wiewort, Eigenschaftswort
    \item Umstandswort
      \Halbzeile
    \item Noch besser die Vermittlungsversuche:
      \begin{itemize}[<+->]
        \item Dingwörter kann man anfassen. \onslide<8->{\rot{D'oh!}}
          \pause
        \item Wie ist die Kanzlerin? -- Katatonisch.
        \item Was macht Johanna? -- Laufen.
        \item Wie, wo oder warum schläft Johanna? -- Ruhig.
      \end{itemize}
    \Halbzeile
    \item Wieso auch nicht?
      \begin{itemize}[<+->]
        \item Anfassen? Wolken, Ideen, Steckdosen, Rasierklingen, \dots
        \item *Die Kanzlerin ist ehemalig.
        \item Was macht Johanna? -- Hausaufgaben.
        \item Was tut Johanna? -- *Verlaufen. \slash *Unterliegen.
        \item *Was macht\slash tut das Yoghurt? -- Verschimmeln.
        \item Wie schläft Johanna? -- *Erstaunlicherweise.
      \end{itemize}
  \end{itemize}
\end{frame}

\begin{frame}
  {Ein paar neue Wortarten nach Bedeutungen I}
  \pause
  \begin{itemize}[<+->]
    \item "`Wie, wo, warum?"' \onslide<3->{--- Warum eigentlich nicht drei Wortarten?}
      \Halbzeile
      \pause
    \item \alert{Bewegungsverben}: \textit{laufen}, \textit{springen}, \textit{fahren}, \dots
    \item \alert{Zustandsverben}: \textit{duften}, \textit{wohnen}, \textit{liegen}, \dots
      \Halbzeile
    \item \alert{Konkreta}: \textit{Haus}, \textit{Buch}, \textit{Blume}, \textit{Stier}, \dots
    \item \alert{Abstrakta}: \textit{Konzept}, \textit{Glaube}, \textit{Wunder}, \textit{Kausalität}, \dots
    \item \alert{Zählsubstantive}: \textit{Kumquat}, \textit{Student*in}, \textit{Mikrobe}, \textit{Kneipe}, \dots
    \item \alert{Stoffsubstantive}: \textit{Wasser}, \textit{Wein}, \textit{Zement}, \textit{Mehl}, \dots
  \end{itemize}
\end{frame}

\begin{frame}
  {Ein paar neue Wortarten nach Bedeutungen II}
  \pause
  Aber Moment mal\dots\\
  \pause
  \Zeile
  \begin{exe}
    \ex
    \begin{xlist}
      \ex \alert{Wein} kann lecker sein.
      \ex \alert{Kumquats können}\slash \alert{Eine Kumquat kann} lecker sein.
    \end{xlist}
%     \pause
%     \ex
%     \begin{xlist}
%       \ex Ein Glas \alert{guter Wein}\slash\alert{guten Weins} kostet 10€.
%       \ex Ein Glas \alert{?gute Kumquats}\slash\alert{guter Kumquats} kostet 4€.
%     \end{xlist}
%     \pause
%     \ex
%     \begin{xlist}
%       \ex Johanna hätte gerne \alert{eine Kumquat}.
%       \ex Johanna hätter gerne \alert{einen Wein}.
%     \end{xlist}
  \end{exe}
  \pause
  \Zeile
  Es gibt hier durchaus auch \alert{formale} Unterschiede.
\end{frame}

\begin{frame}
  {Morphologische Klassifikation}
  \pause
  \begin{exe}
    \ex
    \begin{xlist}
      \ex{Ich pfeif\alert{e}.\\
      Du pfeif\alert{st}.\\
      Die Schiedsrichterin pfeif\alert{t}.}
        \pause
        \ex{Ich schlaf\alert{e}.\\
        {Du schl\rot{ä}f\alert{st}.}\\
        Die Schiedsrichterin schl\rot{ä}f\alert{t}.}
    \end{xlist}
        \pause
    \ex
    \begin{xlist}
      \ex{der Berg\\
        des Berg\alert{es}\\
        die Berg\alert{e}}
        \pause
      \ex{der Mensch\\
        des Mensch\alert{en}\\
        die Mensch\alert{en}}
        \pause
      \ex{der Staat\\
        des Staat\alert{es}\\
        die Staat\alert{en}}
    \end{xlist}
  \end{exe}
\end{frame}

\begin{frame}
  {Achtung!}
  \pause
  \alert{Änderung der Paradigmenzugehörigkeit} eines Wortes:
  \pause
  \Zeile
  \begin{exe}
    \ex\label{ex:paradigmatischeklassifikation017}\begin{xlist}
      \ex{Wir sind des \alert{Wanderns} müde.}
      \pause
      \ex{Wir \alert{wandern}.}
    \end{xlist}
  \end{exe}
  \pause
  \Zeile
  $\Rightarrow$ \rot{Zwei verschiedene} lexikalische Wörter.
\end{frame}


\begin{frame}
  {Syntaktische Klassifikation}
  \pause
  \begin{exe}
    \ex
    \begin{xlist}
      \ex[]{Alexandra spielt schnell \alert{und} präzise.}
      \pause
      \ex[*]{Alexandra spielt schnell \alert{obwohl} präzise.}
      \pause
      \ex[]{Alexandra \alert{und} Dzsenifer spielen eine gute Saison.}
      \pause
      \ex[*]{Alexandra \alert{obwohl} Dzsenifer spielen eine gute Saison.}
    \end{xlist}
    \pause
    \Zeile
    \ex
    \begin{xlist}
      \ex[]{Alexandra spielt herausragend,\\
        \alert{obwohl} der Leistungsdruck hoch ist.}
      \pause
      \ex[*]{Alexandra spielt herausragend, \alert{und} der Leistungsdruck hoch ist.}
    \end{xlist}
  \end{exe}
    \pause
    \Zeile
    Alles nur wegen der Bedeutung?
    \pause
  \begin{exe}
    \ex Der Marmorkuchen spielt schnell \alert{und} präzise.
  \end{exe}
\end{frame}

\begin{frame}[fragile]
  {Filter}
  \begin{itemize}
    \item<2-> Kapitel 2: \alert{Kategorien} definiert über Merkmale und Werte.
      \begin{itemize}[<+->]
        \item<3-> Hat \textsc{Numerus} oder nicht?
        \item<4-> Hat \textsc{Genus} oder nicht?
        \item<5-> \dots
      \end{itemize}
  \end{itemize}
  \begin{center}
    \begin{forest}
      /tikz/every node/.append style={font=\footnotesize},
      for tree={l sep=2em, s sep=2.5em},
      [\textit{Wort}, intrme, {visible on=<6->}, for children={visible on=<7->}
        [{Hat  Numerus?}, decide, for children={visible on=<8->}
          [\textit{flektierbar}, intrme, yes, {visible on=<9->}, for children={visible on=<11->}
            [{Ist finit  flektierbar?}, decide, {visible on=<11->}, for children={visible on=<12->}
              [\textbf{Verb}, finall, yes, {visible on=<13->}]
              [\textit{Nomen}, intrme, no, {visible on=<14->}]
            ]
          ]
          [\textit{nicht flektierbar}, intrme, no, {visible on=<10->}, for children={visible on=<15->}
            [{Hat Valenz-\slash  Kasusrektion?}, decide, {visible on=<15->}, for children={visible on=<16->}
              [\textbf{Präposition}, finall, yes, {visible on=<17->}]
              [\textit{andere}, intrme, no, {visible on=<18->}]
            ]
          ]
        ]
      ]
    \end{forest}
  \end{center}
\end{frame}


\section{Wortklassen}

\begin{frame}
  {Flektierbare Wörter: Numerus}
  \pause
  \begin{exe}
    \ex
    \begin{xlist}
      \ex Tüte, Tüten
      \pause
      \ex Baum, Bäume
    \end{xlist}
    \pause
    \ex
    \begin{xlist}
      \ex (ich) gehe, (wir) gehen
      \pause
      \ex (du) gehst, (ihr) geht
    \end{xlist}
    \Zeile
    \pause
    \ex
    \begin{xlist}
      \ex \alert<12->{Ein} \alert<13->{roter} \alert<8->{Apfel} \alert<9->{hängt} am Baum.
      \pause
      \ex \alert<14->{Rote} \alert<10->{Äpfel} \alert<11->{hängen} am Baum.
    \end{xlist}
  \end{exe}
  \Zeile
  \pause
  \pause
  \pause
  \pause
  \pause
  \pause
  \pause
  \pause
  Als \alert{Kongruenzmerkmal} ist Numerus in der Definition\\
  der flektierbaren Wortklassen \alert{strukturell motiviert}.
\end{frame}

\begin{frame}
  {Substantive vs.\ Nomina}
  \pause
  \begin{exe}
    \ex \alert<5->{Die stärkste} Gewichtheberin wurde Weltmeisterin.
    \pause
    \ex \alert<5->{Der stärkste} Versuch war der zweite.
    \pause
    \ex \alert<5->{Das höchste} Gewicht wurde von Tatjana gerissen.
  \end{exe}
  \Zeile
  \pause
  \pause
  \begin{itemize}[<+->]
    \item Substantive: festes Genus
    \item andere Nomina (Artikel\slash Pronomen, Adjektiv):\\
      \alert{Genuskongruenz mit dem Substantiv}
  \end{itemize}
\end{frame}

\begin{frame}
  {Adjektive}
  \pause
  \begin{exe}
    \ex
    \begin{xlist}
      \ex Kein \alert<3->{großer} Ball wurde gespielt.
      \ex Der \alert<3->{große} Ball wurde gespielt.
    \end{xlist}
    \pause
    \pause
    \ex
    \begin{xlist}
      \ex Keine \alert<5->{großen} Bälle wurden gespielt.
      \ex Die \alert<5->{großen} Bälle wurden gespielt.
      \ex Große \alert<5->{Bälle} wurden gespielt.
    \end{xlist}
  \end{exe}
  \Zeile
  \pause
  \pause
  \centering
  \resizebox{0.45\textwidth}{!}{
    \begin{tabular}{lllllll}
      \toprule
      \multicolumn{3}{l}{} & \textbf{Mask} & \textbf{Neut} & \textbf{Fem} & \textbf{Pl} \\
      \midrule
      \multirow{4}{*}{\textbf{stark}} & \textbf{Nom} & \multirow{4}{*}{heiß-} & er & es & e & e \\
      & \textbf{Akk} && en & es & e & e \\
      & \textbf{Dat} && em & em & er & en \\
      & \textbf{Gen} && en & en & er & er \\
      \midrule
      \multirow{4}{*}{\textbf{schwach}} & \textbf{Nom} & \multirow{4}{*}{(der) heiß-} & e & e & e & en \\
      & \textbf{Akk} && en & e & e & en \\
      & \textbf{Dat} && en & en & en & en \\
      & \textbf{Gen} && en & en & en & en \\
      \midrule
      \multirow{4}{*}{\textbf{gemischt}} & \textbf{Nom} & \multirow{4}{*}{(kein) heiß-} & er \Dim & es \Dim & e & en \\
      & \textbf{Akk} && en & es \Dim & e & en \\
      & \textbf{Dat} && en & en & en & en \\
      & \textbf{Gen} && en & en & en & en \\
      \bottomrule
    \end{tabular}
  }
\end{frame}

\begin{frame}
  {Präpositionen}
  \pause
  \begin{exe}
    \ex
    \begin{xlist}
      \ex{\alert<3-4>{Mit} \alert<4>{dem kaputten Rasen} ist nichts mehr anzufangen.}
      \pause
      \pause
      \pause
      \ex{\alert<6-7>{Angesichts} \alert<7>{des kaputten Rasens} wurde das Spiel abgesagt.}
    \end{xlist}
  \end{exe}
  \pause
  \pause
  \pause
  \Zeile
  \large
  In einer \alert{Rektionsrelation} werden durch die regierende Einheit (das \alert{Regens}) Werte für bestimmte Merkmale (und ggf.\ auch die Form) beim regierten Element (dem \alert{Rectum}) verlangt.\\
  \Zeile
  \pause
  \alert{Präpositionen} kasusregieren eine obligatorische Nominalphrase.
\end{frame}

\begin{frame}
  {Komplementierer}
  \pause
  \begin{exe}
    \ex
    \begin{xlist}
      \ex[]{Ich glaube, [\alert<3->{dass} dieser Nebensatz ein Verb \alert<4->{enthält}].}
      \ex[]{[\alert<6->{Während} die Spielzeit \alert<7->{läuft}], zählt jedes Tor.}
      \ex[]{Es fällt ihnen schwer [\rot<8->{zu laufen}].}
      \ex[\rot<11->{*}]{[\alert<9->{Obwohl} kein Tor \alert<10->{fiel}].}
    \end{xlist}
  \end{exe}
  \Zeile
  \pause
  \pause
  \pause
  \pause
  \pause
  \pause
  \pause
  \pause
  \pause
  \pause
  \large
  \alert{Komplementierer} leiten Nebensätze ein.\\
  Die Rede von der \textit{unterordnenden Konjunktion} ist Unsinn.
\end{frame}

\begin{frame}
  {Nicht-flektierbare Wörter im Vorfeld}
  \pause
  Was steht im unabhängigen Aussagesatz am Satzanfang?\\
  \pause
  {\rot{Antworten Sie nie mehr mit "`das Subjekt"'!}}
  \pause
  \begin{exe}
    \ex\label{ex:adverbenadkopulasundpartikeln038}
    \begin{xlist}
      \ex[ ]{\alert<5->{Gestern} hat der FCR Duisburg gewonnen.}
      \pause
      \pause
      \ex[ ]{\alert<7->{Erfreulicherweise} hat der FCR Duisburg gestern gewonnen.}
      \pause
      \pause
      \ex[ ]{\alert<9->{Oben} finden wir andere Beispiele.}
      \pause
      \pause
      \ex[*]{\alert<11->{Doch} ist das aber nicht das Ende der Saison.}
      \pause
      \pause
      \ex[*]{\alert<13->{Und} ist die Saison zuende.}
      \pause
      \pause
    \end{xlist}
    \ex\label{ex:adverbenadkopulasundpartikeln044} Das ist aber \alert{doch} nicht das Ende der Saison.
  \end{exe}
  \pause
  \Zeile
  \alert{Adverben} sind die übriggebliebenen nicht-flektierbaren Wörter,\\
  die im Vorfeld stehen können.
\end{frame}

\begin{frame}
  {Adkopulas}
  \pause
  Kopulas: \textit{sein}, \textit{bleiben}, \textit{werden}\\
  \pause
  Spezielle Klasse von Hilfsverben\dots
  \pause
  \Zeile
  \begin{exe}
    \ex
    \begin{xlist}
      \ex[ ]{Hamlet \alert<6->{ist} \alert<5->{meschugge}.}
      \pause
      \pause
      \pause
      \ex[ ]{\alert<8->{Quitt} \alert<9->{bin} ich mit dir noch lange nicht.}
      \pause
      \pause
      \pause
    \end{xlist}
    \Zeile
    \ex
    \begin{xlist}
      \ex[ ]{Tatjana \alert<11->{ist} \alert<12->{stark}.}
      \pause
      \pause
      \pause
      \ex[ ]{Die \alert<14->{starke} \alert<15->{Gewichtheberin} ist Weltmeisterin.}
      \pause
      \pause
      \pause
    \end{xlist}
    \Zeile
    \ex
    \begin{xlist}
      \ex[ ]{Der Staat \alert<18->{ist} \alert<17->{pleite}.}
      \pause
      \pause
      \pause
      \ex[*]{Der \alert<20->{pleite} \alert<21->{Staat} bricht zusammen.}
      \pause
      \pause
      \pause
    \end{xlist}
  \end{exe}
  \Zeile
  \large
  \alert{Adkopulas} treten immer in Abhängigkeit einer Kopula auf.
\end{frame}

\begin{frame}
  {Konjunktionen}
  \pause
  \begin{exe}
    \ex\label{ex:konjunktionen052}
    \begin{xlist}
      \ex{[Dzsenifer] \alert<3->{und} [eine andere Spielerin] haben Tore geschossen.}
      \pause
      \pause
      \ex{Sätze können wir [aufschreiben] \alert<5->{oder} [aussprechen].}
      \pause
      \pause
      \ex{Spielt bitte [konzentriert] \alert<7->{und} [offensiv].}
      \pause
      \pause
    \end{xlist}
  \end{exe}
  \Zeile
  \large
  \alert{Konjunktionen} verbinden Satzteile der gleichen Kategorie.\\
  Die Rede von der \textit{neben-}\slash\textit{beiordnenden Konjunktion} ist Unsinn.
\end{frame}

\begin{frame}[fragile]
  {"`Alle Wortklassen"'}
  \pause
  \begin{center}
    \scalebox{0.4}{
    \begin{minipage}{\textwidth}
    \centering
    \begin{forest}
      /tikz/every node/.append style={font=\footnotesize},
      for tree={l sep=2em, s sep=2.5em, align=center},
      [\textit{Wort}, intrme
        [{Hat\\Numerus?}, decide
          [\textit{flektierbar}, intrme, yes
            [{Ist finit\\flektierbar?}, decide
              [\textbf{Verb}, finall, yes]
              [\textit{Nomen}, intrme, no
                [{Hat festes\\Genus?}, decide
                  [\textbf{Substantiv}, finall, yes]
                  [{\textit{anderes}\\\textit{Nomen}}, intrme, no
                    [{Hat Stärke-\\flexion?}, decide
                      [\textbf{Adjektiv}, finall, yes]
                      [{\textit{Artikel\slash}\\\textit{Pronomen}}, intrme, no]
                    ]
                  ]
                ]
              ]
            ]
          ]
          [\textit{nicht flektierbar}, intrme, no
          [{Hat Valenz-\slash\\Kasusrektion?}, decide
              [\textbf{Präposition}, finall, yes]
              [\textit{andere}, intrme, no
                [{Leitet Neben-\\Sätze ein?}, decide
                  [\textbf{Komplementierer}, finall, yes]
                  [{\textit{Partikel\slash}\\\textit{Adverb}}, intrme, no
                    [{Kann das Vor-\\feld besetzen?}, decide
                      [{\textit{Adverb\slash}\\\textit{Adkopula}}, intrme, yes
                        [{Wird typisch mit\\Kopula verwendet?}, decide
                          [\textbf{Adkopula}, finall, yes]
                          [\textbf{Adverb}, finall, no]
                        ]
                      ]
                      [\textit{Partikel}, intrme, no
                        [{Kann Sätze\\ersetzen?}, decide
                          [\textbf{Satzäquivalent}, finall, yes]
                          [\textit{andere}, intrme, no
                            [{Kann Konsti-\\tuenten verbinden?}, decide
                              [\textbf{Konjunktion}, finall, yes]
                              [\textit{Rest}, intrme, no]
                            ]
                          ]
                        ]
                      ]
                    ]
                  ]
                ]
              ]
            ]
          ]
        ]
      ]
    \end{forest}
    \end{minipage}
    }
  \end{center}
\end{frame}

\begin{frame}
  {Wie viele Wortklassen gibt es?}
  \pause
  \begin{itemize}[<+->]
    \item Alle Wörter sind \alert{Wörter}.
    \item Also gibt es \rot{eine Wortklasse}.
      \Zeile
    \item Jedes Wort hat \alert{individuelle Eigenschaften}.
    \item Also gibt es \rot{so viele Wortklassen wie Wörter}.
      \Zeile
    \item Wozu brauchen wie überhaupt Wortklassen?\\
      Wortklassen\dots
      \begin{itemize}[<+->]
        \item \dots sind \alert{das Rüstzeug für Morphologie und Syntax}.
        \item \dots erlauben die Formulierung von \rot{Generalisierungen}.
        \item \dots sind so fein unterteilt, wie es unsere Beschreibung erfordert.
        \item \dots sind \rot{nicht universell}!
        \item \dots sind \alert{Artefakte unserer Theorie bzw.\ Grammatik}.
      \end{itemize}
  \end{itemize}
\end{frame}

\section{Vorschau}

\begin{frame}
  {Morphologie}
  \pause
  \textit{"`Das ist wegen der Spannendheit."'}\\
  \pause
  \textit{"`Die Vase ist vollansichtlich reliefiert."}\\
  \Zeile
  \pause
  \begin{itemize}[<+->]
    \item (Wort-)Formen, ihre Bestandteile und ihre Funktionen
    \item Umlaut und Ablaut und ihre Funktionen
    \item Unterschied von Flexion und Wortbildung
      \Zeile
    \item Funktion nominaler Flexionskategorien
    \item \rot{Wichtig!} \alert{Inklusive: Was ist Kasus?}
    \item Funktionen verbaler Flexionskategorien
    \item \rot{Wichtig!} \alert{Inklusive: Was ist Tempus?}
  \end{itemize}
  \pause
  \Zeile
  \centering
  Bitte lesen: \alert{Kapitel 7 (195--220), 9.1 (248--257), 10.1 (287--299)}\\
  Bitte denken Sie daran: Sie haben dafür \alert{3 Wochen} Zeit!
\end{frame}

\end{document}
