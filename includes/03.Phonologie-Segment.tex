\documentclass[handout,aspectratio=1610]{beamer}
%\documentclass[aspectratio=1610]{beamer}

%\usepackage[T1]{fontenc}
\usepackage[ngerman]{babel}
\usepackage{color}
\usepackage{colortbl}
\usepackage{textcomp}
\usepackage{multirow}
\usepackage{nicefrac}
\usepackage{multicol}
\usepackage{gb4e-}
\usepackage{verbatim}
\usepackage{cancel}
\usepackage{graphicx}
\usepackage{hyperref}
\usepackage{verbatim}
\usepackage{boxedminipage}
\usepackage{rotating}
\usepackage{booktabs}
\usepackage{bbding}
\usepackage{soul}

\usepackage{tikz}
\usetikzlibrary{positioning,arrows,cd}
\tikzset{>=latex}

\usepackage{forest}

\usepackage{FiraSans}

\usepackage[maxbibnames=99,
  maxcitenames=2,
  uniquelist=false,
  backend=biber,
  doi=false,
  url=false,
  isbn=false,
  bibstyle=biblatex-sp-unified,
  citestyle=sp-authoryear-comp]{biblatex}
\addbibresource{rs.bib}

\forestset{
  Ephr/.style={draw, ellipse, thick, inner sep=2pt},
  Eobl/.style={draw, rounded corners, inner sep=5pt},
  Eopt/.style={draw, rounded corners, densely dashed, inner sep=5pt},
  Erec/.style={draw, rounded corners, double, inner sep=5pt},
  Eoptrec/.style={draw, rounded corners, densely dashed, double, inner sep=5pt},
  Ehd/.style={rounded corners, fill=gray, inner sep=5pt,
    delay={content=\whyte{##1}}
  },
  Emult/.style={for children={no edge}, for tree={l sep=0pt}},
  phrasenschema/.style={for tree={l sep=2em, s sep=2em}},
  decide/.style={draw, chamfered rectangle, inner sep=2pt},
  finall/.style={rounded corners, fill=gray, text=white},
  intrme/.style={draw, rounded corners},
  yes/.style={edge label={node[near end, above, sloped, font=\scriptsize]{Ja}}},
  no/.style={edge label={node[near end, above, sloped, font=\scriptsize]{Nein}}},
  sake/.style={tier=preterminal},
  ake/.style={
    tier=preterminal
    },
}


\tikzset{
    invisible/.style={opacity=0,text opacity=0},
    visible on/.style={alt=#1{}{invisible}},
    alt/.code args={<#1>#2#3}{%
      \alt<#1>{\pgfkeysalso{#2}}{\pgfkeysalso{#3}} % \pgfkeysalso doesn't change the path
    },
}
\forestset{
  visible on/.style={
    for tree={
      /tikz/visible on={#1},
      edge+={/tikz/visible on={#1}}}}}

\definecolor{lg}{rgb}{.8,.8,.8}
\newcommand{\Dim}{\cellcolor{lg}}


\newcommand{\Sub}[1]{\ensuremath{_{\text{#1}}}}
\newcommand{\Up}[1]{\ensuremath{^{\text{#1}}}}

\newcommand{\Ck}{\CheckmarkBold}
\newcommand{\Fl}{\XSolidBrush}

\usetheme[hideothersubsections]{Goettingen}

%\newcommand{\rot}[1]{{\color[rgb]{0.4,0.2,0}#1}}
%\newcommand{\blau}[1]{{\color[rgb]{0,0,.9} #1}}
\newcommand{\gruen}[1]{{\color[rgb]{0,0.4,0}#1}}
\newcommand{\grau}[1]{{\color[rgb]{0.5,0.5,0.5}#1}}

\newcommand{\rot}[1]{{\color[rgb]{0.6,0.2,0.0}#1}}
\newcommand{\blau}[1]{{\color[rgb]{0.0,0.0,0.9}#1}}
\newcommand{\orongsch}[1]{{\color[RGB]{255,165,0}#1}}

\renewcommand<>{\rot}[1]{%
  \alt#2{\beameroriginal{\rot}{#1}}{#1}%
}
\renewcommand<>{\blau}[1]{%
  \alt#2{\beameroriginal{\blau}{#1}}{#1}%
}
\renewcommand<>{\orongsch}[1]{%
  \alt#2{\beameroriginal{\orongsch}{#1}}{#1}%
}

\definecolor{trueblue}{rgb}{0,0.0,0.7}
\setbeamercolor{alerted text}{fg=trueblue}

\newcommand{\xxx}{\hspaceThis{[}}
\newcommand{\zB}{z.\,B.\ }
\newcommand{\down}[1]{\ensuremath{\mathrm{#1}}}

\newcommand{\Zeile}{\vspace{\baselineskip}}
\newcommand{\Halbzeile}{\vspace{0.5\baselineskip}}
\newcommand{\Viertelzeile}{\vspace{0.25\baselineskip}}

\resetcounteronoverlays{exx}


\title{Einführung in die Sprachwissenschaft\\3.~Phonologie: Segmente und Verteilungen}
\author{Roland Schäfer}
\institute{Deutsche und niederländische Philologie\\Freie Universität Berlin}
\date{Wintersemester 2018/2019\\23.~Oktober 2018}


\begin{document}

\frame{\titlepage}

\section{Rückblick}

\begin{frame}
  {Erinnerung an letzte Woche: Phonetik}
  \pause
  \begin{itemize}[<+->]
    \item Artilulationsorgane
    \item Konsonanten
      \begin{itemize}
        \item Stimmton
        \item Art: Plosiv, Frikativ, Affrikate, Nasal, Approximant
      \end{itemize}
    \item Vokale:
      \begin{itemize}
        \item vorne -- hinten
        \item hoch -- tief
        \item gerundet -- ungerundet
        \item lang -- kurz
        \item Diphthonge
      \end{itemize}
    \item Sonoranten und Obstruenten
    \item r-Laute und sekundäre Diphthonge
  \end{itemize}
\end{frame}


\section{Phonologie}

\begin{frame}
  {Übersicht}
  \pause
  \begin{itemize}[<+->]
    \item \alert{Segmente} als Einheiten der Phonetik\slash Phonologie
    \item nicht alle Segmente überall: \alert{Verteilungen}
    \item Endrand-Desonorisierung, r-Vokalisierung, \textit{ich}\slash\textit{ach}-Laute usw.\\
      und \alert{Ableitung} phonetischer Formen aus lexikalischen Formen
    \item längbare, betonbare und unbetonbare Vokale
      \Zeile
    \item empfohlene Literatur: \citet{Eisenberg2013a} (Grundriss: Wort)
  \end{itemize}
\end{frame}

\begin{frame}
  {Was hat Phonologie mit Bildungs- und Normsprache zu tun?}
  \pause
  \begin{itemize}[<+->]
    \item mit Bildungssprache nicht viel
    \item mit Normsprache sehr viel
      \begin{itemize}[<+->]
        \item viele dialektale und soziolektale Einflüsse phonologisch\\
          statt phonetisch
        \item graphematisches System angelehnt an phonologischem
        \item Worttrennung
      \end{itemize}
  \end{itemize}
\end{frame}

\begin{frame}
  {Segmente}
  \pause
  \begin{itemize}[<+->]
    \item Transkriptionen: \textit{Tier} [ti͡ɐ], \textit{Tür} [ty͡ɐ], \textit{rotem} [ʁoːtəm],\\
      \textit{Lob} [loːp], \textit{Bades} [baːdəs], \textit{Pfanne} [p͡fanə], \textit{Osten} [ʔɔstən]
      \vspace{\baselineskip}
    \item Warum gibt es die Basiszeichen im IPA, die es gibt? (a, ə, ɪ, ʔ, p, ʁ usw.)
      \begin{itemize}
        \item \alert{artikulatorische Untrennbarkeit}
        \item \alert{nicht-autonomes "`Verhalten"'}
      \end{itemize}
      \vspace{\baselineskip}
    \item Sind p͡f und a͡ɔ usw.\ ein oder zwei Segmente? 
      \begin{itemize}
        \item artikulatorisch trennbar
        \item autonomes Verhalten?
        \item eigentlich eine phonologische Frage → Verteilungen
      \end{itemize}
  \end{itemize}
\end{frame}

\begin{frame}
  {Verteilungen: Beispiele}
  \pause
  \begin{exe}
    \ex
      \begin{xlist}
        \ex Tod [toːt], Kot [koːt]
        \pause
        \ex Schott [ʃɔt], Schock [ʃɔk]
      \end{xlist}
        \pause
    \ex Hang [haŋ], *[ŋah]
        \pause
    \ex
      \begin{xlist}
        \ex Sog [zoːk], besingen [bəzɪŋən], *[soːk]
        \pause
        \ex fließ [fliːs], Boss [bɔs], *[fliːz]
        \pause
        \ex heißer [ha͡ɛsɐ], heiser [ha͡ɛzɐ], Base [baːzə], Basse [basə], *[bazə]
      \end{xlist}
  \end{exe}
\end{frame}


\begin{frame}
  {Verteilung: Definition}
  \pause
  \Large
  Die \alert{Verteilung} eines Segments ist die Menge\\
  der Umgebungen, in denen es vorkommt.\\[\baselineskip]
  \pause
  Zwei phonetisch unterschiedliche Segmente bzw.\\
  Merkmale stehen in einem \alert{phonologischen\\
  Kontrast}, wenn sie eine teilweise oder vollständig übereinstimmende Verteilung haben und dadurch einen lexikalischen bzw.\ grammatischen Unterschied markieren können.
\end{frame}

\begin{frame}
  {Neutralisierung: Beispiele}
  \pause
  \begin{exe}
    \ex
    \begin{xlist}
      \ex{Weg [veːk], Weges [veːgəs]}
      \pause
      \ex{Bock [bɔk], Bockes [bɔkəs]}
      \pause
    \end{xlist}
    \ex
    \begin{xlist}
      \ex{Bad [baːt], Bades [baːdəs]}
      \pause
      \ex{Blatt [blat], Blattes [blatəs]}
      \pause
    \end{xlist}
    \ex
    \begin{xlist}
      \ex{Lob [loːp], Lobes [loːbəs]}
      \pause
      \ex{Depp [dɛp], Deppen [dɛpən]}
    \end{xlist}
    \ex
    \begin{xlist}
      \ex aktiv [ʔaktiːf], aktive [ʔaktiːvə]
      \ex tief [tiːf], tiefe [tiːfə]
    \end{xlist}
    \ex
    \begin{xlist}
      \ex fies [fiːs], fiese [fiːzə]
      \ex Bus [bʊs], Busse [bʊsə]
    \end{xlist}
  \end{exe}
\end{frame}

\begin{frame}
  {Neutralisierung: Definition}
  \pause
  \Large
  Eine \alert{Neutralisierung} ist die Aufhebung\\
  eines phonologischen Kontrasts\\
  in einer bestimmten Position.
\end{frame}

\begin{frame}
  {Das Lexikon (Kapitel 2)}
  \pause
  \Large
  Das \alert{Lexikon} ist die Menge aller Wörter\\
  einer Sprache, definiert durch die vollständige\\
  Angabe ihrer Merkmale und deren Werte.\\[\baselineskip]
  \pause
  Hier ist das relevante Merkmal die \alert{Kette\\
  von Segmenten}, die ein Wort phonetisch bzw.\\
  phonologisch definiert.
\end{frame}

\begin{frame}
  {Muss man ʔ lexikalisch lernen?}
  \pause
  \begin{itemize}
    \item{[ʔan], [dan], [kan], [ʁan], [van], [man], [ban]}
    \item{[ʔoːnə], [boːnə], [loːnə], [t͡soːnə], [foːnə], [moːnə], [zoːnə]}
    \item{[ʔe͡ɐt], [ve͡ɐt], [le͡ɐt], [ke͡ɐt], [te͡ɐt], [ge͡ɐt], [he͡ɐt]}
  \end{itemize}
  \Zeile
  \pause
  \begin{itemize}[<+->]
    \item{\alert{[ʔ] steht immer dann, wenn sonst kein anderer Konsonant steht.}}
    \item{[ʔ] ist artikulatorisch und perzeptorisch wenig salient.}
    \item also: nicht lexikalisch, \alert{automatisch einsetzbar}
  \end{itemize}
\end{frame}

\begin{frame}
  {Nochmal Endrand-Desonorisierung}
  \pause
  \begin{exe}
    \ex
    \begin{xlist}
      \ex{Weg [veːk], Weges [veːgəs]}
      \ex{Bock [bɔk], Bockes [bɔkəs]}
    \end{xlist}
    \ex
    \begin{xlist}
      \ex{Bad [baːt], Bades [baːdəs]}
      \ex{Blatt [blat], Blattes [blatəs]}
    \end{xlist}
    \ex
    \begin{xlist}
      \ex{Lob [loːp], Lobes [loːbəs]}
      \ex{Depp [dɛp], Deppen [dɛpən]}
    \end{xlist}
    \ex
    \begin{xlist}
      \ex aktiv [ʔaktiːf], aktive [ʔaktiːvə]
      \ex tief [tiːf], tiefe [tiːfə]
    \end{xlist}
    \ex
    \begin{xlist}
      \ex fies [fiːs], fiese [fiːzə]
      \ex Bus [bʊs], Busse [bʊsə]
    \end{xlist}
  \end{exe}
  \pause
  \Zeile
  \begin{itemize}
    \item \alert{Aus welcher Form kann man die andere jeweils "`herleiten"'?}
  \end{itemize}
\end{frame}


\begin{frame}
  {Zugrundeliegende Form und Strukturbedingung}
  \pause
  \Large
  Die \alert{zugrundeliegende Form} (eines Wortes) ist\\
  genau die Folge von Segmenten, die im Lexikon\\
  gespeichert wird, und auf die alle zugehörigen\\
  phonetischen Formen zurückgeführt werden können.\\[0.5\baselineskip]
  \pause
  Die Formen werden ggf. an die phonologischen\\
  \alert{Strukturbedingungen} (die Regularitäten\\
  der phonologischen Grammatik) angepasst.
\end{frame}

\begin{frame}
  {Architektur der Grammatik und externer Systeme}
  \pause
  \centering
  \resizebox{0.9\textwidth}{!}{
    \begin{tabular}{ccc}
      \toprule
      \multicolumn{2}{c}{\textbf{Grammatik}} & \textbf{Externe Systeme} \\
      \midrule
      \textbf{Lexikon} & \textbf{Phonologie} & \textbf{Phonetik} \\
      \midrule
      /~/& $\Rightarrow$ & [~]\\
      zugrundeliegende Form & Anpassung an Strukturbedingungen & phonetische Realisierung \\
      \bottomrule
    \end{tabular}
  }
\end{frame}

\begin{frame}
  {Also für ʔ und Endrand-Desonorisierung}
  \pause
  \begin{itemize}[<+->]
    \item ʔ
      \begin{itemize}[<+->]
        \item /an/ $\Rightarrow$ [ʔan] 
        \item /oːnə/ $\Rightarrow$ [?oːnə]
        \item /e͡ɐt/ $\Rightarrow$ [ʔe͡ɐt]
      \end{itemize}
      \Zeile
    \item Endrand-Desonorisierung
      \begin{itemize}[<+->]
        \item /veːg/ $\Rightarrow$ [veːk], /bɔk/ $\Rightarrow$ [bɔk]
        \item /baːd/ $\Rightarrow$ [baːt], /blat/ $\Rightarrow$ [blat]
        \item /loːb/ $\Rightarrow$ [loːp], /dɛp/ $\Rightarrow$ [dɛp]
        \item /aktiːv/ $\Rightarrow$ [ʔaktiːf], /tiːf/ $\Rightarrow$ [tiːf]
        \item /fiːz/ $\Rightarrow$ [fiːs], /bʊs/ $\Rightarrow$ [bʊs]
      \end{itemize}
  \end{itemize}
\end{frame}


\begin{frame}
  {Merkmale, phonetisch motiviert (Kapitel 4)}
  \pause
  \resizebox{0.9\textwidth}{!}{
  \begin{minipage}{\textwidth}
  \begin{exe}
    \ex{\label{ex:phonetischemerkmale009} \textsc{Art}: \textit{plosiv}, \textit{frikativ}, \textit{affrikate}, \textit{nasal}, \textit{approximant}, \textit{vokal}}

    \vspace{0.5\baselineskip}
    \pause

    \ex{\label{ex:phonetischemerkmale010} \textbf{Für Konsonanten:}}\\
      \textsc{Obstruent}: $+$, $-$
    \pause

    \vspace{0.5\baselineskip} 

    \ex \textbf{Für Vokale:}
    \pause
      \begin{xlist}
        \ex \textsc{Höhe}: \textit{hoch}, \textit{halbhoch}, \textit{mittel}, \textit{halbtief}, \textit{tief}
        \pause
        \ex \textsc{Lage}: \textit{vorn}, \textit{halbvorn}, \textit{zentral}, \textit{halbhinten}, \textit{hinten}
        \pause
        \ex \textsc{Rund}: $+$, $-$
        \pause
        \ex \textsc{Lang}: $+$, $-$
      \end{xlist}

    \pause
    \vspace{0.5\baselineskip}

    \ex \textbf{Für Konsonanten:}\\
        \textsc{Ort}: \textit{laryngal}, \textit{uvular}, \textit{velar}, \textit{palatal}, \textit{palatoalveolar}, \textit{alveolar}
    \pause
    \ex \textbf{Für Obstruenten:}\\
      \textsc{Stimme}: $+$, $-$
  \end{exe}
  \end{minipage}
  }
\end{frame}


\begin{frame}
  {Endrand-Desonorisierung als Strukturbedingung}
  \pause
  \Large
  Alle Segmente mit [\textsc{Obstruent}:~$+$]\\
  sind am Silbenende [\textsc{Stimme}:~$-$].
\end{frame}


\begin{frame}
  {Verteilung von [ç] und [χ]}
  \pause
  \begin{exe}
    \ex
    \begin{xlist}
      \ex krieche, schlich, Bücher, Küche, Recht, Köche
      \pause
      \ex Tuch, Geruch, hoch, Koch, Schmach, Bach
    \end{xlist}
  \end{exe}
  \pause
  \Zeile
  \Large
  [ç] kann nicht nach Vokalen stehen, die nicht\\
  {[\textsc{Lage}: \textit{vorne}]} sind. Zugrundeliegendes /ç/\\
  wird daher nach zentralen und hinteren Vokalen\\
  weiter hinten artikuliert, nämlich als [χ].
\end{frame}

\begin{frame}
  {r-Vokalisierung}
  \pause
  \begin{exe}
    \ex
    \begin{xlist}
      \ex \textit{kleiner} [kla͡ɛ.nɐ], \textit{kleinere} [kla͡ɛ.nə.ʁə]
      \pause
      \ex \textit{Bär} [bɛ͡ɐ], \textit{Bären} [bɛː.ʁən]
      \pause
      \ex \textit{knarr} [kna͡ə], \textit{knarre} [kna.ʁə]
    \end{xlist}
  \end{exe}
  \pause
  \Zeile
  \Large
  Zugrundeliegendes /ʁ/ kann nicht am Silbenende\\
  stehen. Es wird in dieser Position als\\
  Schwa-Segment im sekundären Diphthong\\
  realisiert. Nach gespanntem Vokal folgt [ɐ],\\
  nach ungespanntem folgt [ə]. Schwa und /ʁ/\\
  werden zusammen durch [ɐ] substituiert.\\[0.5\baselineskip]
  \pause
  \alert{Gespannt?}
\end{frame}


\begin{frame}[fragile]
  {Erinnerung an die Vokale des Deutschen}
  \begin{center}
  \resizebox{0.6\textwidth}{!}{
  \begin{tikzpicture}[scale=2.5,baseline=default]
    \large
    \tikzset{
      vowel/.style={fill=white, anchor=mid, text depth=0ex, text height=1ex},
      dot/.style={circle,fill=black,minimum size=0.4ex,inner sep=0pt,outer sep=-1pt},
    }

    \coordinate (hf) at (0,2); % high front
    \coordinate (hb) at (2,2); % high back
    \coordinate (lf) at (1,0); % low front
    \coordinate (lb) at (2,0); % low back
    \def\V(#1,#2){barycentric cs:hf={(3-#1)*(2-#2)},hb={(3-#1)*#2},lf={#1*(2-#2)},lb={#1*#2}}

    % Chart key (vorne -- hinten).
    \draw [{Latex[round]}-] (\V (-.25,0))   -- (\V (-.25,.5)) node [above left] {\footnotesize vorne};
    \draw [-{Latex[round]}] (\V (-.25,1.5)) -- (\V (-.25,2))  node [above left] {\footnotesize hinten};
    \path (\V (-.25,1)) node[above] {\footnotesize zentral};

    % Chart key (hoch--tief).
    \draw [{Latex[round]}-] (\V (0,-.25)) -- +(270:.5cm)  node [above right,rotate=90] (vokaltrapez1) {\footnotesize hoch};
    \draw [{Latex[round]}-] (\V (3,-2.5)) -- +(270:-.5cm) node [above left,rotate=90] (vokaltrapez2) {\footnotesize tief};
    \path (\V (1.5,-1)) node[above,rotate=90] {\footnotesize mittel};

    % Grid. 
    \draw [gray, thick] (\V(0,0)) -- (\V(0,2));
    \draw [gray, thick] (\V(1,0)) -- (\V(1,2));
    \draw [gray, thick] (\V(2,0)) -- (\V(2,2));
    \draw [gray, thick] (\V(3,0)) -- (\V(3,2));
    \draw [gray, thick] (\V(0,0)) -- (\V(3,0));
    \draw [gray, thick] (\V(0,1)) -- (\V(3,1));
    \draw [gray, thick] (\V(0,2)) -- (\V(3,2));

    % Unrounded-rounded pairs.
    \path (\V(0,0))     node[vowel, left]     {i} node[vowel, right] (y) {y} node[dot] {};
    \path (\V(0.5,0.5)) node[vowel, left]     {ɪ} node[vowel, right] (Y) {ʏ} node[dot] {};
    \path (\V(1,0))     node[vowel, left]     {e} node[vowel, right] (e) {ø} node[dot] {};
    \path (\V(2,0))     node[vowel, left] (E) {ɛ} node[vowel, right] (ee) {œ} node[dot] {};

    % Unpaired symbols.
    \path (\V(1.5,1))    node [vowel] (schwa)  {ə};
    \path (\V(2.5,1))    node [vowel] (schwaa) {ɐ};
    \path (\V(3,1))      node [vowel] (a)      {a};
    \path (\V (2,2))     node [vowel] (oo)     {ɔ};
    \path (\V (1,2))     node [vowel] (o)      {o};
    \path (\V (0,2))     node [vowel] (u)      {u};
    \path (\V (0.5,1.5)) node [vowel] (uu)     {ʊ};

  \end{tikzpicture}
  }
  \end{center}
\end{frame}


\begin{frame}
  {Länge und Betonung und Vokalqualität im Systemkern}
  \pause
  \centering
  \begin{tabular}{cllp{0.25cm}cll}
    \toprule
    \textbf{gespannt} & \textbf{Beispiel} & \textbf{IPA} & & \textbf{ungespannt} & \textbf{Beispiel} & \textbf{IPA} \\
    \midrule
    i  & \textit{bieten} & biːtən && ɪ & \textit{bitten}  & bɪtən   \\
    y  & \textit{fühlt}  & fyːlt  && ʏ & \textit{füllt}   & fʏlt    \\
    u  & \textit{Mus}    & muːs   && ʊ & \textit{muss}    & mʊs     \\
    e  & \textit{Kehle}  & keːlə  && ɛ & \textit{Kelle}   & kɛlə    \\
    ɛ  & \textit{stähle} & ʃtɛːlə && ɛ & \textit{Ställe}  & ʃtɛlə   \\
    ø  & \textit{Höhle}  & høːlə  && œ & \textit{Hölle}   & hœlə \\
    o  & \textit{Ofen}   & ʔoːfən && ɔ & \textit{offen}   & ʔɔfən   \\
    a  & \textit{Wahn}   & vaːn   && a & \textit{wann}    & van     \\
    \bottomrule
  \end{tabular}\\
  \pause
  \Zeile
  \begin{itemize}[<+->]
    \item Laut\alert{e}, b\alert{e}schreib\alert{e}n, \dots
    \item L\alert{i}thografie, H\alert{y}draulik, B\alert{u}tan, Ph\alert{e}nol, \alert{Ö}nologie, Mes\alert{o}zoon, \dots
  \end{itemize}
\end{frame}

\begin{frame}
  {Gespanntheit im Kernwortschatz}
  \pause
  \Large
  \rot{Im Kernwortschatz sind gespannte Vokale immer\\
  betont und lang.} Zu jedem gespannten Vokal gibt es\\
  einen entsprechenden ungespannten Vokal.\\
  Der ungespannte ist betont oder unbetont,\\
  aber immer kurz.\\
  \Zeile
  Im kernwortschatz muss die Länge also eigentlich nicht markiert werden, sondern folgt aus\\
  Betonung und Gespanntheit.
\end{frame}

\begin{frame}[fragile]
  {Gespanntheit}
  \pause
  \begin{center}
    \resizebox{0.6\textwidth}{!}{
      \begin{tikzpicture}[scale=3.5,baseline=default]
        \large
        \tikzset{
        vowel/.style={fill=white, anchor=mid, text depth=0ex, text height=1ex},
        vowelgespannt/.style={circle,fill=gray!30, anchor=mid, text depth=0ex, text height=1ex,minimum size=4ex},
        dot/.style={circle,fill=black,minimum size=0.4ex,inner sep=0pt,outer sep=-1pt},
        }

        \coordinate (hf) at (0,2); % high front
        \coordinate (hb) at (2,2); % high back
        \coordinate (lf) at (1,0); % low front
        \coordinate (lb) at (2,0); % low back
        \def\V(#1,#2){barycentric cs:hf={(3-#1)*(2-#2)},hb={(3-#1)*#2},lf={#1*(2-#2)},lb={#1*#2}}

        % Chart key (vorne -- hinten).
        \draw [{Latex[round]}-] (\V (-.25,0)) -- (\V (-.25,.5))  node [above left] {\footnotesize vorne};
        \draw [-{Latex[round]}] (\V (-.25,1.5)) -- (\V (-.25,2)) node [above left] {\footnotesize hinten};
        \path (\V (-.25,1)) node[above] {\footnotesize zentral};

        % Chart key (hoch--tief).
        \draw [{Latex[round]}-] (\V (0,-.25)) -- +(270:.5cm)  node [above right,rotate=90] (vokaltrapez1) {\footnotesize hoch};
        \draw [{Latex[round]}-] (\V (3,-2.5)) -- +(270:-.5cm) node [above left,rotate=90] (vokaltrapez2) {\footnotesize tief};
        \path (\V (1.5,-1)) node[above,rotate=90] {\footnotesize mittel};

        % Grid.
        \draw [gray,thick] (\V(0,0)) -- (\V(0,2));
        \draw [gray,thick] (\V(3,0)) -- (\V(3,2));
        \draw [gray,thick] (\V(0,0)) -- (\V(3,0));
        \draw [gray,thick] (\V(0,2)) -- (\V(3,2));

        \path (\V(0,0))      node[vowelgespannt] (i)   {i};
        \path (\V(0.25,0))   node[vowelgespannt] (y)   {y};
        \path (\V(0.4,0.5))  node[vowel]         (ii)  {ɪ};
        \path (\V(0.65,0.5)) node[vowel]         (yy)  {ʏ};
        \path (\V(1,0))      node[vowelgespannt] (e)   {e};
        \path (\V(1.25,0))   node[vowelgespannt] (oe)  {ø};
        \path (\V(2,0))      node[vowelgespannt] (ee)  {ɛ};
        \path (\V(1.4,0.7))  node[vowel]         (eee) {ɛ̆};
        \path (\V(1.65,0.7)) node[vowel]         (oee) {œ};
        \path (\V(3,1))      node[vowelgespannt] (a)   {a};
        \path (\V(2.5,1))    node[vowel]         (aa)  {ă};
        \path (\V (1,2))     node[vowelgespannt] (o)   {o};
        \path (\V (1.5,1.4)) node[vowel]         (oo)  {ɔ};
        \path (\V (0,2))     node[vowelgespannt] (u)   {u};
        \path (\V (0.5,1.5)) node[vowel]         (uu)  {ʊ};

        \draw (i)  -- (ii);
        \draw (y)  -- (yy);
        \draw (e)  -- (eee);
        \draw (oe) -- (oee);
        \draw (ee) -- (eee);
        \draw (a)  -- (aa);
        \draw (o)  -- (oo);
        \draw (u)  -- (uu);
      \end{tikzpicture}
    }
  \end{center}
\end{frame}


\begin{frame}
  {Und Schwa?}
  \pause
  Warum kommt Schwa (also [ə] und [ɐ]) im System der gespannten\\
  und ungespannten Vokale nicht vor?\\
  \pause
  \Zeile
  \Zeile
  \centering
  \Large
  \alert{Schwa ist nicht betonbar!}
\end{frame}

\begin{frame}
  {Merkmale, phonetisch motiviert, phonologisch reduziert (Kern)}
  \pause
  \resizebox{0.9\textwidth}{!}{
  \begin{minipage}{\textwidth}
  \begin{exe}
    \ex\grau{\textsc{Art}: \textit{plosiv}, \textit{frikativ}, \textit{affrikate}, \textit{nasal}, \textit{approximant}, \textit{vokal}}

    \vspace{0.5\baselineskip}
    \ex\grau{\label{ex:phonetischemerkmale010} \textbf{Für Konsonanten:}\\
      \textsc{Obstruent}: $+$, $-$}
    \vspace{0.5\baselineskip} 

    \ex \textbf{Für Vokale:}
      \begin{xlist}
        \ex \textsc{Höhe}: \textit{hoch}, \textit{halbhoch}, \textit{mittel}, \textit{halbtief}, \textit{tief}
        \ex \textsc{Lage}: \textit{vorn}, \rot{\st{\textit{halbvorn}}}, \textit{zentral}, \rot{\st{\textit{halbhinten}}}, \textit{hinten}
        \ex \textsc{Rund}: $+$, $-$
        \ex \rot{\st{\textsc{Lang}: $+$, $-$}}
        \ex \alert{\textsc{Gespannt}: $+$, $-$}

      \end{xlist}

    \vspace{0.5\baselineskip}

    \ex\grau{\textbf{Für Konsonanten:}\\
    \textsc{Ort}: \textit{laryngal}, \textit{uvular}, \textit{velar}, \textit{palatal}, \textit{palatoalveolar}, \textit{alveolar}}
    \ex\grau{\textbf{Für Obstruenten:}\\
    \textsc{Stimme}: $+$, $-$}
  \end{exe}
  \end{minipage}
  }
  
\end{frame}


\begin{frame}
  {Und der erweiterte Wortschatz?}
  \resizebox{0.9\textwidth}{!}{
  \begin{minipage}{\textwidth}
  \begin{exe}
    \ex\label{ex:gespanntheitbetonungundlaenge021}
    \begin{xlist}
      \ex{\label{ex:gespanntheitbetonungundlaenge022} \textit{Idee} [ʔideː]\\
        \textit{Initiative} [ʔinit͡sʝatiːvə]\\
        \textit{inspirieren} [ʔɪnspiʁiːʁən] }
      \ex{\label{ex:gespanntheitbetonungundlaenge023} \textit{Methyl} [metyːl]\\
        \textit{Québec} [kebɛk]\\
        \textit{integriert} [ʔɪntegʁi͡ɐt]\\
        \textit{debattieren} [debatiːʁən] }
      \ex{\label{ex:gespanntheitbetonungundlaenge024} \textit{Utopie} [ʔutopiː]\\
        \textit{Uran} [ʔuʁaːn] }
      \ex{\label{ex:gespanntheitbetonungundlaenge025} \textit{Motiv} [motiːf]\\
        \textit{politisch} [poliːtɪʃ]\\
        \textit{Phonologie} [fonologiː] }
      \ex{\label{ex:gespanntheitbetonungundlaenge026} \textit{Ökonomie} [ʔøkonomiː]\\
        \textit{manövrieren} [manøvʁiːʁən] }
      \ex{\label{ex:gespanntheitbetonungundlaenge027} \textit{Büro} [byʁoː]\\
      \textit{Cuvée} [kyveː] }
    \end{xlist}
  \end{exe}
  \end{minipage}
  }
\end{frame}

\begin{frame}
  {Gespanntheit im erweiterten Wortschatz}
  \pause
  \Large
  \rot{Im erweiterten Wortschatz sind gespannte Vokale\\
  lang, wenn sie betont sind, und kurz, wenn sie \\
unbetont sind.} Auch im erweiterten Wortschatz\\
  gibt es keine ungespannten langen Vokale.\\
  \Zeile
  \normalsize
  \begin{exe}
    \ex\label{ex:gespanntheitbetonungundlaenge028} \begin{xlist}
      \ex /veg/ $\Rightarrow$ [veːk]
      \ex /hølə/ $\Rightarrow$ [høːlə]
      \ex /ofən/ $\Rightarrow$ [ʔoːfən]
    \end{xlist}
  \end{exe}
\end{frame}


\section{Ausblick auf die Graphematik}

\begin{frame}
  {Segmente und Buchstaben}
  \pause
  \centering
  \resizebox{0.45\textwidth}{!}{
    \begin{tabular}{lll}
      \toprule
      \textbf{Segment} & \textbf{Buchstabe(n)} & \textbf{Beispielwörter} \\
      \midrule
     p & p & \textit{Plan} \\
     b & b & \textit{Baum}, \textit{Trab} \\
     p͡f & pf & \textit{Pfad} \\
     f & f & \textit{Fahrt} \\
     v & w & \textit{Wand} \\
     m & m & \textit{Mus} \\
     t & t & \textit{Tau} \\
     d & d & \textit{Dach}, \textit{Bild}\\
     t͡s & z & \textit{Zeit} \\
     s & s & \textit{Los} \\
     z & s & \textit{Sau} \\
     ʃ & sch & \textit{Schiff} \\
     n & n & \textit{Not}, \textit{Klang} \\
     l & l & \textit{Lob} \\
     ç & ch & \textit{Blech}, \textit{Wacht} \\
     ʝ & j & \textit{Jahr} \\
     k & k & \textit{Kiel} \\
     g & g & \textit{Gans}, \textit{Weg}, \textit{König} \\
     ʁ & r & \textit{Ritt}, \textit{Tür} \\
     h & h & \textit{Herz} \\
      \bottomrule
    \end{tabular}
  }
\end{frame}

\begin{frame}
  {Invarianz der Konsonantenschreibungen}
  \centering
  \resizebox{0.9\textwidth}{!}{
      \begin{tabular}{lp{0.15cm}lp{0.15cm}llp{0.15cm}llp{0.15cm}l}
      \toprule
      \textbf{zugr.} && \textbf{Buch-} && \multicolumn{2}{l}{\textbf{phonetische}}    && \multicolumn{2}{l}{\textbf{phonologische}} && \textbf{phonetische} \\
      \textbf{Segm.} && \textbf{stabe(n)} && \multicolumn{2}{l}{\textbf{Realisierungen}} && \multicolumn{2}{l}{\textbf{Schreibungen}}  && \textbf{Schreibung} \\
      \midrule
      b && b && ba͡ɔm & loːp && \textit{Baum} & \textit{Lob} && *\textit{Lop} \\
      d && d && daχ & ʁɪnt && \textit{Dach} & \textit{Rind} && *\textit{Rint} \\
      n && n && naχt & klaŋ && \textit{Nacht} & \textit{Klang} && *\textit{Klaŋ} \\
      ç && ch && lɪçt & vaχt && \textit{Licht} & \textit{Wacht} && *\textit{Waχt} \\
      g && g && gans & køːnɪç && \textit{Gans} & \textit{König} && *\textit{Könich} \\
      ʁ && r && ʁuːm & to͡ɐ && \textit{Ruhm} & \textit{Tor} && *\textit{Toe} \\
      \bottomrule
    \end{tabular}
  }
\end{frame}


\begin{frame}
  {Vokalschreibungen}
  \centering
  \resizebox{0.9\textwidth}{!}{
    \begin{tabular}{lp{0.5cm}llp{0.25cm}ll}
      \toprule
      \multirow{2}{*}{\textbf{Buchstabe}} && \multicolumn{2}{l}{\textbf{Segment}} && \multicolumn{2}{l}{\textbf{Segment}} \\
       && \textbf{gespannt} & \textbf{Beispiel} && \textbf{ungespannt} & \textbf{Beispiel} \\
      \midrule
      i  && i  & \textit{Igel} && ɪ & \textit{Licht} \\
      ü  && y  & \textit{Rübe} && ʏ & \textit{Rücken} \\
      u  && u  & \textit{Mut} && ʊ & \textit{Butter} \\
      e  && e  & \textit{Mehl} && ɛ̆ & \textit{Bett} \\
      ö  && ø & \textit{Höhle} && œ & \textit{Löffel} \\
      o  && o  & \textit{Ofen} && ɔ & \textit{Motte} \\
      ä  && ɛ  & \textit{Gräte} && ɛ̆ & \textit{Säcke} \\
      a  && a  & \textit{Wal} && ă & \textit{Wall} \\
      \bottomrule
    \end{tabular}
  }
\end{frame}


\section{Vorschau}

\begin{frame}
  {Nächste Woche: Vom Segment zur Silbe}
  \pause
  \begin{itemize}[<+->]
    \item Bildung von Silben als Anpassung an Strukturbedingungen
    \item Silben als rhythmische Einheiten in der phonologischen Kombinatorik
    \item das eng eingegrenzte Strukturschema der (deutschen) Silbe: (C)CV(C)(C)
    \item Silben als Schließen--Öffnen--Schließen des Vokaltrakts
    \item Sonoritätskontur als Zeichen davon
    \item Segmente, die nicht zur Silbe gehören (\textit{\rot{S}paß}, \textit{Herb\rot{sts}})
    \item Liquide: /l/ und /ʁ/
    \item begrenzte Optionen für die \alert{Länge} bzw.\ das \alert{Gewicht} von Silben
    \item Silbifizierung: Grundlage der Wortrennung\\
      (\textit{But- ter} als optimales Trennmuster)
  \end{itemize}
  \pause
  \Zeile
  \centering
  \Large
  \alert{Bitte lesen: Kapitel 5, Abschnitt 5.2, Seiten 123--152}
  \pause
  \pause
  \pause
  \pause
  \pause
\end{frame}


\begin{frame}
  {Literatur}
  \renewcommand*{\bibfont}{\footnotesize}
  \setbeamertemplate{bibliography item}{}
  \printbibliography
\end{frame}

\end{document}
