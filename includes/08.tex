\documentclass[handout,aspectratio=1610]{beamer}
%\documentclass[aspectratio=1610]{beamer}

%\usepackage[T1]{fontenc}
\usepackage[ngerman]{babel}
\usepackage{color}
\usepackage{colortbl}
\usepackage{textcomp}
\usepackage{multirow}
\usepackage{nicefrac}
\usepackage{multicol}
\usepackage{gb4e-}
\usepackage{verbatim}
\usepackage{cancel}
\usepackage{graphicx}
\usepackage{hyperref}
\usepackage{verbatim}
\usepackage{boxedminipage}
\usepackage{rotating}
\usepackage{booktabs}
\usepackage{bbding}

\usepackage{tikz}
\usetikzlibrary{positioning,arrows,cd}
\tikzset{>=latex}

\usepackage[linguistics]{forest}

\usepackage{FiraSans}

\usepackage[maxbibnames=99,
  maxcitenames=2,
  uniquelist=false,
  backend=biber,
  doi=false,
  url=false,
  isbn=false,
  bibstyle=biblatex-sp-unified,
  citestyle=sp-authoryear-comp]{biblatex}
\addbibresource{rs.bib}

\forestset{
  Ephr/.style={draw, ellipse, thick, inner sep=2pt},
  Eobl/.style={draw, rounded corners, inner sep=5pt},
  Eopt/.style={draw, rounded corners, densely dashed, inner sep=5pt},
  Erec/.style={draw, rounded corners, double, inner sep=5pt},
  Eoptrec/.style={draw, rounded corners, densely dashed, double, inner sep=5pt},
  Ehd/.style={rounded corners, fill=gray, inner sep=5pt,
    delay={content=\whyte{##1}}
  },
  Emult/.style={for children={no edge}, for tree={l sep=0pt}},
  phrasenschema/.style={for tree={l sep=2em, s sep=2em}},
  decide/.style={draw, chamfered rectangle, inner sep=2pt},
  finall/.style={rounded corners, fill=gray, text=white},
  intrme/.style={draw, rounded corners},
  yes/.style={edge label={node[near end, above, sloped, font=\scriptsize]{Ja}}},
  no/.style={edge label={node[near end, above, sloped, font=\scriptsize]{Nein}}},
  sake/.style={tier=preterminal},
  ake/.style={
    tier=preterminal
    },
}


\tikzset{
    invisible/.style={opacity=0,text opacity=0},
    visible on/.style={alt=#1{}{invisible}},
    alt/.code args={<#1>#2#3}{%
      \alt<#1>{\pgfkeysalso{#2}}{\pgfkeysalso{#3}} % \pgfkeysalso doesn't change the path
    },
}
\forestset{
  visible on/.style={
    for tree={
      /tikz/visible on={#1},
      edge+={/tikz/visible on={#1}}}}}

\definecolor{lg}{rgb}{.8,.8,.8}
\newcommand{\Dim}{\cellcolor{lg}}


\newcommand{\Sub}[1]{\ensuremath{_{\text{#1}}}}
\newcommand{\Up}[1]{\ensuremath{^{\text{#1}}}}

\newcommand{\Ck}{\CheckmarkBold}
\newcommand{\Fl}{\XSolidBrush}

\usetheme[hideothersubsections]{Goettingen}

%\newcommand{\rot}[1]{{\color[rgb]{0.4,0.2,0}#1}}
%\newcommand{\blau}[1]{{\color[rgb]{0,0,.9} #1}}
\newcommand{\gruen}[1]{{\color[rgb]{0,0.4,0}#1}}


\newcommand{\blaw}[1]{{\color[rgb]{0,0,.9}#1}}

\newcommand{\rot}[1]{{\color[rgb]{0.6,0.2,0.0}#1}}
\newcommand{\blau}[1]{{\color[rgb]{0.0,0.0,0.9}#1}}
\newcommand{\orongsch}[1]{{\color[RGB]{255,165,0}#1}}
\newcommand{\grau}[1]{{\color[rgb]{0.5,0.5,0.5}#1}}

\renewcommand<>{\rot}[1]{%
  \alt#2{\beameroriginal{\rot}{#1}}{#1}%
}
\renewcommand<>{\blau}[1]{%
  \alt#2{\beameroriginal{\blau}{#1}}{#1}%
}
\renewcommand<>{\orongsch}[1]{%
  \alt#2{\beameroriginal{\orongsch}{#1}}{#1}%
}

\definecolor{trueblue}{rgb}{0,0.0,0.7}
\setbeamercolor{alerted text}{fg=trueblue}

\newcommand{\xxx}{\hspaceThis{[}}
\newcommand{\zB}{z.\,B.\ }
\newcommand{\down}[1]{\ensuremath{\mathrm{#1}}}

\newcommand{\Zeile}{\vspace{\baselineskip}}
\newcommand{\Halbzeile}{\vspace{0.5\baselineskip}}
\newcommand{\Viertelzeile}{\vspace{0.25\baselineskip}}

\newcommand{\whyte}[1]{\textcolor{white}{#1}}

\resetcounteronoverlays{exx}

\title{Einführung in die Sprachwissenschaft\\
8.~Flexion}
\author{Roland Schäfer}
\institute{Deutsche und niederländische Philologie\\Freie Universität Berlin}
\date{Wintersemester 2018/2019\\11.~Dezember 2018}

\begin{document}

\frame{\titlepage}

% \section{Rückblick}
% 
% \begin{frame}
%   {Rückblick: Wortbildung}
%   \pause
%   \begin{itemize}[<+->]
%     \item x
%   \end{itemize}
% \end{frame}

\section{Überblick}

% \begin{frame}
%   {Überblick: Flexion}
%   \pause
%   \begin{itemize}[<+->]
%     \item v
%   \end{itemize}
% \end{frame}

\begin{frame}
  {Warum über Flexion sprechen?}
  \pause
  \begin{itemize}[<+->]
    \item \alert{Wir beherrschen doch alle Formen!}
      \Halbzeile
    \item Funktion der Flexionskategorien
      \begin{itemize}
        \item semantisch\slash pragmatisch
        \item \alert{systemintern} als Hilfe zu \alert{Rekonstruktion der Satzstruktur}
      \end{itemize}
      \Halbzeile
    \item Flexion im Deutschen ein ideales und gut durchschaubares Beispiel\\
      für die klassische \alert{reduktionistische} Methode der Linguistik\\
      (= Analyse der Sprache als \alert{System})
      \Halbzeile
    \item Heute keine Beispiele? Doch, aber es sind ganze Paradigmen!
      \Halbzeile
    \item \alert{Können} vs.\ \rot{Erklären}
    \item Reaktion auf Erwerbsschwierigkeiten
    \item Reaktion auf nicht-deutsche Erstsprache
      \Halbzeile
    \item Habe ich eigtl.\ schonmal erzählt, wie ich Kasus verstanden habe?
  \end{itemize}
\end{frame}

\begin{frame}
  {Übrigens}
  \pause
  \centering
  \Large
  \alert{Lesen Sie irgendwann in Ihrem Leben Kapitel 5\\
    aus Peter Eisenbergs \textit{Grundriss}!}\\
    \citep[145--200]{Eisenberg2013a}
\end{frame}

\section{Nominalflexion}

\subsection{Substantive}

\begin{frame}
  {Substantive: Kasus und Numerus}
  Das traditionelle Chaos der Flexionstypen mit Kasus-Numerus-Formen\ldots\\
  \Zeile
  \pause
  \Zeile
  \resizebox{\textwidth}{!}{
    \begin{tabular}{llp{0mm}lp{2mm}llp{1mm}lp{2mm}llp{2mm}l}
      \toprule
      \multicolumn{2}{c}{} && \multicolumn{1}{l}{\textbf{Maskulinum}} && \multicolumn{4}{l}{\textbf{Maskulinum und Neutrum}} && \multicolumn{2}{l}{\textbf{Femininum}} && \multicolumn{1}{l}{\textbf{s-Flexion}} \\
      \multicolumn{2}{c}{} && \multicolumn{1}{l}{\textbf{schwach (S1)}} && \multicolumn{2}{l}{\textbf{stark (S2)}} && \multicolumn{1}{l}{\textbf{gemischt (S3)}} && \multicolumn{2}{l}{\textbf{(S4)}} && \multicolumn{1}{l}{\textbf{(S5)}} \\
      \midrule
      \multirow{4}{*}{\textbf{Sg}} & \textbf{Nom} && Mensch && Stuhl & Haus && Staat && Frau & \multicolumn{1}{l}{Sau} && Auto \\
      & \textbf{Akk} && Mensch-en && Stuhl & Haus && Staat && Frau & \multicolumn{1}{l}{Sau} && Auto \\
      & \textbf{Dat} && Mensch-en && Stuhl & Haus && Staat && Frau & \multicolumn{1}{l}{Sau} && Auto \\
      & \textbf{Gen} && Mensch-en && Stuhl-es & Haus-es && Staat-(e)s && Frau & \multicolumn{1}{l}{Sau} && Auto-s \\
      \midrule
      \multirow{4}{*}{\textbf{Pl}} & \textbf{Nom} && Mensch-en && Stühl-e & Häus-er && Staat-en && Frau-en & \multicolumn{1}{l}{Säu-e} && Auto-s \\
      & \textbf{Akk} && Mensch-en && Stühl-e & Häus-er && Staat-en && Frau-en & \multicolumn{1}{l}{Säu-e} && Auto-s \\
      & \textbf{Dat} && Mensch-en && Stühl-en & Häus-ern && Staat-en && Frau-en & \multicolumn{1}{l}{Säu-en} && Auto-s \\
      & \textbf{Gen} && Mensch-en && Stühl-e & Häus-er && Staat-en && Frau-en & \multicolumn{1}{l}{Säu-e} && Auto-s \\
      \bottomrule
    \end{tabular}
  }
\end{frame}

\begin{frame}
  {Das traditionelle Chaos als "`System"'}
  \pause
  Das geht irgendwie nach Genus und Pluralbildung, aber nicht nur\ldots\\
  \pause
  \Zeile
  \begin{center}
    \resizebox{0.8\textwidth}{!}{
    \begin{tikzpicture}[every text node part/.style={align=center}]
      \node (MaskN)    at (2,4)   {\textbf{Maskulin}};
      \node (NeutN)    at (4,4)   {\textbf{Neutral}};
      \node (FemN)     at (8.5,4) {\textbf{Feminin}};

      \node (schwachN) at (0,2)   {\textbf{schwach}\\\textit{Mensch-en}};
      \node (starkN)   at (2,2)   {\textbf{stark}};
      \node (gemistN)  at (4,2)   {\textbf{gemischt}\\\textit{Staat-en}};
      \node (sFlexN)   at (6.5,2) {\textbf{s-Flexion}\\\textit{Auto-s}\\\textit{Papaya-s}};
      \node (s4N)      at (8.5,2) {\textbf{"`S4"'}};

      \node (EPlu)     at (0.5,0) {Plural\\\textit{\char`~e}\\\textit{Stühl-e}};
      \node (ePlu)     at (2,0)   {Plural\\\textit{-e}\\\textit{Gurt-e}};
      \node (erPlu)    at (3.5,0) {Plural\\\textit{\char`~er}\\Lämm-er};

      \node (enPlu)    at (7.5,0) {Plural\\\textit{-en}\\\textit{Tasch-en}};
      \node (EnPlu)    at (9.5,0) {Plural\\\textit{\char`~en}\\\textit{Säu-e}};

      \draw (MaskN.south)  -- (schwachN.north);
      \draw (MaskN.south)  -- (starkN.north);
      \draw (MaskN.south)  -- (gemistN.north);
      \draw (MaskN.south)  -- (sFlexN.north);

      \draw (NeutN.south)  -- (starkN.north);
      \draw (NeutN.south)  -- (gemistN.north);
      \draw (NeutN.south)  -- (sFlexN.north);

      \draw (FemN.south)   -- (s4N.north);
      \draw (FemN.south)   -- (sFlexN.north);

      \draw [dashed] (starkN.south) -- (EPlu.north);
      \draw [dashed] (starkN.south) -- (ePlu.north);
      \draw [dashed] (starkN.south) -- (erPlu.north);

      \draw [dashed] (s4N.south)    -- (enPlu.north);
      \draw [dashed] (s4N.south)    -- (EnPlu.north);
    \end{tikzpicture}
  }      
  \end{center}
\end{frame}

\begin{frame}
  {Aber das war noch nicht alles: mit und ohne Schwa}
  \pause
  Es gibt Varianten der Affixe \alert{ohne} Schwa:\\
  \Zeile
  \pause
  \begin{center}
    \resizebox{\textwidth}{!}{
      \begin{tabular}{llp{1mm}llp{1mm}llp{1mm}ll}
        \toprule
        \multicolumn{2}{l}{\textbf{schwach}} && \multicolumn{2}{l}{\textbf{gemischt}} && \multicolumn{2}{l}{\textbf{Fem S4a}} && \multicolumn{2}{l}{\textbf{Fem S4b}}\\
        \textbf{voll} & \textbf{reduziert} && \textbf{voll} & \textbf{reduziert} && \textbf{voll} & \textbf{reduziert} && \textbf{voll} & \textbf{reduziert} \\
        \midrule
        Mensch-\rot{e}n & Löwe-n && Staat-\rot{e}n & Ende-n && Frau-\rot{e}n & Nudel-n && Säu-\rot{e} & Mütter-$\emptyset$ \\
        \bottomrule
      \end{tabular}
    }
  \end{center}
\end{frame}

\begin{frame}
  {Zusammenfassung (außer Substantive mit s-Plural)}
  \pause
  Die traditionelle \alert{Klassenzugehörigkeiten}, nicht aber die vollen\\
  \alert{Paradigmen}, lassen sich als \alert{Entscheidungsbaum} zusammenfassen:\\
  \pause
  \begin{center}
    \resizebox{0.7\textwidth}{!}{
      \begin{forest}
        for tree={l sep=2em},
        decide/.style={draw, chamfered rectangle, inner sep=2pt},
        finall/.style={rounded corners, fill=gray}
          [Genus?, s sep=10em, decide
            [Genitiv\\Singular?, s sep+=2em, decide,
              edge label={node[pos=0.6, above, sloped, font=\scriptsize]{maskulin\slash neutrum}}
              [\whyte{Schwache}\\\whyte{Maskulina (S1)}, finall,
                edge label={node[pos=0.7, above, sloped, font=\scriptsize]{-(e)n}}
              ]
              [Nominativ\\Plural?, s sep+=0.5em, decide,
                edge label={node[pos=0.7, above, sloped, font=\scriptsize]{-(e)s}}
                [\whyte{Gemischte Maskulina}\\\whyte{und Neutra (S3)}, finall,
                  edge label={node[pos=0.7, above, sloped, font=\scriptsize]{-(e)n}}]
                [\whyte{Starke Maskulina}\\\whyte{und Neutra (S2)}, finall,
                  edge label={node[pos=0.7, above, sloped, font=\scriptsize]{\char`~er\slash \char`~e\slash-e}}]
              ]
            ]
            [\whyte{Feminina (S4)}, finall,
              edge label={node[midway, above, sloped, font=\scriptsize]{feminin}}]
          ]
      \end{forest}
    }
  \end{center}
\end{frame}


\begin{frame}
  {Der Ansatz in EGBD}
  \large \alert{Sauber trennen zwischen Numerus- und Kasusmarkierung!}\\
  \pause
  \Halbzeile
  \normalsize
  Erstens: \alert{Der Plural ist immer stärker markiert als\\
  oder mindestens gleich stark markiert wie der Singular.}\\
  → \rot{Pluralbildung ist die dominante Flexionseigenschaft}.
  \pause
  \Zeile
  \begin{center}
    \begin{tabular}{llll}
      \toprule
      \textbf{Klasse} & \textbf{Kasus} & \textbf{Sg} & \textbf{Pl} \\
      \midrule
      S1 & Nom & (der) Mensch & (die) Mensch-en \\
      S2a & Gen & (des) Stuhl-es & (der) Stühl-e \\
      S2b & Akk & (den) Gurt & (die) Gurt-e \\
      S2c & Dat & (dem) Haus & (den) Häus-ern \\
      S3 & Akk & (den) Staat & (die) Staat-en \\
      S4a & Nom & (die) Frau & (die) Frau-en \\
      S4b & Nom & (die) Sau & (die) Säu-e \\
      \midrule
      S1 & Akk & (den) Mensch-en & (die) Mensch-en \\
      S5 & Gen & (des) Auto-s & (der) Auto-s \\
      \bottomrule
    \end{tabular}  
  \end{center}
\end{frame}


\begin{frame}
  {Pluralbildungen}
  \pause
  Zweitens: Isolierung der Plural-Affixe.\\
  \Zeile
  \pause
  \begin{center}
    \resizebox{\textwidth}{!}{
      \begin{tabular}{llp{0mm}lp{2mm}llp{1mm}lp{2mm}llp{2mm}l}
        \toprule
        \multicolumn{2}{c}{} && \multicolumn{1}{l}{\textbf{Maskulinum}} && \multicolumn{4}{l}{\textbf{Maskulinum und Neutrum}} && \multicolumn{2}{l}{\textbf{Femininum}} && \multicolumn{1}{l}{\textbf{s-Flexion}} \\
        \multicolumn{2}{c}{} && \multicolumn{1}{l}{\textbf{schwach (S1)}} && \multicolumn{2}{l}{\textbf{stark (S2)}} && \multicolumn{1}{l}{\textbf{gemischt (S3)}} && \multicolumn{2}{l}{\textbf{(S4)}} && \multicolumn{1}{l}{\textbf{(S5)}} \\
        \midrule
        \multirow{4}{*}{\textbf{Sg}} & \textbf{Nom} && Mensch && Stuhl & Haus && Staat && Frau & \multicolumn{1}{l}{Sau} && Auto \\
        & \textbf{Akk} && Mensch\rot<5->{-en} && Stuhl & Haus && Staat && Frau & \multicolumn{1}{l}{Sau} && Auto \\
        & \textbf{Dat} && Mensch\rot<5->{-en} && Stuhl(-e) & Haus(-e) && Staat(-e) && Frau & \multicolumn{1}{l}{Sau} && Auto \\
        & \textbf{Gen} && Mensch\rot<5->{-en} && Stuhl-(e)s & Haus-(e)s && Staat-(e)s && Frau & \multicolumn{1}{l}{Sau} && Auto\orongsch<6->{-s} \\
        \midrule
        \multirow{4}{*}{\textbf{Pl}} & \textbf{Nom} && Mensch\rot<5->{\alert<4>{-en}} && Stühl\alert<4->{-e} & Häus\alert<4->{-er}   && Staat\alert<4->{-en} && Frau\alert<4->{-en} & \multicolumn{1}{l}{Säu\alert<4->{-e}}   && Auto\alert<4->{-s} \\
        & \textbf{Akk} && Mensch\rot<5->{\alert<4>{-en}} && Stühl\alert<4->{-e}                             & Häus\alert<4->{-er}   && Staat\alert<4->{-en} && Frau\alert<4->{-en} & \multicolumn{1}{l}{Säu\alert<4->{-e}}   && Auto\alert<4->{-s} \\
        & \textbf{Dat} && Mensch\rot<5->{\alert<4>{-en}} && Stühl\alert<4->{-e}-n                           & Häus\alert<4->{-er}-n && Staat\alert<4->{-en} && Frau\alert<4->{-en} & \multicolumn{1}{l}{Säu\alert<4->{-e}-n} && Auto\alert<4->{-s} \\
        & \textbf{Gen} && Mensch\rot<5->{\alert<4>{-en}} && Stühl\alert<4->{-e}                             & Häus\alert<4->{-er}   && Staat\alert<4->{-en} && Frau\alert<4->{-en} & \multicolumn{1}{l}{Säu\alert<4->{-e}}   && Auto\alert<4->{-s} \\
        \bottomrule
      \end{tabular}
    }
  \end{center}
  \pause
  \pause
  \pause
  \pause
  \begin{itemize}[<+->]
    \item schwache Maskulina raus! → \alert{Sonderklasse mit niedriger Typfrequenz}
    \item Genitiv Singular bei s-Flexion: \rot{nicht} rausnehmen (s.~unten)
    \item was an Affixen übrig bleibt: \alert{Kasus}
  \end{itemize}
\end{frame}


\begin{frame}
  {Kasusmarkierungen}
  \pause
  Was bleibt denn übrig für Kasus?
  \Zeile
  \pause
  \begin{center}
    \resizebox{\textwidth}{!}{
      \begin{tabular}{llp{0mm}llp{1mm}lp{2mm}llp{2mm}l}
        \toprule
        \multicolumn{2}{c}{} && \multicolumn{4}{l}{\textbf{Maskulinum und Neutrum}} && \multicolumn{2}{l}{\textbf{Femininum}} && \multicolumn{1}{l}{\textbf{s-Flexion}} \\
        \multicolumn{2}{c}{} && \multicolumn{2}{l}{\textbf{stark (S2)}} && \multicolumn{1}{l}{\textbf{gemischt (S3)}} && \multicolumn{2}{l}{\textbf{(S4)}} && \multicolumn{1}{l}{\textbf{(S5)}} \\
        \midrule
        \multirow{4}{*}{\textbf{Sg}} & \textbf{Nom} && Stuhl & Haus && Staat && Frau & \multicolumn{1}{l}{Sau} && Auto \\
        & \textbf{Akk} && Stuhl & Haus && Staat && Frau & \multicolumn{1}{l}{Sau} && Auto \\
        & \textbf{Dat} && Stuhl & Haus && Staat && Frau & \multicolumn{1}{l}{Sau} && Auto \\
        & \textbf{Gen} && Stuhl\rot{-es} & Haus\rot{-(e)s} && Staat\rot{-(e)s} && Frau\onslide<5->{\rot{*-s}} & \multicolumn{1}{l}{Sau\onslide<5->{\rot{*-s}}} && Auto\rot{-s} \\
        \midrule
        \multirow{4}{*}{\textbf{Pl}} & \textbf{Nom} && Stühl-e & Häus-er && Staat-en && Frau-en & \multicolumn{1}{l}{Säu-e} && Auto-s \\
        & \textbf{Akk} && Stühl-e & Häus-er && Staat-en && Frau-en & \multicolumn{1}{l}{Säu-e} && Auto-s \\
        & \textbf{Dat} && Stühl-e\alert{-n} & Häus-er\alert{-n} && Staat-en\onslide<4->{\alert{*-n}} && Frau-en\onslide<4->{\alert{*-n}} & \multicolumn{1}{l}{Säu-e\alert{-n}} && Auto-s\onslide<4->{\alert{*-n}} \\
        & \textbf{Gen} && Stühl-e & Häus-er && Staat-en && Frau-en & \multicolumn{1}{l}{Säu-e} && Auto-s \\
        \bottomrule
      \end{tabular}
    }
  \end{center}
\end{frame}

\begin{frame}
  {Regularitäten der Substantivflexion}
  \pause
  \begin{itemize}[<+->]
    \item \rot{schwache Maskulina sind die einzige "`Sonderklasse"'}
    \item \alert{Pluralklasse determniniert Flexionsverhalten}
    \item Genus determiniert teilweise Pluralklasse
      \begin{itemize}[<+->]
        \item \alert{Mask prototypisch \textit{\char`~e} oder \textit{-e}}
        \item \alert{Fem prototypisch \textit{-en}}
        \item Kleinstklasse: Mask und Neut \textit{-er}
        \item Subst endet mit Vollkvokal (\textit{Kanu-s}) oder Kurzwort (\textit{LKWs}): s-Plural
      \end{itemize}
    \Halbzeile
  \item \alert{Maskulin Genitiv Singular: \textit{-(e)s}} \rot{außer phonotaktisch unmöglich}
    \item \alert{alle Genera Dativ Plural: \textit{-(e)n}}
    \item keine Sequenzen von Schwa-Silben: \textit{die Tüte-n} statt \textit{*Tüte-en}
    \item keine Doppelkonsonanten: \textit{die Bolzen} statt \textit{*Bolzen-en} oder \textit{Bolzen-n}
      \Halbzeile
    \item Genitiv-Regularität auch bei s-Substantiven
      \begin{itemize}[<+->]
        \item \textit{des Kanu-s}
        \item \rot{\textit{*der Papaya-s}} (Sg)
      \end{itemize}
  \end{itemize}
\end{frame}

\begin{frame}
  {Grafische Darstellung des Klassensystems}
  \pause
  \begin{center}
    \resizebox{0.7\textwidth}{!}{
      \begin{forest}
        [Substantive, calign=last, l sep+=2em
          [\textit{en}-Maskulina]
          [normale Flexion{,}\\differenziert\\nur nach\\Pluralbildung, l sep+=2em
            [\textit{\char`~er}\\nur Maskulina\\und Neutra\\(Kleinstklasse)]
            [\textit{\char`~e}\slash\textit{-e}\\Protoyp\\der \textbf{Maskulina}\\und \textbf{Neutra}]
            [\textit{-en}\\Prototyp\\der \textbf{Feminina}]
            [\textit{-s}\\lexikalisch oder\\phonotaktisch\\motiviert]
          ]
        ]
      \end{forest}
    }
  \end{center}
\end{frame}

\subsection{Pronomina und Artikel}

\begin{frame}
  {Pronomina in Pronominalfunktion}
  \pause
  \begin{exe}
    \ex
    \begin{xlist}
      \ex{\alert{[Der Autor dieses Textes]} schreibt \alert{[Sätze, die noch niemand vorher geschrieben hat]}.}
      \ex{\alert{[Dieser]} schreibt \alert{[etwas]}.}
    \end{xlist}
    \ex
    \begin{xlist}
      \ex{Block: Was ist mit den Texten?\\
        Henry: Martin schreibt gerade \alert{[einen]}.}
    \end{xlist}
  \end{exe}
  \pause
  \Zeile
  \alert{In dieser Funktion stehen Pronomina\\
  anstelle einer vollen Nominalphrase.}
\end{frame}


\begin{frame}
  {Pronomina in Artikelfunktion}
  \pause
  \begin{exe}
    \ex \label{ex:gemeinsamkeitenundunterschiede074}
    \begin{xlist}
      \ex{[\alert{Dieser} frische Marmorkuchen] schmeckt lecker.}
      \ex{[\alert{Jeder} leckere Marmorkuchen] ist mir recht.}
    \end{xlist}
  \end{exe}
  \pause
  \Zeile
  \alert{In dieser Funktion stehen Pronomina\\
  vor einem Substantiv, mit dem sie kongruieren.}\\
  \Zeile
  \pause
  Wörter in dieser Position allgemein: \alert{Artikelwörter} (auch Determinative)\\
  \Zeile
  \pause
  Im weiteren: nur regelmäßig flektierende ("`normale"') Pronomina\\
  (nicht \textit{ich}, \textit{du}, \textit{man}, \textit{etwas} usw.)
\end{frame}


\begin{frame}
  {Warum ist das so schwer? I}
  \pause
  \rot{Wenn die Formen in Artikelfunktion und Pronominalfunktion\\
    nicht durchgehend gleich sind, nehmen wir \textbf{zwei verschiedene\\
    lexikalische Wörter mit gleichlautendem Stamm} an: Artikel und Pronomen.}\\
    \Zeile
    \pause
    \begin{center}
      \begin{tabular}[h]{lp{1em}llp{2em}l}
        \toprule
        \textbf{Kasus (Singular)} &&  \textbf{Artikel} &       && \textbf{Pronomen} \\
        \midrule
        \textbf{Nominativ}        &&  \rot{ein}   & Mantel  && \rot{einer} \\
        \textbf{Akkusativ}        &&  einen & Mantel  && einen \\
        \textbf{Dativ}            &&  einem & Mantel  && einem \\
        \textbf{Genitiv}          &&  eines & Mantels && eines \\
        \bottomrule
      \end{tabular}
    \end{center}
    \pause
    \Zeile
    \alert{Also gibt es einen Artilel \textit{ein} und ein Pronomen \textit{ein}.}
\end{frame}


\begin{frame}
  {Warum ist das so schwer? II}
  \pause
  \rot{Wenn die Formen in Artikelfunktion und Pronominalfunktion\\
    nicht durchgehend gleich sind, nehmen wir \textbf{zwei verschiedene\\
    lexikalische Wörter mit gleichlautendem Stamm} an: Artikel und Pronomen.}\\
    \Zeile
    \pause
    \begin{center}
      \begin{tabular}[h]{lp{1em}llp{2em}l}
        \toprule
        \textbf{Kasus (Plural)}   &&  \textbf{Artikel} &        && \textbf{Pronomen} \\
        \midrule
        \textbf{Nominativ}        && die               & Rottweiler  && die \\
        \textbf{Akkusativ}        && die               & Rottweiler  && die \\
        \textbf{Dativ}            && \rot{den}         & Rottweilern && \rot{denen} \\
        \textbf{Genitiv}          && \rot{der}         & Rottweiler  && \rot{derer} \\
        \bottomrule
      \end{tabular}
    \end{center}
    \pause
    \Zeile
    \alert{Also gibt es einen Artikel \textit{d-} und ein Pronomen \textit{d-}.}
\end{frame}


\begin{frame}
  {Warum ist das so schwer? III}
  \pause
  \rot{Wenn die Formen in Artikelfunktion und Pronominalfunktion\\
    nicht durchgehend gleich sind, nehmen wir \textbf{zwei verschiedene\\
    lexikalische Wörter mit gleichlautendem Stamm} an: Artikel und Pronomen.}\\
    \Zeile
    \pause
    \begin{center}
      \resizebox{0.8\textwidth}{!}{
        \begin{tabular}[h]{llp{1em}llp{2em}l}
          \toprule
          &\textbf{Kasus}       &&  \textbf{Pronomen} &         && \textbf{Pronomen} \\
          &&& \multicolumn{2}{l}{\textbf{in Artikelfunktion}} && \textbf{in Pronominalfunktion} \\
          \midrule 
          \textbf{Sg} & \textbf{Nominativ}   && dieser  & Rottweiler         && dieser \\
          &\textbf{Akkusativ}   && diesen  & Rottweiler         && diesen \\
          &\textbf{Dativ}       && diesem  & Rottweiler         && diesem \\
          &\textbf{Genitiv}     && dieses  & Rottweilers        && dieses \\
          \midrule 
          \textbf{Pl}&\textbf{Nominativ}   && diese   & Rottweiler         && diese \\
          &\textbf{Akkusativ}   && diese   & Rottweiler         && diese \\
          &\textbf{Dativ}       && diesen  & Rottweilern        && diesen \\
          &\textbf{Genitiv}     && dieser  & Rottweiler         && dieser \\
          \bottomrule
        \end{tabular}
      }
    \end{center}
    \pause
    \Zeile
    \alert{Also gibt es nur ein Pronomen \textit{dies}, das in beiden Funktionen auftritt.}
\end{frame}


\begin{frame}
  {Warum ist das so schwer? IV}
  \pause
  \Large Zum Mitschreiben:\\
  \pause
  \Zeile
  \rot{Treten die Stämme \textit{ein}, \textit{kein}, \textit{mein}, \textit{dein}, \textit{sein}, \textit{ihr}, \textit{euer}, \textit{unser} oder \textit{d-} in Artikelfunktion auf, \textbf{sind sie Artikel}.}\\
  \pause
  \Zeile
  \rot{Treten sie hingegen in Pronominalfunktion auf,\\
    \textbf{sind sie Pronomina}.}\\
  \pause
  \Zeile
  \alert{Alle anderen pronominalen\slash artikelartigen Stämme\\
    gehören immer nur zu einem Pronomen und treten\\
    \textbf{als Pronomen} in Artikel- oder Pronominalfunktion auf.}
\end{frame}


\begin{frame}
  {Das (ganz) normale Pronomen}
  \pause
  \begin{center}
    \begin{tabular}{lllll}
      \toprule
      \multicolumn{1}{c}{} & \textbf{Mask} & \textbf{Neut} & \textbf{Fem} & \textbf{Pl} \\
      \midrule
      \textbf{Nom} & dies-er & dies-es & dies-e & dies-e \\
      \textbf{Akk} & dies-en & dies-es & dies-e & dies-e \\
      \textbf{Dat} & dies-em & dies-em & dies-er & dies-en \\
      \textbf{Gen} & dies-es & dies-es & dies-er & dies-er \\
      \bottomrule
    \end{tabular}
  \end{center}
\end{frame}


\begin{frame}
  {Synkretismen}
  \pause
  Wo ist das Vier-Kasus-System?
  \pause
  \Zeile
  \begin{center}
    \begin{tabular}{|l|c|c|c|c|}
      \cline{2-5}
      \multicolumn{1}{c|}{} & \alert<4->{\textbf{Mask}} & \textbf{Neut} & \textbf{Fem} & \textbf{Pl} \\
      \hline
      \textbf{Nom} & \alert<4->{-er} & \multirow{2}{*}{-es} & \multicolumn{2}{c|}{\multirow{2}{*}{-e}} \\ \cline{1-2}
      \textbf{Akk} & \alert<4->{-en} && \multicolumn{2}{c|}{} \\ \hline
      \textbf{Dat} & \multicolumn{2}{c|}{\alert<4->{-em}} && -en \\ \cline{1-3} \cline{5-5}
      \textbf{Gen} & \multicolumn{2}{c|}{\alert<4->{-es}} & \multicolumn{2}{c|}{-er} \\
      \hline
    \end{tabular}
  \end{center}
\end{frame}


\begin{frame}
  {Abweichungen bei den Definita}
  \pause
  Definitartikel\\
  \begin{center}
    \begin{tabular}{lllll}
      \toprule
      \multicolumn{1}{c}{} & \textbf{Mask} & \textbf{Neut} & \textbf{Fem} & \textbf{Pl} \\
      \midrule
      \textbf{Nom} & d-er & d-as \Dim & d-ie \Dim & d-ie \Dim \\
      \textbf{Akk} & d-en & d-as \Dim & d-ie \Dim & d-ie \Dim \\
      \textbf{Dat} & d-em & d-em & d-er & d-en \\
      \textbf{Gen} & d-es & d-es & d-er & d-er \\
      \bottomrule
    \end{tabular}
  \end{center}
  \pause

  Definitpronomen\\
  \begin{center}
    \begin{tabular}{lllll}
      \toprule
      \multicolumn{1}{c}{} & \textbf{Mask} & \textbf{Neut} & \textbf{Fem} & \textbf{Pl} \\
      \hline
      \textbf{Nom} & d-er & d-as & d-ie & d-ie \\
      \textbf{Akk} & d-en & d-as & d-ie & d-ie \\
      \textbf{Dat} & d-em & d-em & d-er & d-en-en \Dim \\
      \textbf{Gen} & d-ess-en \Dim & d-ess-en \Dim & d-er-er \Dim & d-er-er \Dim\\
      \bottomrule
    \end{tabular}
  \end{center}
\end{frame}


\begin{frame}
  {Abweichung des Indefinitartikels}
  Das Indefinitpronomen flektiert als normales Pronomen. Aber:\\
  \pause
  \Zeile
  \begin{center}
    \begin{tabular}{lllll}
      \toprule
      \multicolumn{1}{c}{} & \textbf{Mask} & \textbf{Neut} & \textbf{Fem} & \textbf{Pl} \\
      \hline
      \textbf{Nom} & d-er & d-as & d-ie & d-ie \\
      \textbf{Akk} & d-en & d-as & d-ie & d-ie \\
      \textbf{Dat} & d-em & d-em & d-er & d-en-en \Dim \\
      \textbf{Gen} & d-ess-en \Dim & d-ess-en \Dim & d-er-er \Dim & d-er-er \Dim\\
      \bottomrule
    \end{tabular}
  \end{center}
\end{frame}


\begin{frame}
  {Nochmal zurück zu Artikel vs.\ Pronomen}
  \pause
  Die auf den letzten Folien gezeigten Abweichungen von der normalen Pronominalflexion sind die systematische Aufarbeitung des eingangs gemachten Unterschieds zwischen Pronomina und Artikeln.\\
  \pause
  \Zeile
  \begin{center}
    \resizebox{0.8\textwidth}{!}{
      \begin{forest}
        [systematisch flektierende\\Pronomina und Artikel
          [Indefinit- und\\Possessivartikel\\(\textit{kein}{,} \textit{mein} usw.)]
          [normale Pronomina\\und Definita
            [normale Pronomina\\(\textit{jener}{,} \textit{meiner} usw.)]
            [Definita
              [Definitartikel\\(\textit{der}{,} \textit{des} usw.)]
              [Definitpronomina\\(\textit{der}{,} \textit{dessen} usw.)]
            ]
          ]
        ]
      \end{forest}
    }
  \end{center}
  \pause
  \Halbzeile
  Übrigens: Wir definieren hier gerade weitere Wortklassen.
\end{frame}


\subsection{Adjektive}

\begin{frame}
  {Das traditionelle Chaos}
  \pause
  \begin{center}
    \resizebox{0.6\textwidth}{!}{
    \begin{tabular}{lllllll}
      \toprule
      \multicolumn{3}{l}{} & \textbf{Mask} & \textbf{Neut} & \textbf{Fem} & \textbf{Pl} \\
      \midrule
      \multirow{4}{*}{\textbf{stark}} & \textbf{Nom} & \multirow{4}{*}{heiß-} & er & es & e & e \\
      & \textbf{Akk} && en & es & e & e \\
      & \textbf{Dat} && em & em & er & en \\
      & \textbf{Gen} && en & en & er & er \\
      \midrule
      \multirow{4}{*}{\textbf{schwach}} & \textbf{Nom} & \multirow{4}{*}{(der) heiß-} & e & e & e & en \\
      & \textbf{Akk} && en & e & e & en \\
      & \textbf{Dat} && en & en & en & en \\
      & \textbf{Gen} && en & en & en & en \\
      \midrule
      \multirow{4}{*}{\textbf{gemischt}} & \textbf{Nom} & \multirow{4}{*}{(kein) heiß-} & er & es & e & en \\
      & \textbf{Akk} && en & es & e & en \\
      & \textbf{Dat} && en & en & en & en \\
      & \textbf{Gen} && en & en & en & en \\
      \bottomrule
    \end{tabular}
  }
  \end{center}
  \pause
  \begin{itemize}[<+->]
    \item "`Merke"' (oder vielleicht auch nicht):
      \begin{itemize}[<+->]
        \item \alert{ohne} Artikel: \alert{starkes} Adjektiv
        \item mit \alert{definitem} Artikel: \alert{schwaches} Adjektiv
        \item mit \alert{indefinitem} Artikel: \alert{gemischtes} Adjektiv
      \end{itemize} 
  \end{itemize} 
\end{frame}


\begin{frame}[fragile]
  {Das System}
  \pause
  \begin{center}
    \begin{tikzpicture}[every text node part/.style={align=center}]
      \node [draw, chamfered rectangle] (FlexAVoran) at (3,6) {Geht ein Artikelwort\\mit Flexionsendung voraus?};

      \node [rounded corners, fill=gray] (SchwaFlexi) at (0,3) {\whyte{struktureller Singular: \textit{-e}}\\\whyte{Rest: \textit{-en}}};
      \node [rounded corners, fill= gray] (pronomAffi) at (6,3) {\whyte{pronominale}\\\whyte{Affixe}};

      \node [draw, rounded corners, inner sep=6pt] (AdFlexAusn) at (3,0) {\textit{-en}};

      \draw (FlexAVoran) -- node [above, sloped] {\footnotesize Ja}       (SchwaFlexi);
      \draw (FlexAVoran) -- node [above, sloped] {\footnotesize Nein}     (pronomAffi);
      \draw [dashed] (SchwaFlexi) -- node [above, sloped] {\footnotesize Ausnahme:} node [below, sloped] {\footnotesize Akkusativ Maskulinum} (AdFlexAusn);
      \draw [dashed] (pronomAffi) -- node [above, sloped] {\footnotesize Ausnahme: Genitiv} node [below, sloped] {\footnotesize Maskulinum\slash Neutrum} (AdFlexAusn);
    \end{tikzpicture}
  \end{center}
\end{frame}


\begin{frame}
  {Ohne Artikelwort: Adjektive flektieren wie Artikelwort}
  \pause
  \begin{center}
    \begin{tabular}{llp{2em}ll}
      \toprule
      dies\alert{-er} & Kaffee  && heiß\alert{-er}   & Kaffee  \\
      dies\alert{-en} & Kaffee  && heiß\alert{-en}   & Kaffee  \\
      dies\alert{-em} & Kaffee  && heiß\alert{-em}   & Kaffee  \\
      dies\rot{-es}   & Kaffees && heiß\rot{-en}     & Kaffees \\
      \midrule
      dies\alert{-es} & Dessert && heiß\alert{-es}   & Dessert \\
      dies\alert{-em} & Dessert && heiß\alert{-em}   & Dessert \\
      dies\rot{-es}   & Desserts&& heiß\rot{-en}     & Desserts \\
      \midrule
      dies\alert{-e}  & Brühe   && lecker\alert{-e}  & Brühe \\
      dies\alert{-er} & Brühe   && lecker\alert{-er} & Brühe \\
      \midrule
      dies\alert{-e}  & Kekse   && heiß\alert{-e}    & Keks \\
      dies\alert{-en} & Kekse   && heiß\alert{-en}   & Kekse \\
      dies\alert{-er} & Kekse   && heiß\alert{-er}   & Kekse \\
      \bottomrule
    \end{tabular}
  \end{center}
\end{frame}

\begin{frame}
  {Artikelwort mit normalen Affixen: "`adjektivische"' Flexion}
  \pause
  \begin{center}
    \begin{tabular}{lll}
      \toprule
      dies\alert{-er} & lecker\grau{-e}  & Kaffee   \\
      dies\alert{-en} & lecker\rot{-en} & Kaffee   \\
      dies\alert{-em} & lecker\grau{-en} & Kaffee   \\
      dies\alert{-es} & lecker\grau{-en} & Kaffees  \\
      \midrule
      dies\alert{-es} & lecker\grau{-e}  & Dessert  \\
      dies\alert{-em} & lecker\grau{-en} & Dessert  \\
      dies\alert{-es} & lecker\grau{-en} & Desserts \\
      \midrule
      dies\alert{-e}  & lecker\grau{-e}  & Brühe    \\
      dies\alert{-er} & lecker\grau{-en} & Brühe    \\
      \midrule
      dies\alert{-e}  & lecker\grau{-en} & Kekse    \\
      dies\alert{-en} & lecker\grau{-en} & Kekse    \\
      dies\alert{-er} & lecker\grau{-en} & Kekse    \\
      \bottomrule
    \end{tabular}
  \end{center}
\end{frame}

\begin{frame}
  {Die adjektivische Flexion}
  \pause
  Ein Meisterstück der systeminternen Funktionsoptimierung!\\
  \Zeile
  \pause
  \begin{center}
    \begin{tabular}{|l|llll|}
      \cline{2-5}
      \multicolumn{1}{c|}{}& \textbf{Mask} & \textbf{Neut} & \textbf{Fem} & \textbf{Pl} \\
      \hline
      \textbf{Nom} && \multirow{2}{*}{-e} & \multicolumn{1}{c|}{} & \\ \cline{2-2}
      \textbf{Akk} & \multicolumn{1}{c|}{-en} && \multicolumn{1}{c|}{} & \\ \cline{2-4}
      \textbf{Dat} &&& \multirow{2}{*}{-en} & \\
      \textbf{Gen} &&&& \\
      \hline
    \end{tabular}
  \end{center}
  \pause
  "`Zielsystem"':\\
  \begin{center}
    \begin{tabular}{|l|c|c|}
      \cline{2-3}
      \multicolumn{1}{c|}{} & \multicolumn{1}{c|}{\textbf{Singular}} & \multicolumn{1}{c|}{\textbf{Plural}} \\
      \hline
      \multicolumn{1}{|c|}{\textbf{strukturell}} & \multirow{2}{*}{-e} &  \\
      \multicolumn{1}{|c|}{\textbf{$-$ Akk Mask}} &  &  \\
      \cline{1-2}
      \multicolumn{1}{|c|}{\textbf{oblique}} & \multicolumn{1}{c}{} & \multirow{2}{*}{-en} \\
      \multicolumn{1}{|c|}{\textbf{$+$ Akk Mask}} & \multicolumn{1}{c}{} & \\
      \hline
    \end{tabular}
  \end{center}
\end{frame}


\begin{frame}
  {Gemischt?}
  \pause
  Die Besonderheiten des Indefinit- und Possessivartikels treffen auf die Regularitäten der Adjektivflexion!
  \pause
  \begin{center}
    \begin{tabular}{lll}
      \toprule
      mein\hspace{2em}\rot{\HandCuffRight}  & lecker\rot{-er}   & Kaffee   \\
      mein\alert{-en} & lecker\grau{-en} & Kaffee   \\
      mein\alert{-em} & lecker\grau{-en} & Kaffee   \\
      mein\alert{-es} & lecker\grau{-en} & Kaffees  \\
      \midrule
      mein\hspace{2em}\rot{\HandCuffRight}  & lecker\rot{-es}   & Dessert  \\
      mein\alert{-em} & lecker\grau{-en} & Dessert  \\
      mein\alert{-es} & lecker\grau{-en} & Desserts \\
      \midrule
      mein\alert{-e}  & lecker\grau{-e}  & Brühe    \\
      mein\alert{-er} & lecker\grau{-en} & Brühe    \\
      \midrule
      mein\alert{-e}  & lecker\grau{-en} & Kekse    \\
      mein\alert{-en} & lecker\grau{-en} & Kekse    \\
      mein\alert{-er} & lecker\grau{-en} & Kekse    \\
      \bottomrule
    \end{tabular}
  \end{center}
\end{frame}

\begin{frame}[fragile]
  {Das System, Wiederholung}
  \pause
  \begin{center}
    \begin{tikzpicture}[every text node part/.style={align=center}]
      \node [draw, chamfered rectangle] (FlexAVoran) at (3,6) {Geht ein Artikelwort\\mit Flexionsendung voraus?};

      \node [rounded corners, fill=gray] (SchwaFlexi) at (0,3) {\whyte{struktureller Singular: \textit{-e}}\\\whyte{Rest: \textit{-en}}};
      \node [rounded corners, fill= gray] (pronomAffi) at (6,3) {\whyte{pronominale}\\\whyte{Affixe}};

      \node [draw, rounded corners, inner sep=6pt] (AdFlexAusn) at (3,0) {\textit{-en}};

      \draw (FlexAVoran) -- node [above, sloped] {\footnotesize Ja}       (SchwaFlexi);
      \draw (FlexAVoran) -- node [above, sloped] {\footnotesize Nein}     (pronomAffi);
      \draw [dashed] (SchwaFlexi) -- node [above, sloped] {\footnotesize Ausnahme:} node [below, sloped] {\footnotesize Akkusativ Maskulinum} (AdFlexAusn);
      \draw [dashed] (pronomAffi) -- node [above, sloped] {\footnotesize Ausnahme: Genitiv} node [below, sloped] {\footnotesize Maskulinum\slash Neutrum} (AdFlexAusn);
    \end{tikzpicture}
  \end{center}
\end{frame}


\section{Verbalflexion}

\begin{frame}
  {Flexionsklassen der Verben}
  \pause
  Welche Klassen von Verben haben eigene Flexionsmuster?
  \begin{itemize}[<+->]
    \item \alert{schwache} Verben (die meisten)
    \item \alert{starke} Verben (\alert{Vokalstufen}, nicht nur Ablaut)
    \item "`gemischte"' Verben (wenn es sein muss)
      \Halbzeile
    \item Modalverben
    \item  Hilfsverben
  \end{itemize}
  \pause
  \Zeile
  Was sind die Markierungsfunktionen der Affixe in der Verbalflexion?
  \begin{itemize}[<+->]
    \item Person und Numerus
    \item Tempus
    \item Modus
      \Halbzeile
    \item Infinitheit (verschiedene Sorten)
  \end{itemize}
\end{frame}

\begin{frame}
  {Flexionstypen von Vollverben}
  \pause
  \begin{center}
    \resizebox{\textwidth}{!}{
      \begin{tabular}{llllll}
        \toprule
         & \textbf{2-stufig} & \textbf{3-stufig} & \textbf{U3-stufig} & \textbf{4-stufig} & \textbf{schwach} \\
        \midrule
        \textbf{1 Pers Präs} & heb-e & spring-e & lauf-e & brech-e & lach-e \\
        \textbf{2 Pers Präs} & heb-st & spring-st & läuf-st & brich-st & lach-st \\
        \textbf{1 Pers Prät} & hob & sprang & lief & brach & lach-te \\
        \textbf{Partizip} & ge-hob-en & ge-sprung-en & ge-lauf-en & ge-broch-en & ge-lach-t \\
        \bottomrule
      \end{tabular}
    }
  \end{center}
\end{frame}


\begin{frame}
  {Flexion in den beiden Tempora und den Hauptklassen}
  \pause
  \begin{center}
    \resizebox{0.8\textwidth}{!}{
      \begin{tabular}{llll}
        \toprule
        && \multicolumn{2}{c}{\textbf{schwach}} \\
        \multicolumn{2}{c}{} & \textbf{Präsens} & \textbf{Präteritum} \\
        \midrule
        \multirow{3}{*}{\textbf{Singular}} & \textbf{1} & lach\orongsch{-(e)} & lach\rot{-te} \\
        & \textbf{2} & lach-st & lach\rot{-te}-st \\
        &\textbf{3} & lach\orongsch{-t} & lach\rot{-te} \\
        \midrule
        \multirow{3}{*}{\textbf{Plural}} & \textbf{1} & lach-en & lach\rot{-te}-n \\
        & \textbf{2} & lach-t & lach\rot{-te}-t \\
        & \textbf{3} & lach-en & lach\rot{-te}-n \\
        \bottomrule
      \end{tabular}~\begin{tabular}{ll}
        \toprule
        \multicolumn{2}{c}{\textbf{stark}} \\
        \textbf{Präsens} & \textbf{Präteritum} \\
        \midrule
        brech\orongsch{-(e)} & brach \\
        brich-st & brach-st \\
        brich\orongsch{-t} & brach \\
        \midrule
        brech-en & brach-en \\
        brech-t & brach-t \\
        brech-en & brach-en \\
        \bottomrule
      \end{tabular}
    }
  \end{center}
  \pause
  \begin{itemize}[<+->]
    \item Person-Numerus:
      \begin{itemize}[<+->]
        \item erste Singular \textit{-(e)} nur im Präsens
        \item dritte Singular \textit{-t} nur im Präsens
      \end{itemize}
    \item Präteritum
      \begin{itemize}[<+->]
        \item mit Vokalstufe (stark)
        \item mit Affix \textit{-te} (schwach)
      \end{itemize} 
  \end{itemize}
\end{frame}



\begin{frame}
  {Person-Numerus-Affixe}
  \pause
  Mehr gibt es im ganzen System nicht.\\
  \pause
  \Zeile
  \begin{center}
    \begin{tabular}{llcc}
      \toprule
      \multicolumn{2}{c}{} & \textbf{PN1} & \textbf{PN2} \\
      \midrule
      \multirow{3}{*}{\textbf{Singular}} & \textbf{1} & -(e) & \Dim \\
        & \textbf{2} & \multicolumn{2}{c}{-st} \\
        & \textbf{3} & -t & \Dim \\
      \midrule
      \multirow{2}{*}{\textbf{Plural}} & \textbf{1/3} & \multicolumn{2}{c}{-en} \\
        & \textbf{2} & \multicolumn{2}{c}{-t} \\
      \bottomrule
    \end{tabular}
  \end{center}
\end{frame}

\begin{frame}
  {Konjunktiv}
  \pause
  \begin{center}
   \resizebox{0.8\textwidth}{!}{
      \begin{tabular}{llll}
        \toprule
        && \multicolumn{2}{c}{\textbf{schwach}} \\
        \multicolumn{2}{c}{} & \textbf{Präsens} & \textbf{Präteritum} \\
        \midrule
        \multirow{3}{*}{\textbf{Singular}} & \textbf{1} & lach\alert{-e} & lach-t\alert{-e} \\
        & \textbf{2} & lach\alert{-e}-st & lach-t\alert{-e}-st \\
        & \textbf{3} & lach\alert{-e} & lach-t\alert{-e} \\
        \midrule
        \multirow{3}{*}{\textbf{Plural}} & \textbf{1} & lach\alert{-e}-n & lach-t\alert{-e}-n \\
        & \textbf{2} & lach\alert{-e}-t & lach-t\alert{-e}-t \\
        & \textbf{3} & lach\alert{-e}-n & lach-t\alert{-e}-n \\
        \bottomrule
      \end{tabular}~\begin{tabular}{ll}
        \toprule
        \multicolumn{2}{c}{\textbf{stark}} \\
        \textbf{Präsens} & \textbf{Präteritum} \\
        \midrule
        brech\alert{-e} & bräch\alert{-e} \\
        brech\alert{-e}-st & bräch\alert{-e}-st \\
        brech\alert{-e} & bräch\alert{-e} \\
        \midrule
        brech\alert{-e}-n & bräch\alert{-e}-n \\
        brech\alert{-e}-t & bräch\alert{-e}-t \\
        brech\alert{-e}-n & bräch\alert{-e}-n \\
        \bottomrule
      \end{tabular}
    }
  \end{center}
  \pause
  \begin{itemize}[<+->]
    \item unabhängig von Funktion: Präsens und Präteritum
    \item \alert{immer PN2}
    \item \alert{Umlaut} bei starken Verben
    \item \alert{immer -e} nach Stamm bzw.\ Stamm\textit{-t}(\textit{e})
  \end{itemize}
\end{frame}


\begin{frame}
  {Infinite Formen}
  \pause
  Kein Tempus, keine Person, keinen Numerus, keinen Modus\ldots\\
  \alert{aber verbregiert}.\\

  \pause
  \Halbzeile
  \begin{center}
    \scalebox{0.7}{
      \begin{tabular}{lp{11em}p{11em}}
        \toprule
        & \textbf{Infinitiv} & \textbf{Partizip} \\
        \midrule
        \textbf{schwach} & lach\rot{-en} & ge-lach\rot{-t} \\
        \textbf{stark} & brech\rot{-en} & ge-broch\rot{-en} \\
        \bottomrule
      \end{tabular}
    }

    \pause
    \Zeile
    \scalebox{0.7}{
      \begin{tabular}{lp{11em}p{11em}}
        \toprule
        & \textbf{Infinitiv} & \textbf{Partizip} \\
        \midrule
        \textbf{schwach} & Stamm-\textit{en} & (\textit{ge})-Stamm-\textit{t} \\
        \textbf{stark} & Präsensstamm-\textit{en} & (\textit{ge})-Partizipstamm-\textit{en} \\
        \bottomrule
      \end{tabular}
    }

    \pause
    \Zeile
    \Zeile
    \scalebox{0.7}{
      \begin{tabular}{lp{11em}p{11em}}
        \toprule
        & \textbf{Präfixverb} & \textbf{Partikelverb} \\
        \midrule
        \textbf{schwach} & \textbf{ver:}lach\rot{-t} & \textbf{aus=ge-}lach\rot{-t} \\
        \textbf{stark} & \textbf{unter:}broch\rot{-en} & \textbf{ab=ge-}broch\rot{-en} \\
        \bottomrule
      \end{tabular}
    }
  \end{center}
\end{frame}



\section{Vorschau}

\begin{frame}
  {Konstituentenanalyse und Phrasenbildung}
  \pause
  \begin{itemize}[<+->]
    \item Was ist das Ziel der Syntax?
    \item Wortformen bilden \alert{Phrasen}.
    \item Konstituententests sind \rot{immer heuristisch}!
    \item Wie strukturieren Wörter bestimmter Klassen\\
      den syntaktischen Aufbau in "`ihrer Umgebung"'?
  \end{itemize}
  \pause
  \Zeile
  \begin{center}
    Bitte lesen Sie bis zum 8.~Januar:\\
    \alert{Kapitel 11 und 12 (S.~323--382)}
  \end{center}
\end{frame}

\begin{frame}
  {Literatur}
  \renewcommand*{\bibfont}{\footnotesize}
  \setbeamertemplate{bibliography item}{}
  \printbibliography
\end{frame}


\end{document}
