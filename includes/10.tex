% \section{Rückblick}
% 
% \begin{frame}
%   {Konstituenten}
%   \pause
%   \begin{itemize}[<+->]
%     \item .
%   \end{itemize}
% \end{frame}
% 
% \begin{frame}
%   {Phrasen}
%   \pause
%   \begin{itemize}[<+->]
%     \item .
%   \end{itemize}
% \end{frame}

\section{Überblick}

\begin{frame}
  {Nebensätze und unabhängige Sätze}
  \pause
  \begin{enumerate}[<+->]
    \item \alert{Form}: Nebensätze als Komplementiererphrasen
    \Halbzeile
    \item \alert{Funktion}:
    \begin{itemize}[<+->]
      \item Matrix(satz), Nebensatz, Hauptsatz
      \item Funktionen der unabhängigen und eingebetteten Sätze 
    \end{itemize}
    \item \alert{Form}: Aufbau der unabhängigen Satztypen
  \end{enumerate}
  \pause
  \Zeile
  \rot{Kein Feldermodell in der Vorlesung!}
\end{frame}


\section{Phrasentypen, Fortsetzung}

\begin{frame}
  {Komplementiererphrasen = eingeleitete Nebensätze}
  \pause
  \begin{exe}
    \ex\label{ex:komplementiererphrase111}
    \begin{xlist}
      \ex[]{\label{ex:komplementiererphrase112} Der Arzt möchte, [dass [der Privatpatient die Rechnung \alert{bezahlt}]].}
      \pause
      \ex[*]{\label{ex:komplementiererphrase113} Der Arzt möchte, [dass [der Privatpatient \rot{bezahlt} die Rechnung]].}
      \pause
      \ex[*]{\label{ex:komplementiererphrase114} Der Arzt möchte, [dass [\rot{bezahlt} der Privatpatient die Rechnung]].}
    \end{xlist}
  \end{exe}
  \pause
  \Halbzeile
  \centering
  \begin{forest}
    [KP, calign=first
      [\bf K, tier=preterminal
        [\it dass, name=Kpkopf]
      ]
      [\alert{VP}, tier=preterminal
        [\it der Kassenpatient \alert{geht}, narroof]
      ]
    ]
  \end{forest}\\
  \pause
  \Zeile
  \alert{Verb-Letzt-Stellung!}\\
\end{frame}


\begin{frame}
  {Beispiele für Verbphrasen}
  \pause
  \begin{exe}
  \ex
    \begin{xlist}
      \ex{dass [Ischariot \alert{malt}]}
      \pause
      \ex{dass [Ischariot [das Bild] \alert{malt}]}
      \pause
      \ex{dass [Ischariot [dem Arzt] [das Bild] \alert{verkauft}]}
      \pause
      \ex{dass [Ischariot [wahrscheinlich] [dem Arzt] [heimlich] [das Bild] \\schnell \alert{verkauft}]}
    \end{xlist}
  \end{exe}
  \pause
  \Halbzeile
  \centering
  \scalebox{0.8}{
    \begin{forest}
      l sep+=3em
      [VP, calign=last
        [NP, tier=preterminal
          [\it Ischariot, narroof]
        ]
        [AdvP, tier=preterminal
          [\it wahrscheinlich, narroof]
        ]
        [NP, tier=preterminal
          [\it dem Arzt, narroof]
        ]
        [AdvP, tier=preterminal
          [\it heimlich, narroof]
        ]
        [NP, tier=preterminal
          [\it das Bild, narroof]
        ]
        [AdvP, tier=preterminal
          [\it schnell, narroof]
        ]
        [\bf V, tier=preterminal, bluetree
          [\it verkauft]
        ]
      ]
    \end{forest}
  }
\end{frame}

\begin{frame}
  {Warum Verbkomplexe?}
  \pause
  \begin{exe}
    \ex{\label{ex:verbkomplex121} dass der Junge ein Eis \alert{[isst]}}
    \pause
    \ex\label{ex:verbkomplex122}
    \begin{xlist}
      \ex{\label{ex:verbkomplex123} dass der Junge ein Eis \alert{[essen wird]}}
      \pause
      \ex{\label{ex:verbkomplex124} dass das Eis \alert{[gegessen wird]}}
      \pause
      \ex{\label{ex:verbkomplex125} dass die Freundin das Eis \alert{[kaufen wollen wird]}}
    \end{xlist}
  \end{exe}
  \Zeile
  \pause
  Deutsch: \alert{Verben werden miteinander kombiniert, um Tempora,\\
  Modalität, Diathese usw.\ zu kodieren.}\\
\end{frame}


\begin{frame}
  {Verbkomplexe und Statusrektion}
  \pause
  \Halbzeile
  \centering
  \scalebox{0.7}{
    \begin{forest}
      [\bf V, tier=preterminal
        [\it isst, baseline]
      ]
    \end{forest}
  }
  \hspace{1em}\scalebox{0.7}{
    \begin{forest}
      [\bf V\Sub{B+A}, calign=last, bluetree
        [\bf V\Sub{B}, tier=preterminal, gruentree
          [\it essen\\(1.~Status)]
        ]
        [\bf V\Sub{A}, tier=preterminal
          [\it wird, baseline]
          {\draw [->, trueblue, bend left=30] (.south) to (!uu11.south);}
        ]
      ]
    \end{forest}
  }
  \hspace{1em}\scalebox{0.7}{
    \begin{forest}
      [\bf V\Sub{B+A}, calign=last, bluetree
        [\bf V\Sub{B}, tier=preterminal, gruentree
          [\it gegessen\\(3.~Status)]
        ]
        [\bf V\Sub{A}, tier=preterminal
          [\it wird, baseline]
          {\draw [->, trueblue, bend left=30] (.south) to (!uu11.south);}
        ]
      ]
    \end{forest}
  }
  \hspace{1em}\scalebox{0.7}{
    \begin{forest}
      [\bf V\Sub{C+B+A}, calign=last, bluetree
        [\bf V\Sub{C+B}, calign=last, gruentree
          [\bf V\Sub{C}, tier=preterminal, orongschtree
            [\it kaufen\\(1.~Status)]
          ]
          [\bf V\Sub{B}, tier=preterminal
            [\it wollen\\(1.~Status)]
            {\draw [->, gruen, bend left=30] (.south) to (!uu11.south);}
          ]
        ]
        [\bf V\Sub{A}, tier=preterminal
          [\it wird, baseline]
          {\draw [->, trueblue, bend left=30] (.south) to (!uu121.south);}
        ]
      ]
    \end{forest}
  }
  \pause
  \Halbzeile
  \begin{itemize}[<+->]
    \item Buchstaben (im Buch Zahlen): \alert{Verb A} regiert \gruen{Verb B} regiert \orongsch{Verb C}
    \item Numerierung: Status
      \begin{itemize}[<+->]
        \item 1.~Status: Infinitiv ohne \textit{zu}
        \item 2.~Status: Infinitiv mit \textit{zu}
        \item 3.~Status: Partizip
      \end{itemize}
    \item infinite Verbformen: solche, die von anderen Verben regiert werden
%    \item "`Partizip 1"' keine infinite Verbform (Derivation zum Adjektiv)
  \end{itemize}
\end{frame}


\begin{frame}
  {Verbkomplex und Rektion in der VP}
  \pause
  Die Hilfsverben \textit{heben} die \alert{Valenz-Anforderungen}\\
  lexikalischer Verben zu sich \textit{an}.\\
  \pause
  \centering
  \adjustbox{max width=0.5\textwidth}{%
    \begin{forest}
      l sep+=2em
      [VP, calign=last
        [NP, tier=preterminal, name=subj
          [\it die Freundin, narroof]
        ]
        [NP, tier=preterminal, name=obj
          [\it das Eis, narroof]
        ]
        [\bf V\Sub{\orongsch{C}\gruen{+B}\alert{+A}}, calign=last, bluetree, name=CBAnode
          [\bf V\Sub{\orongsch{C}\gruen{+B}}, calign=last, gruentree, name=CBnode
            [\bf V\Sub{C}, tier=preterminal, orongschtree
              [\it kaufen]
            ]
            {\draw [->, orongsch, bend left=15] (.north) to node [above, near start] {\tiny{(Nom, Akk)}} (CBnode.west);}
            [\bf V\Sub{B}, tier=preterminal
              [\it wollen]
              {\draw [->, gruen, bend left=30] (.south) to node [below] {\footnotesize{1.~Status}} (!uu11.south);}
            ]
          ]
          {\draw [->, orongsch, bend left=15] (.north) to node [above, near start] {\tiny{(Nom, Akk)}} (CBAnode.west);}
          [\bf V\Sub{A}, tier=preterminal
            [\it wird]
            {\draw [->, trueblue, bend left=30] (.south) to node [below, near start] {\footnotesize{1.~Status}} (!uu121.south);}
          ]
        ]
        {\draw [->, orongsch, bend right=80] (.north) to node [above] {\footnotesize{Nom}} (subj.north);}
        {\draw [->, orongsch, bend right=60] (.north) to node [above] {\footnotesize{Akk}} (obj.north);}
      ]
    \end{forest}
  }
\end{frame}


\begin{frame}
  {Komplementiererphrase, Verbphrase und Verbkomplex (Schemata)}
  \centering
  \begin{forest}
    phrasenschema
    [KP, Ephr, calign=first
      [K, Ehd, name=Kpkopf]
      [VP, Eobl]
      {\draw [bend left=45, <-] (.south) to (Kpkopf.south);}
    ]
  \end{forest}
  \pause\hspace{2em}
  \begin{forest}
    phrasenschema
    [VP, Ephr, calign=last
      [Angaben, Emult, Eopt, Erec [Ergänzungen, Eopt, Emult, Erec, name=Vpergaenzi]]
      [V, Ehd]
      {\draw [->, bend left=90] (.south) to (Vpergaenzi.south);}
    ]
  \end{forest}
  \pause\hspace{2em}
  \begin{forest}
    phrasenschema
    [V\Sub{j+i}, Ephr, , calign=last
      [V\Sub{j}, Eopt, name=Vkkopf]
      [V\Sub{i}, Ehd]
      {\draw [->, bend left=30] (.south) to (Vkkopf.south);}
    ]
  \end{forest}
\end{frame}


\section{Sätze}

\subsection{Funktion}

\begin{frame}
  {Definition des "`unabhängigen Satzes"'}
  \pause
  \begin{exe}
    \ex Das Bild hängt an der Wand.
    \pause
    \ex Hängt das Bild an der Wand?
    \pause
    \ex Was hängt an der Wand?
  \end{exe}
  \pause
  \begin{itemize}[<+->]
    \item Definitionskriterien?
      \begin{itemize}[<+->]
        \item Struktur mit \alert{allen Abhängigen} des Verb(komplexe)s
        \item \alert{von keiner anderen Struktur abhängig}
      \end{itemize}
  \end{itemize}
\end{frame}

\begin{frame}
  {Pragmatik unabhängiger Sätze als Definitionskriterium?}
  \pause
  Ein Satz "`kann eine Aussage\slash einen Sprechakt bilden."' \pause --- \rot{Echt jetzt?}
  \pause
  \begin{exe}
    \ex\label{ex:hauptsatzundmatrixsatz007}
    \begin{xlist}
      \ex{\label{ex:hauptsatzundmatrixsatz008} Die Post ist da.}
      \pause
      \ex{\label{ex:hauptsatzundmatrixsatz009} A: Sie geht zum Training.\\
    B: Obwohl es regnet!}
      \pause
      \ex{\label{ex:hauptsatzundmatrixsatz010} Hurra!}
      \pause
      \ex{\label{ex:hauptsatzundmatrixsatz011} Nieder mit dem König!}
    \end{xlist}
  \end{exe}
  \pause
  Sprechakt = Äußerungsakt mit pragmatischen Funktionen,\\
  mit sprachlicher Handlungswirkung
  \Viertelzeile
  \pause
  \begin{itemize}[<+->]
    \item Sind unabhängige Sätze \alert{sprechaktkonstituierend}? --- Ja.
    \item \rot{(\ref{ex:hauptsatzundmatrixsatz009})[B]--(\ref{ex:hauptsatzundmatrixsatz011}) sind Sprechakte, aber keine Sätze.}
    \item \alert{Nebensätze}? --- Sind \alert{vollständig} wie unabhängige Sätze,\\
      aber \alert{syntaktisch abhängig} (oder sogar \alert{regiert}).
  \end{itemize}
\end{frame}


\begin{frame}
  {Funktion: Hypotaxe und komplexe Sachverhalte}
  \pause
  \begin{exe}
    \ex\begin{xlist}
      \ex Es regnet. Juliette geht \alert{trotzdem} zum Training.
      \pause
      \ex \alert{Obwohl} es regnet, geht Juliette zum Training.
    \end{xlist}
    \Viertelzeile
    \pause
    \ex\begin{xlist}
      \ex Es regnet. \alert{Deswegen} fährt Adrianna noch nicht nachhause.
      \pause
      \ex \alert{Weil} es regnet, fährt Adrianna noch nicht nachhause.
    \end{xlist}
    \pause
    \Viertelzeile
    \ex\begin{xlist}
      \ex Kristine bleibt im Garten, \rot{damit} sie nach der Hitze\\
      mehr vom Regen abbekommt.
      \pause
      \ex Kristine bleibt im Garten. \rot{Das Ziel ist, dass} sie nach der Hitze\\
      mehr vom Regen abbekommt.
      \pause
      \ex Kristine bleibt im Garten. \rot{Das Ziel ist} das Abbekommen\\
      von mehr Regen nach der Hitze.
    \end{xlist}
  \end{exe}
  \pause
  \begin{itemize}[<+->]
    \item Komplexe Sachverhalte: \alert{Para- und Hypotaxe} oft austauschbar\\
       bzw.\ \alert{Hypotaxe optional}.
  \end{itemize}
\end{frame}

\begin{frame}
  {Funktionen einzelner Nebensatztypen}
  \pause
  \begin{exe}
    \ex Adrianna weiß, [\alert{dass es bald regnen wird}].\label{ex:ergsatz}
    \pause
    \ex Adrianna und Kristine spielen Tennis, [\alert{während es regnet}].\label{ex:angsatz}
    \pause
    \ex Kristine trifft später die Freundin, [\alert{die eine Katze zu versorgen hat}].\label{ex:relsatz2}
  \end{exe}
  \pause
  \Halbzeile
  \begin{itemize}[<+->]
    \item \alert{Komplementsatz} oder \alert{Ergänzungssatz} in (\ref{ex:ergsatz})
    \item \alert{Adverbialsatz} oder \alert{Angabensatz} in (\ref{ex:angsatz})
    \item \alert{Relativsatz} in (\ref{ex:relsatz2})
    \item Funktionen?
      \begin{itemize}[<+->]
        \item für alle: auf jeden Fall \alert{Hypotaxe =\\
          Erweiterung bildungssprachlicher Möglichkeiten}
      \end{itemize}
    \item systeminterne Funktionen
      \begin{itemize}[<+->]
        \item Semantik des Nebensatzes und der Matrix
        \item konzeptuelle Unabhängigkeit (beider)
      \end{itemize}
  \end{itemize}
\end{frame}

\begin{frame}
  {Konzeptuelle Unabhängigkeit von\\Komplementsatz und Matrix}
  \pause
  \begin{itemize}[<+->]
    \item \alert{Matrix}? --- Die \alert{einbettende} Konstituente.
    \item \alert{konzeptuelle Unabhängigkeit}? --- Enthält alle Konstituenten,\\
      um einen unabhängigen Satz zu bilden.
  \end{itemize}
  \pause\Halbzeile
  \begin{exe}
    \ex
    \begin{xlist}
      \ex Adrianna weiß, [dass \alert{es bald regnen wird}].
      \pause
      \ex → \alert{es bald regnen wird}
      \pause
      \ex → \alert{Es wird bald regnen.}
    \end{xlist}
    \pause
    \ex[*]{\rot{Adrianna weiß.}}
  \end{exe}
  \pause\Halbzeile
  \begin{itemize}[<+->]
    \item Komplement\slash Ergänzungssatz
      \begin{itemize}[<+->]
        \item selber \alert{konzeptuell unabhängig}
        \item Matrix \rot{nicht konzeptuell unabhängig} (ohne Nebensatz)
      \end{itemize}
  \end{itemize}
\end{frame}

\begin{frame}
  {Konzeptuelle Unabhängigkeit von Adverbialsatz und Matrix}
  \pause
  \begin{exe}
    \ex
    \begin{xlist}
      \ex Adrianna und Kristine spielen Tennis, [während \alert{es regnet}].
      \pause
      \ex → \alert{Es regnet.}
    \end{xlist}
    \pause
    \ex \gruen{Adrianna und Kristine spielen Tennis.}
  \end{exe}
  \pause\Halbzeile
  \begin{itemize}[<+->]
    \item Adverbialsatz\slash Angabensatz
      \begin{itemize}[<+->]
        \item selber \alert{konzeptuell unabhängig}
        \item Matrix \alert{konzeptuell unabhängig}
      \end{itemize}
  \end{itemize}
\end{frame}


\begin{frame}
  {Konzeptuelle Unabhängigkeit von Relativsatz und Matrix}
  \pause
  Matrix des Relativsatzes: eine NP
  \pause
  \Halbzeile
  \begin{exe}
    \ex
    \begin{xlist}
      \ex[ ]{Kristine trifft später [die Freundin,\\
        {}[\alert{deren Katze sie verwahren soll}]].}
        \pause
      \ex[ ]{→ deren Katze sie verwahren soll}
      \pause
      \ex[?]{→ \orongsch{Sie soll deren Katze verwahren.}}
    \end{xlist}
    \pause
    \ex \rot{die Freundin}
  \end{exe}
  \pause\Halbzeile
  \begin{itemize}[<+->]
    \item Relativsatz
      \begin{itemize}[<+->]
        \item selber \orongsch{eingeschränkt konzeptuell unabhängig}
        \item Matrix \rot{nicht konzeptuell unabhängig}
      \end{itemize}
  \end{itemize}
\end{frame}

\begin{frame}
  {Semantik: Sachverhalte und Objekte}
  \pause
  \begin{exe}
    \ex{\alert{[Chloë lacht über den Regen]\Sub{S}.}}
    \pause
    \ex{\orongsch{[eine Kommilitonin, die immer gute Fragen stellt]\Sub{NP}}}
  \end{exe}
  \pause\Halbzeile
  \begin{itemize}[<+->]
    \item Sätze bezeichnen \alert{(Mengen von) Sachverhalten (SV)}.
    \item NPs bezeichnen \alert{(Mengen von) (ontologischen) Objekten (OBJ).}
      \Halbzeile
    \item \grau{Achtung: Sachverhalte können wie Objekte behandelt werden\\
      (Reifikation). Wir behandeln den prototypischen Basisfall.}
  \end{itemize}
\end{frame}


\begin{frame}
  {Semantik der Nebensätze und Matrixkonstituenten}
  \pause
  \begin{exe}
    \ex{\alert{\SL Chloë weiß,} \rot{dass} \orongsch{\SL ihre Freundinnen keinen Regen mögen\SR\Sub{SV\Sub{2}}}\alert{\SR\Sub{SV\Sub{1}}}.}
    \pause
    \ex{\alert{\SL Chloë geht zum Sport\SR\Sub{SV\Sub{1}}}, \rot{obwohl} \orongsch{\SL es regnet\SR\Sub{SV\Sub{2}}}.}
    \pause
    \ex{\grau{Chloë ist} \alert{\SL eine Sportlerin,} \orongsch{\SL der Regen nichts ausmacht\SR\Sub{SV}}\alert{\SR\Sub{OBJ}}.}
  \end{exe}
  \pause
  \Halbzeile
  \begin{itemize}[<+->]
    \item Komplement- oder Ergänzungssätze
      \begin{itemize}[<+->]
        \item zwei Sachverhalte
        \item Nebensatz-Sachverhalt ist \alert{Teil des Matrix-Sachverhalts}
      \end{itemize}
    \item Adverbial- oder Angabensätze
      \begin{itemize}[<+->]
        \item zwei Sachverhalte
        \item keine Einschlussrelation
        \item \alert{argumentative\slash rhethorische Relation} (gem.\ Komplementierer)
      \end{itemize}
    \item Relativsätze
      \begin{itemize}[<+->]
        \item (Menge von) Objekten
        \item \alert{zusätzlicher Sachverhalt bzgl.\ dieser Objekte}
      \end{itemize}
  \end{itemize}
\end{frame}

\subsection{Syntax}

\begin{frame}
  {Sätze und Satzähnliches}
  \pause
  \begin{exe}
    \ex{Wir wissen, dass \alert{[der Arzt das Bild schnell \rot{gemalt hat}].}\label{ex:nebensatz}}
    \pause
    \ex{[\alert{Der Arzt} \rot{hat} \alert{das Bild schnell} \rot{gemalt}].}
    \pause
    \ex{[\rot{Hat} \alert{der Arzt das Bild schnell} \rot{gemalt}?]}
    \pause
    \ex{Nihil besucht [\orongsch{den} \alert{Arzt}, [\orongsch{der} \alert{das Bild schnell} \rot{gemalt hat}]].\label{ex:relsatz}}
  \end{exe}
  \pause
  \Halbzeile
  \begin{itemize}[<+->]
    \item Aufgabe der Syntax: \alert{Beschreib das!} Gemeinsamkeiten, Unterschiede?
    \item Vorteil an (\ref{ex:nebensatz}): \alert{Alle Ergänzungen und Angaben des Verbs\\
      werden in einer Kette (der intakten VP) realisiert!}
    \item sonst: Abhängige des Verbs irgendwo verteilt
    \item $\Rightarrow$ Wenn wir die VP in der KP zugrundelegen, kann das Verhältnis\\
      des Verbs und seinen Abhängigen in einer Phrase abgehandelt werden.
%    \item \alert{Einige systematische Ausnahmen stehen im Buch.}
  \end{itemize}
\end{frame}

\begin{frame}
  {Zur Erinnerung: KPs}
  \pause
  \centering
  \begin{forest}
    [\textcolor{gray}{KP}, calign=first
      [\textcolor{gray}{\bf K}, tier=preterminal, edge={gray}
        [\textcolor{gray}{\it dass}, edge={gray}]
      ]
      l sep+=2em
      [VP, calign=last, edge={gray}
        [NP, tier=preterminal
          [\it Ischariot, narroof]
        ]
        [NP, tier=preterminal
          [\it dem Arzt, narroof]
        ]
        [NP, tier=preterminal
          [\it das Bild, narroof]
        ]
        [AdvP, tier=preterminal
          [\it heimlich, narroof]
        ]
        [\bf V, calign=last
          [\bf V, tier=preterminal
            [\it verkauft]
          ]
          [\bf V, tier=preterminal, bluetree
            [\it hat]
          ]
        ]
      ]
    ]
  \end{forest}\\
  \pause
  \Zeile
  {\LARGE\alert{In der KP: Verb-Letzt-Stellung (VL)!}}
\end{frame}

\begin{frame}
  {Von der VP zum V1-Satz: Verb-Erst-Stellung}
  \pause
  {\Large\alert{Finites Verb ganz nach links stellen:}}\\
  \pause
  \begin{center}
    \begin{forest}
        [, phantom, l sep+=2em
          [\bf V\Sub{1}, tier=preterminal, bluetree
            [\it hat, name=BeweV]
          ]
          [VP, calign=last, s sep+=0.5em
            [NP, tier=preterminal
              [\it Ischariot, narroof]
            ]
            [NP, tier=preterminal
              [\it dem Arzt, narroof]
            ]
            [NP, tier=preterminal
              [\it das Bild, narroof]
            ]
            [AdvP, tier=preterminal
              [\it heimlich, narroof]
            ]
            [\bf V, calign=last
              [\bf V, tier=preterminal
                [\it verkauft]
              ]
              [\Ti, bluetree]
              {\draw[dotted, trueblue, thick, ->] (.south) |- ++(0,-4.5em) -| (BeweV.south);}
            ]
          ]
        ]
      \end{forest}\\
    \pause
    \Halbzeile
    {\Large\alert{Resultat: Verb-Erst-Stellung (V1)!}}
  \end{center}
\end{frame}

\begin{frame}
  {Von der V1-Stellung zum V2-Satz: Verb-Zweit-Stellung}
  \pause
  {\Large\alert{Eine beliebige Phrase aus der VP ganz nach links stellen:}}\\
  \pause
  \begin{center}
    \begin{forest}
        [, phantom, l sep+=2em
          [NP, tier=preterminal, orongschtree
            [\it Ischariot, narroof, name=BeweK]
          ]
          [\bf V\Sub{1}, tier=preterminal, bluetree
            [\it hat, name=BeweV]
          ]
          [VP, calign=last, s sep+=0.5em
            [\Tii, tier=preterminal, orongschtree]
            {\draw[dotted, orongsch, thick, ->] (.south) |- ++(0,-5.5em) -| (BeweK.south);}
            [NP, tier=preterminal
              [\it dem Arzt, narroof]
            ]
            [NP, tier=preterminal
              [\it das Bild, narroof]
            ]
            [AdvP, tier=preterminal
              [\it heimlich, narroof]
            ]
            [\bf V, calign=last
              [\bf V, tier=preterminal
                [\it verkauft]
              ]
              [\Ti, bluetree]
              {\draw[dotted, trueblue, thick, ->] (.south) |- ++(0,-4.5em) -| (BeweV.south);}
            ]
          ]
        ]
      \end{forest}\\
    \pause
    \Halbzeile
    {\Large\alert{Resultat: Verb-Zweit-Stellung (V2)!}}
  \end{center}
\end{frame}



\begin{frame}
  {Flexibilität der zweiten Herausstellung}
  \pause
  \centering
  \gruen{Akkusativ-Objekt:} \scalebox{0.5}{%
   \begin{forest}
      [, phantom, l sep+=2em
        [NP, tier=preterminal, gruentree
          [\it das Bild, narroof, name=BeweK]
        ]
        [\bf V\Sub{1}, tier=preterminal, bluetree
          [\it hat, name=BeweV]
        ]
        [VP, calign=last, s sep+=0.5em
          [NP, tier=preterminal
            [\it Ischariot, narroof]
          ]
          [NP, tier=preterminal
            [\it dem Arzt, narroof]
          ]
          [\Tii, tier=preterminal, gruentree]
          {\draw[dotted, gruen, thick, ->] (.south) |- ++(0,-5.5em) -| (BeweK.south);}
          [AdvP, tier=preterminal
            [\it heimlich, narroof]
          ]
          [\bf V, calign=last
            [\bf V, tier=preterminal
              [\it verkauft]
            ]
            [\Ti, bluetree]
            {\draw[dotted, thick, ->] (.south) |- ++(0,-4.5em) -| (BeweV.south);}
          ]
        ]
      ]
    \end{forest}
  }\\
  \pause
  \braun{Dativ-Objekt:} \scalebox{0.5}{%
    \begin{forest}
      [, phantom, l sep+=2em
        [NP, tier=preterminal, brauntree
          [\it dem Arzt, narroof, name=BeweK]
        ]
        [\bf V\Sub{1}, tier=preterminal, bluetree
          [\it hat, name=BeweV]
        ]
        [VP, calign=last, s sep+=0.5em
          [NP, tier=preterminal
            [\it Ischariot, narroof]
          ]
          [\Tii, tier=preterminal, brauntree]
          {\draw[dotted, braun, thick, ->] (.south) |- ++(0,-5.5em) -| (BeweK.south);}
          [NP, tier=preterminal
            [\it das Bild, narroof]
          ]
          [AdvP, tier=preterminal
            [\it heimlich, narroof]
          ]
          [\bf V, calign=last
            [\bf V, tier=preterminal
              [\it verkauft]
            ]
            [\Ti, bluetree]
            {\draw[dotted, thick, ->] (.south) |- ++(0,-4.5em) -| (BeweV.south);}
          ]
        ]
      ]
    \end{forest}
  }\\
  \pause
  \rot{Adverbialphrase:} \scalebox{0.5}{%
    \begin{forest}
      [, phantom, l sep+=2em
        [AdvP, tier=preterminal, rottree
          [\it heimlich, narroof, name=BeweK]
        ]
        [\bf V\Sub{1}, tier=preterminal, bluetree
          [\it hat, name=BeweV]
        ]
        [VP, calign=last, s sep+=0.5em
          [NP, tier=preterminal
            [\it Ischariot, narroof]
          ]
          [NP, tier=preterminal
            [\it dem Arzt, narroof]
          ]
          [NP, tier=preterminal
            [\it das Bild, narroof]
          ]
          [\Tii, tier=preterminal, rottree]
          {\draw[dotted, rot, thick, ->] (.south) |- ++(0,-5.5em) -| (BeweK.south);}
          [\bf V, calign=last
            [\bf V, tier=preterminal
              [\it verkauft]
            ]
            [\Ti, bluetree]
            {\draw[dotted, thick, ->] (.south) |- ++(0,-4.5em) -| (BeweV.south);}
          ]
        ]
      ]
    \end{forest}
  }
\end{frame}

\begin{frame}
  {Schema des V1-Satzes (Ja\slash Nein-Frage)}
  \pause
  \Zeile
  \begin{center}
    \adjustbox{max width=0.5\textwidth}{%
      \begin{minipage}{0.7\textwidth}
        \begin{forest}
        [FS, calign=first
          [\bf V\Sub{1}, tier=preterminal, bluetree
            [\it hat, name=BeweHat]
          ]
          [VP, calign=last
            [NP, tier=preterminal, baseline
              [\it Ischariot, narroof]
            ]
            [AdvP, tier=preterminal
              [\it wahrscheinlich, narroof]
            ]
            [NP, tier=preterminal
              [\it das Bild, narroof]
            ]
            [\bf V, calign=last
              [\bf V, tier=preterminal
                [\it verkauft]
              ]
              [t\Sub{1}, bluetree]
              {\draw[dotted, trueblue, thick, ->] (.south) |- ++(0,-4em) -| (BeweHat.south);}
            ]
          ]
        ]
      \end{forest}
    \end{minipage}
    }
    \pause
    \hspace{0.05\textwidth}\adjustbox{max width=0.5\textwidth}{%
      \begin{minipage}{0.35\textwidth}
        \begin{forest}
        [FS, calign=first, Ephr
          [V\UpSub{finit}{1}, bluetree, Eobl, baseline]
          [VP\\{[\ldots\alert{\Ti}]}, Eobl]
        ]
        \end{forest}
      \end{minipage}
    }
  \end{center}
\end{frame}

\begin{frame}
  {Schema des V2-Satzes ("`Aussagesatz"')}
  \pause
  \begin{center}
    \adjustbox{max width=0.5\textwidth}{%
      \begin{minipage}{0.7\textwidth}
        \begin{forest}
        [S, calign=child, calign child=2
          [NP\Sub{2}, tier=preterminal, gruentree
            [\it das Bild, narroof, name=BeweBild]
          ]
          [\bf V\Sub{1}, tier=preterminal, bluetree
            [\it hat, name=BeweHat]
          ]
          [VP, calign=last
            [NP, tier=preterminal, baseline
              [\it Ischariot, narroof]
            ]
            [AdvP, tier=preterminal
              [\it heimlich, narroof]
            ]
            [t\Sub{2}, tier=preterminal, gruentree
            ]
            {\draw[dotted, gruen, thick, ->] (.south) |- ++(0,-5em) -| (BeweBild.south);}
            [\bf V, calign=last
              [\bf V, tier=preterminal
                [\it verkauft]
              ]
              [t\Sub{1}, bluetree]
              {\draw[dotted, trueblue, thick, ->] (.south) |- ++(0,-4em) -| (BeweHat.south);}
            ]
          ]
        ]
      \end{forest}
    \end{minipage}
    }
    \pause
    \hspace{0.05\textwidth}\adjustbox{max width=0.5\textwidth}{%
      \begin{minipage}{0.35\textwidth}
        \begin{forest}
          [S, calign=child, calign child=2, Ephr
            [XP\Sub{2}, Eobl, baseline, gruentree]
            [V\UpSub{finit}{1}, Eobl, bluetree]
            [VP\\{[\ldots\gruen{\Tii}\ldots\alert{\Ti}]}, Eobl]
          ]
        \end{forest}
      \end{minipage}
    }
  \end{center}
  \pause
  \Halbzeile
  Hat der Satz dann einen \alert{Kopf}?\pause --- In EGBD nicht.\\
  \pause
  \grau{In manchen Theorien\slash Beschreibungen aber schon.}\\
\end{frame}


\begin{frame}
  {Besonderheiten von Partikelverben}
  \pause
  \centering
  \begin{forest}
    [S, calign=child, calign child=2
      [NP\Sub{2}, tier=preterminal, orongschtree
        [\it Sarah, narroof, name=BeweSarah]
      ]
      [\bf V\Sub{1}, tier=preterminal, bluetree
        [\it isst, name=BeweIsst]
      ]
      [VP, calign=last
        [\Tii, tier=preterminal, orongschtree]
        {\draw[dotted, orongsch, thick, ->] (.south) |- ++(0,-5em) -| (BeweSarah.south);}
        [NP, tier=preterminal
          [\it den Kuchen, narroof]
        ]
        [AdvP, tier=preterminal
          [\it alleine, narroof]
        ]
        [\bf V, calign=last
          [Ptkl, tier=preterminal, rottree
            [\it auf{=}]
          ]
          [\Ti, tier=preterminal, bluetree]
          {\draw[dotted, trueblue, thick, ->] (.south) |- ++(0,-4em) -| (BeweIsst.south);}
        ]
      ]
    ]
  \end{forest}\\
  \pause
  \Halbzeile
  {\Large\alert{Wer möchte jetzt immer noch den V2-Satz\\
  ohne Bezug zum VL-Satz beschreiben?}}
\end{frame}

\begin{frame}
  {Relativsätze als etwas andere VL-Sätze}
  \pause
  Das \tuerkis{Relativelement} wird nach links gestellt. Das \alert{Verb} bleibt rechts.\\
  \pause
  \begin{center}
    \adjustbox{max width=0.4\textwidth}{%
      \begin{minipage}{0.7\textwidth}
      \begin{forest}
        [NP, calign=child, calign child=2
          [Art, tier=preterminal
            [\it einen]
          ]
          [\bf N, tier=preterminal
            [\it Tofu]
          ]
          [RS, calign=first
            [NP\Sub{1}, tier=preterminal, tuerkistree
              [\it der, narroof, name=BeweDer]
            ]
            [VP, calign=last
              [NP, tier=preterminal, baseline
                [\it mir, narroof]
              ]
              [\Ti, tier=preterminal, tuerkistree]
              {\draw[dotted, tuerkis, thick, ->] (.south) |- ++(0,-4.5em) -| (BeweDer.south);}
              [Ptkl, tier=preterminal
                [nicht]
              ]
              [\bf V, calign=last
                [\bf V, tier=preterminal
                  [\it geschmeckt]
                ]
                [\bf V, tier=preterminal, bluetree
                  [\it hat]
                ]
              ]
            ]
          ]
        ]
      \end{forest}
    \end{minipage}
    }
    \pause
    \hspace{0.1\textwidth}\adjustbox{max width=0.35\textwidth}{%
      \begin{minipage}{0.35\textwidth}
      \begin{forest}
        [RS, Ephr
          [XP\UpSub{relativ}{1}, Eobl, baseline, tuerkis]
          [VP\\{[\ldots\tuerkis{\Ti}\ldots]}, Eobl]
        ]
      \end{forest}
      \end{minipage}
    }
  \end{center}
  \pause
  \Halbzeile
  \begin{itemize}[<+->]
    \item Relativelement
      \begin{itemize}[<+->]
        \item \alert{Bedeutung}: Bezugs-Substantiv
        \item \alert{Genus, Numerus}: Kongruenz mit Bezugs-Substantiv
        \item \alert{Kasus\slash PP-Form}: gemäß Status als Ergänzung\slash Angabe im RS
      \end{itemize}
  \end{itemize}
\end{frame}


% \begin{frame}
%   {Komplexe Einbettung des Relativelements}
%   \pause
%   Das \tuerkis{Relativelement} als pränominaler Genitiv nimmt die Matrix-NP mit.\\
%   \pause
%   \Halbzeile
%   \centering
%   \begin{forest}
%     [NP, calign=child, calign child=2
%       [Art, tier=preterminal
%         [\it der]
%       ]
%       [\bf N, tier=preterminal
%         [\it Tofu]
%         {\draw [->, bend right=30] (.south) to node [below, near start] {\footnotesize\textsc{Genus,Numerus}} (RekDessen.south);}
%       ]
%       [RS, calign=first
%         [NP\Sub{1}, calign=first, tuerkistree
%           [NP, tier=preterminal
%             [\it dessen, narroof, name=RekDessen]
%           ]
%           [\bf N, tier=preterminal
%             [\it Geschmack, name=RekGeschmack]
%             {\draw [->, bend left=25] (.south) to node [below, near start] {\footnotesize\textsc{Kasus}} (RekDessen.south);}
%           ]
%         ]
%         [VP, calign=last
%           [NP, tier=preterminal
%             [\it ich, narroof]
%           ]
%           [\Ti, tuerkistree]
%           [\bf V, tier=preterminal
%             [\it mag]
%             {\draw [->, bend left=15] (.south) to node [below, near start] {\footnotesize\textsc{Kasus}} (RekGeschmack.south);}
%           ]
%         ]
%       ]
%     ]
%   \end{forest}
% \end{frame}

\begin{frame}
  {Kopulasätze als normale V2-Sätze}
  \pause
  Kopulasätze brauchen kein eigenes Schema.
  \pause
  \begin{center}
    \adjustbox{max width=0.45\textwidth}{%
      \begin{minipage}{0.7\textwidth}
        \begin{forest}
          s sep=3em
          [S, calign=child, calign child=2
            [NP\Sub{2}, tier=preterminal, orongschtree
              [\it die Frau, narroof, name=BeweDiefrau]
            ]
            [\bf V\Sub{1}, tier=preterminal, bluetree
              [\it ist]
            ]
            [VP, calign=last, s sep=2em
              [\Tii, tier=preterminal, forky, orongschtree]
              {\draw[dotted, orongsch, thick, ->] (.south) |- ++(0,-5em) -| (BeweDiefrau.south);}
              [AP, calign=first, gruentree
                [\bf A, tier=preterminal
                  [\it stolz]
                ]
                [PP, tier=preterminal
                  [\it auf ihre Tochter, narroof]
                ]
              ]
              [\Ti, fit=band, bluetree, tier=preterminal]
              {\draw[dotted, blue, thick, ->] (.south) |- ++(0,-4em) -| (BeweIsst.south);}
            ]
          ]
        \end{forest}
      \end{minipage}
    }
  \end{center}
  \pause
  \begin{itemize}[<+->]
    \item Die \alert{Kopula} regiert eine \gruen{AP}, NP oder PP \\
      und eine \orongsch{NP im Nominativ (= "`Subjekt"')}.
    \item Die \gruen{AP} hat eine andere Konstituentenstellung als die attributive.
    \item \grau{Wer sieht ein Problem bei dieser Analyse?}
  \end{itemize}
\end{frame}


\subsection{Objektsätze}

\begin{frame}
  {Objektsätze}
  \pause
  \begin{exe}
    \ex{\label{ex:komplementsaetze127} Michelle weiß, [\rot{dass} die Corvette nicht anspringen wird].}
    \pause
    \ex\label{ex:komplementsaetze128}
    \begin{xlist}
      \ex{\label{ex:komplementsaetze129} Michelle will wissen, [\rot{wer} die Corvette gewartet hat].}
      \pause
      \ex{\label{ex:komplementsaetze130} Michelle will wissen, [\rot{ob} die Corvette gewartet wurde].}
    \end{xlist}
  \end{exe}
  \pause
  \Halbzeile
  \alert{Achtung: \textit{ob} ist eigentlich nur ein w-Wort ohne w (\textit{whether}).}\\
  \pause
  \Halbzeile
\end{frame}

\begin{frame}
  {Regierende Verben und Alternationen}
  \pause
  \alert{Drei primäre Muster}, welche Satz-Objekte Verben regieren.\\
  \pause\Halbzeile
  \begin{exe}
    \ex\label{ex:komplementsaetze131}
    \begin{xlist}
      \ex[]{\label{ex:komplementsaetze132} Michelle behauptet, \alert{dass} die Corvette nicht anspringt.}
      \pause
      \ex[*]{\label{ex:komplementsaetze133} Michelle behauptet, \rot{wie\slash ob} die Corvette nicht anspringt.}
    \end{xlist}
    \pause
    \ex\label{ex:komplementsaetze134}
    \begin{xlist}
      \ex[*]{\label{ex:komplementsaetze135} Michelle untersucht, \rot{dass} der Vergaser funktioniert.}
      \pause
      \ex[]{\label{ex:komplementsaetze136} Michelle untersucht, \alert{wie\slash ob} der Vergaser funktioniert.}
    \end{xlist}
    \pause
    \ex\label{ex:komplementsaetze137}
    \begin{xlist}
      \ex[]{\label{ex:komplementsaetze138} Michelle hört, \alert{dass} die Nockenwelle läuft.}
      \pause
      \ex[]{\label{ex:komplementsaetze139} Michelle hört, \alert{wie\slash ob} die Nockenwelle läuft.}
    \end{xlist}
  \end{exe}
  \pause\Halbzeile
  Außerdem: \textit{dass} alterniert oft mit \textit{zu}-Infinitiv.\\
  \pause
  \Halbzeile
  \begin{exe}
  \ex\label{ex:komplementsaetze140}
  \begin{xlist}
    \ex{\label{ex:komplementsaetze141} Michelle glaubt, [\alert{dass} sie das Geräusch erkennt].}
    \pause
    \ex{\label{ex:komplementsaetze142} Michelle glaubt, [das Geräusch \alert{zu} erkennen].}
  \end{xlist}
  \end{exe}
\end{frame}

\begin{frame}
  {Stellung von Adverbial- und Komplementsätzen}
  \pause
  \Halbzeile
  \begin{exe}
  \ex\label{ex:komplementsaetze146}
  \begin{xlist}
    \ex[]{\label{ex:komplementsaetze147} \alert{[Dass sie unseren Kuchen mag]}, hat Sarah uns eröffnet.}
    \pause
    \ex[]{\label{ex:komplementsaetze148} Sarah hat uns eröffnet, \alert{[dass sie unseren Kuchen mag]}.}
    \pause
    \ex[?]{\label{ex:komplementsaetze149} Sarah hat uns, \rot{[dass sie unseren Kuchen mag]}, eröffnet.}
  \end{xlist}
    \pause

  \ex\label{ex:komplementsaetze150}
  \begin{xlist}
    \ex[]{\label{ex:komplementsaetze151} \alert{[Ob Pavel unseren Kuchen mag]}, haben wir uns oft gefragt.}
    \pause
    \ex[]{\label{ex:komplementsaetze152} Wir haben uns oft gefragt, \alert{[ob Pavel unseren Kuchen mag]}.}
    \pause
    \ex[?]{\label{ex:komplementsaetze153} Wir haben uns, \rot{[ob Pavel unseren Kuchen mag]}, oft gefragt.}
  \end{xlist}
    \pause
  \ex\label{ex:komplementsaetze154}
  \begin{xlist}
    \ex[]{\label{ex:komplementsaetze155} \alert{[Wer die Rosinen geklaut hat]}, wollen wir endlich wissen.}
    \pause
    \ex[]{\label{ex:komplementsaetze156} Wir wollen endlich wissen, \alert{[wer die Rosinen geklaut hat]}.}
    \pause
    \ex[?]{\label{ex:komplementsaetze157} Wir wollen, \rot{[wer die Rosinen geklaut hat]}, endlich wissen.}
  \end{xlist}
  \end{exe}
  \pause
  \begin{itemize}[<+->]
    \item Fast immer Bewegung nach links oder Rechtsversetzung \alert{hinter VK}!\\
    \item \grau{Fehlendes Schema für Rechtsversetzung: Transferaufgabe im Buch.}
  \end{itemize}
  \pause
\end{frame}


\section{Vorschau}

\begin{frame}
  {Prädikate und Relationen}
  \pause
  \begin{itemize}[<+->]
    \item Vorschau auf Phänomene der klassischen theoretischen Syntax
    \item \alert{relevant für grammatische Beschreibung}, auch traditionell
    \item wichtiges Wissen um \rot{Unzulänglichkeiten der Schulterminologie}
      \Halbzeile
    \item Prädikate
    \item Subjekte
    \item Objekte
    \item \alert{Passiv}
  \end{itemize}
  \pause
  \Halbzeile
  \begin{center}
    Bitte lesen Sie bis nächste Woche:\\
    \alert{Kapitel 14 (S.~421--465)} (vor allem 14.1--14.5, S.~421--446)
  \end{center}
  \pause
  \pause
  \pause
  \pause
  \pause
\end{frame}
