\documentclass[handout,aspectratio=1610]{beamer}
%\documentclass[aspectratio=1610]{beamer}

%\usepackage[T1]{fontenc}
\usepackage[ngerman]{babel}
\usepackage{color}
\usepackage{colortbl}
\usepackage{textcomp}
\usepackage{multirow}
\usepackage{nicefrac}
\usepackage{multicol}
\usepackage{gb4e-}
\usepackage{verbatim}
\usepackage{cancel}
\usepackage{graphicx}
\usepackage{hyperref}
\usepackage{verbatim}
\usepackage{boxedminipage}
\usepackage{rotating}
\usepackage{booktabs}
\usepackage{bbding}

\usepackage{tikz}
\usetikzlibrary{positioning,arrows,cd}
\tikzset{>=latex}

\usepackage[linguistics]{forest}

\usepackage{FiraSans}

\usepackage[maxbibnames=99,
  maxcitenames=2,
  uniquelist=false,
  backend=biber,
  doi=false,
  url=false,
  isbn=false,
  bibstyle=biblatex-sp-unified,
  citestyle=sp-authoryear-comp]{biblatex}
\addbibresource{rs.bib}

\forestset{
  Ephr/.style={draw, ellipse, thick, inner sep=2pt},
  Eobl/.style={draw, rounded corners, inner sep=5pt},
  Eopt/.style={draw, rounded corners, densely dashed, inner sep=5pt},
  Erec/.style={draw, rounded corners, double, inner sep=5pt},
  Eoptrec/.style={draw, rounded corners, densely dashed, double, inner sep=5pt},
  Ehd/.style={rounded corners, fill=gray, inner sep=5pt,
    delay={content=\whyte{##1}}
  },
  Emult/.style={for children={no edge}, for tree={l sep=0pt}},
  phrasenschema/.style={for tree={l sep=2em, s sep=2em}},
  decide/.style={draw, chamfered rectangle, inner sep=2pt},
  finall/.style={rounded corners, fill=gray, text=white},
  intrme/.style={draw, rounded corners},
  yes/.style={edge label={node[near end, above, sloped, font=\scriptsize]{Ja}}},
  no/.style={edge label={node[near end, above, sloped, font=\scriptsize]{Nein}}},
  sake/.style={tier=preterminal},
  ake/.style={
    tier=preterminal
    },
}


\tikzset{
    invisible/.style={opacity=0,text opacity=0},
    visible on/.style={alt=#1{}{invisible}},
    alt/.code args={<#1>#2#3}{%
      \alt<#1>{\pgfkeysalso{#2}}{\pgfkeysalso{#3}} % \pgfkeysalso doesn't change the path
    },
}
\forestset{
  visible on/.style={
    for tree={
      /tikz/visible on={#1},
      edge+={/tikz/visible on={#1}}}}}

\definecolor{lg}{rgb}{.8,.8,.8}
\newcommand{\Dim}{\cellcolor{lg}}


\newcommand{\Sub}[1]{\ensuremath{_{\text{#1}}}}
\newcommand{\Up}[1]{\ensuremath{^{\text{#1}}}}

\newcommand{\Ck}{\CheckmarkBold}
\newcommand{\Fl}{\XSolidBrush}

\usetheme[hideothersubsections]{Goettingen}

%\newcommand{\rot}[1]{{\color[rgb]{0.4,0.2,0}#1}}
%\newcommand{\blau}[1]{{\color[rgb]{0,0,.9} #1}}
\newcommand{\gruen}[1]{{\color[rgb]{0,0.4,0}#1}}

\newcommand{\rot}[1]{{\color[rgb]{0.6,0.2,0.0}#1}}
\newcommand{\blau}[1]{{\color[rgb]{0.0,0.0,0.9}#1}}
\newcommand{\orongsch}[1]{{\color[RGB]{255,165,0}#1}}
\newcommand{\grau}[1]{{\color[rgb]{0.5,0.5,0.5}#1}}

\renewcommand<>{\rot}[1]{%
  \alt#2{\beameroriginal{\rot}{#1}}{#1}%
}
\renewcommand<>{\blau}[1]{%
  \alt#2{\beameroriginal{\blau}{#1}}{#1}%
}
\renewcommand<>{\orongsch}[1]{%
  \alt#2{\beameroriginal{\orongsch}{#1}}{#1}%
}

\definecolor{trueblue}{rgb}{0,0.0,0.7}
\setbeamercolor{alerted text}{fg=trueblue}

\newcommand{\xxx}{\hspaceThis{[}}
\newcommand{\zB}{z.\,B.\ }
\newcommand{\down}[1]{\ensuremath{\mathrm{#1}}}

\newcommand{\Zeile}{\vspace{\baselineskip}}
\newcommand{\Halbzeile}{\vspace{0.5\baselineskip}}
\newcommand{\Viertelzeile}{\vspace{0.25\baselineskip}}

\resetcounteronoverlays{exx}

\title{Einführung in die Sprachwissenschaft\\
7.~Wortbildung}
\author{Roland Schäfer}
\institute{Deutsche und niederländische Philologie\\Freie Universität Berlin}
\date{Wintersemester 2018/2019\\11.~Dezember 2018}

\begin{document}

\frame{\titlepage}

\section{Vorab}

\begin{frame}
  {Schreib- und Schriftunterricht}
  \pause
  \begin{itemize}[<+->]
    \item Berliner Schulen (zumindest einige; und wer weiß wo sonst noch):
      \begin{itemize}[<+->]
        \item Kinder sollen in der ersten und zweiten Klasse\\
          erstmal schreiben, wie sie sprechen.
        \item \alert{"`Sehr gut! Aber Erwachsene würden das anders schreiben."'}
      \end{itemize}
      \Halbzeile
    \item schlimmer als die klassische \alert{Hinhörschreibung}:\\
      \rot{\textbf{Hinhörschreibung\slash Sprechschreibung ohne Anleitung\slash Steuerung}}
      \Halbzeile
    \item \rot{ohne jede Fundierung in Lerntheorien}
    \item \rot{kann zu erheblichen permanenten Rechtschreibschwächen führen}
    \item \rot{\textbf{Machen Sie so einen Schwachsinn nicht mit!}}
    \Halbzeile
    \item Natürlich "`lernen es die meisten \rot{trotzdem}"' (durch Lesen),\\
      aber ebenso natürlich gibt es selbst \rot{unter Ihnen} immer noch\\
      Personen mit erheblichen Mängeln in Orthographie, Interpunktion und\\
      bildungssprachlichen Kompetenzen!
    \item \alert{Das muss} \rot{für alle} \alert{vermieden werden.}
  \end{itemize}
\end{frame}

\section{Rückblick}

\begin{frame}
  {Wortbildung und Flexion}
  \pause
  \begin{itemize}[<+->]
    \item Flexion als Mittel zur Dekodierung von (syntaktischer) Struktur
    \item Wortbildung als Mittel der Wortschatzerweiterung und -optimierung
      \Zeile
    \item Markierungsfunktion von Morphen:\\
      \alert{Einschränkung} der möglichen Funktion
    \item Stämme: mit lexikalischer Markierungsfunktion
    \item Affixe: ohne lexikalische Markierungsfunktion; nicht wortfähig
      \Zeile
    \item Umlaut: (morphologisch bedingt und) phonologisch beschreibbar
    \item Ablaut: phonologisch nicht generell beschreibbar
      \Zeile
    \item Wortbildung (gegenüber Flexion)
      \begin{itemize}
        \item Änderung statische Merkmale
        \item Bildung neuer lexikalischer Wörter
        \item meist (semantisch und formal) eingeschränkte Anwendbarkeit
      \end{itemize} 
  \end{itemize}
\end{frame}

% \begin{frame}
%   {Flexionskategorien}
%   \pause
% \end{frame}

\section{Überblick}

\begin{frame}
  {Wortbildung}
  \pause
  \begin{itemize}[<+->]
    \item virtuell\rot{(!)} unbegrenzter Wortschatz
      \Zeile
    \item gut durchschaubares und \alert{gut lernbares} System
    \item (viele Probleme und Einschränkungen im Detail)
      \Zeile
    \item Funktionen der Wortbildung?
      \Halbzeile
      \begin{itemize}
        \item Komposition: \alert{komplexe Konzepte} (\textit{Lötzinnschmelztemperatur})
          \Halbzeile
        \item Konversion: \alert{Reifizierung} (z.B.\ eines Ereignisses als Objekt: \textit{der Lauf})
          \Halbzeile
        \item Derivation: \alert{Modifikation von Bedeutungen} (\textit{un:glaublich}),\\
          \alert{Bezug auf Teilaspekte von Konzepten} (z.\,B.\ Ereigniskonzepten: \textit{Fahr:er})
      \end{itemize}
  \end{itemize}
\end{frame}

\begin{frame}
  {Wichtigkeit von Komposition (inkl.\ Bildungssprache)}
  \pause
  \begin{itemize}[<+->]
    \item Wortbildung als einer der Kerne der Bildungssprache
    \item kann sowohl \alert{verdichten} als auch \alert{präzisieren}
    \Halbzeile
    \item komplexe Sachverhalte \alert{optimiert} formulieren
      \begin{itemize}[<+->]
        \item möglichst kurz
        \item maximal verständlich (Wortbildung hochgradig etabliert\\
          im Deutschen → problemlose Verarbeitung durch Hörer*innen)
      \end{itemize}
      \Halbzeile
    \item Aber meine Position: \rot{Das Unterrichten von\\
      externen Funktionsregularitäten ist gerade im Fall\\
      der Wortbildung extrem schwierig.}
      \Halbzeile
      \begin{itemize}[<+->]
        \item "`Wenn du kommunikativ X erreichen willst,\\
          nimm eine Derivation auf \textit{-igkeit}."'
        \item \dots \alert{wohl kaum!}
        \item \alert{allgemeine souveräne Beherrschung des formalen Systems →\\
          globale Optimierung der Schrift- und Bildungssprache}
      \end{itemize}
  \end{itemize}
\end{frame}

\section{Komposition}

\begin{frame}
  {Beispiele für Komposition}
  \pause
  \begin{exe}
    \ex
    \begin{xlist}
      \ex{Kopf.hörer}
      \pause
      \ex{Laut.sprecher}
      \pause
      \ex{Studenten.werk}
      \pause
      \ex{Lehr.veranstaltung}
      \pause
      \ex{Rot.eiche}
      \pause
      \ex{Lauf.schuhe}
      \pause
      \ex{Ess.besteck}
      \pause
      \ex{Fertig.gericht}
      \pause
      \ex{feuer.rot}
    \end{xlist}
  \end{exe}
\end{frame}

\begin{frame}
  {Produktivität und Transparenz}
  \pause
  \begin{itemize}[<+->]
    \item \alert{alle} Beispiele auf vorheriger Folie: \alert{lexikalisiert}
      \begin{itemize}[<+->]
        \item hohe Häufigkeit
        \item überwiegend spezifischere Bedeutung als Bestandteile vermuten lassen
        \item aber: Art der Bildung erkennbar
        \item \ldots zumindest für erwachsene Sprecher*innen auch bewusst
      \end{itemize}
      \Halbzeile
    \item \alert{transparent}: Rekonstruierbarkeit der Bildung\\
      (auch bei abweichender Gesamtbedeutung)
      \Halbzeile
    \item \alert{produktiv gebildet}: Neubildung durch Sprecher*innen\\
      in einer gegebenen Situation
    \item Produktivität ist \rot{graduell} aufzufassen!
    \item \textit{Buchbutter} > \textit{Batterieschublade} > \textit{Laufschuhe} > \textit{Hundstage}
      \Halbzeile
    \item \alert{produktives Bildungsmuster}: wird häufig spontan\\
      zur Wortbildung verwendet
  \end{itemize}
\end{frame}

\begin{frame}[fragile]
  {Rekursion}
  \pause
  \begin{itemize}[<+->]
    \item Wortbildung: immer \alert{binär}, also \alert{Wort+Wort} (nicht \rot{Wort+Wort+Wort})
      \Viertelzeile
    \item Erinnerung: \alert{hierarchische Strukturbildung} durch \\
      wiederholtes lineares Aneinanderfügen
      \Viertelzeile
    \item Rekursion allgemein: \alert{Eine Verknüpfung hat als Ergebnis\\
      eine Einheit, die wieder auf dieselbe Art verknüpft werden kann.}
    \item linguistische Rekursion: immer eingeschränkt, nicht "`endlos"'
  \end{itemize}
  \pause
  \begin{center}
    \scalebox{0.7}{
      \begin{forest}
        [Bushaltestellenunterstandsreparatur
          [Bushaltestellenunterstand
            [Bushaltestelle
              [Bus]
              [Haltestelle
                [halten]
                [Stelle]
              ]
            ]
            [Unterstand
              [unter]
              [Stand]
            ]
          ]
          [Reparatur]
        ]
      \end{forest}
    }
  \end{center}
\end{frame}

\begin{frame}
  {Köpfe}
  \pause
  \begin{itemize}[<+->]
    \item Wortbildung:
      \begin{itemize}[<+->]
        \item Änderung statischer Merkmale
        \item oder \rot{Löschen} (und Hinzufügen) \rot{von Merkmalen}
      \end{itemize}
      \Viertelzeile
  \end{itemize}
  \pause
  \begin{exe}
    \ex
    \begin{xlist}
      \ex \rot<9->{Laut}.\alert<8->{sprecher} \onslide<9->{\rot{(verliert Wortklasse, \dots)}}
      \pause
      \pause
      \pause
      \ex \rot<12->{Studenten}.\alert<11->{werk} \onslide<12->{\rot{(verliert Wortklasse, Genus, \dots)}}
      \pause
      \pause
      \pause
      \ex \rot<15->{Lauf}.\alert<14->{schuhe} \onslide<15->{\rot{(verliert Wortklasse? Genus? \dots)}}
      \pause
      \pause
      \pause
      \ex \rot<18->{Ess}.\alert<17->{besteck} \onslide<18->{\rot{(verliert Wortklasse, \dots)}}
      \pause
      \pause
      \pause
      \ex \rot<21->{feuer}.\alert<20->{rot} \onslide<21->{\rot{(verliert Wortklasse, \dots)}}
      \pause
      \pause
    \end{xlist}
  \end{exe}
  \pause
  \begin{itemize}[<+->]
    \item \alert{Kopf}:
      \begin{itemize}[<+->]
        \item immer rechts
        \item bestimmt grammatische Merkmale
      \end{itemize}
    \item \alert{Nicht-Kopf}
      \begin{itemize}[<+->]
        \item immer links
        \item verliert alle grammatischen Merkmale
        \item nur Bedeutung bleibt
      \end{itemize}
  \end{itemize}
\end{frame}

\begin{frame}
  {Relevante Kompositionstypen: Determinativkomposita}
  \pause
  \textit{Schulheft}, \textit{Regalbrett} usw.
  \pause
  \Halbzeile
  \begin{itemize}[<+->]
    \item Kopf-Kern-Test:
      \begin{itemize}[<+->]
        \item Ein Schulheft ist ein Heft. \Ck
        \item Ein Regalbrett ist ein Brett. \Ck
      \end{itemize}
    \item Nicht-Kopf-Kern-Test:
      \begin{itemize}[<+->]
        \item Ein Schulheft ist eine Schule. \Fl
        \item Ein Regalbrett ist ein Regal. \Fl
      \end{itemize}
      \Halbzeile
    \item Rektionstest:
      \begin{itemize}[<+->]
        \item \rot{Bei einem Schulheft wird eine geheftet\slash verheftet\slash beheftet... \Fl}
        \item \rot{Bei einem Regalbrett wird ein Regal gebrettert\slash\dots \Fl}
      \end{itemize}
  \end{itemize}
\end{frame}


\begin{frame}
  {Relevante Kompositionstypen: Rektionskomposita}
  \pause
  \textit{Hemdenwäsche}, \textit{Geldfälschung} usw.
  \pause
  \Halbzeile
  \begin{itemize}[<+->]
    \item Kopf-Kern-Test:
      \begin{itemize}[<+->]
        \item Eine Hemdenwäsche ist eine Wäsche. \Ck
        \item Eine Geldfälschung ist eine Fälschung. \Ck
      \end{itemize}
    \item Nicht-Kopf-Kern-Test:
      \begin{itemize}[<+->]
        \item Eine Hemdenwäsche ist ein Hemd. \Fl
        \item Eine Geldfälschung ist Geld. \Fl
      \end{itemize}
      \Halbzeile
    \item Rektionstest:
      \begin{itemize}[<+->]
        \item \alert{Bei einer Hemdenwäsche werden Hemden gewaschen. \Ck}
        \item \alert{Bei einer Geldfälschung wird Geld gefälscht. \Ck}
      \end{itemize}
      \Halbzeile
    \item Kopf: prototypischerweise von einem Verb abgeleitet
    \item Nicht-Kopf zu Kopf wie Objekt zu Verb
  \end{itemize}
\end{frame}

\begin{frame}
  {Kompositionsfugen bei Substantiv-Substantiv-Komposita}
  \pause
  \begin{center}
    \scalebox{1}{
      \begin{tabular}{llrr}
        \toprule
        Fuge          & Beispiel                        & Komposita \% & Erstglieder \% \\
        \midrule                                                                                                    
        $\varnothing$ & \textit{Garten.tür}             & 60.25        & 41.77          \\ 
        -(e)s         & \textit{Gelegenheit-s.dieb}     & 23.69        & 45.74          \\ 
        -n            & \textit{Katze-n.pfote}          & 10.38        &  5.29          \\ 
        -en           & \textit{Frau-en.stimme}         &  3.02        &  4.19          \\ 
        *e            & \textit{Kirsch.kuchen}          &  0.78        &  0.20          \\ 
        -e            & \textit{Geschenk-e.laden}       &  0.71        &  1.90          \\ 
        -er           & \textit{Kind-er.buch}           &  0.38        &  0.07          \\ 
        \char`~er     & \textit{Büch-er.regal}          &  0.37        &  0.11          \\ 
        \char`~e      & \textit{Händ-e.druck}           &  0.22        &  0.63          \\ 
        -ns           & \textit{Name-ns.schutz}         &  0.13        &  0.04          \\ 
        \char`~       & \textit{Mütter.zentrum}        &  0.05        &  0.06           \\ 
        -ens          & \textit{Herz-ens.angelegenheit} &  0.03        &  0.01          \\ 
        \bottomrule
      \end{tabular}
    }\\
    \Halbzeile
    \footnotesize{(aus: \citealt{SchaeferPankratz2018})}
  \end{center}
\end{frame}

\begin{frame}
  {Fugen: erstgliedkontrolliert}
  \pause
  \begin{itemize}[<+->]
    \item Wörter mit s-Plural (\textit{Kaffees}, \textit{Omas}) \rot{niemals mit s-Fuge}
      \Halbzeile
    \item \alert{derivierte} Substantive (meist Abstrakta) (\textit{-heit}, \textit{-keit}, \textit{-tum}):\\
      \alert{prototypisch s-Fuge}
      \begin{itemize}[<+->]
        \item sehr viele Feminina, Fuge nicht paradigmatisch (= keine Flexionsform)
      \end{itemize}
      \Halbzeile
    \item starke\slash gemischte Maskulina: manchmal -(\textit{e})\textit{s}
      \begin{itemize}[<+->]
        \item Genitiv? Welche Funktion sollte ein Genitiv im Kompositum haben?
        \item Lassen sich die Komposita mit s-Fuge mit Genitiv umformulieren?
        \item \textit{Freundeskreis → \rot{*Kreis des Freundes}}
        \item \textit{Geschlechtsverkehr → \rot{*Verkehr des Geschlechts}}
        \item \textit{Berufstätigkeit → \rot{*Tätigkeit des Berufs}}
        \item \textit{Auslandsaufenthalt → \rot{*Aufenthalt des Auslands}}
      \end{itemize}
    \Halbzeile
  \item o.\,g.\ s-Fugen an \alert{Feminina} sowieso nicht als Genitiv möglich:
      \begin{itemize}
        \item \textit{der Dieb \rot{*der Gelegenheits}}
      \end{itemize}
  \end{itemize}
\end{frame}

\section{Konversion}

\begin{frame}
  {Beispiele für Konversion}
  \pause
  \begin{exe}
    \ex[ ]{einkauf-en → Einkauf}
    \pause
    \ex[ ]{einkauf-en → Einkaufen}
    \pause
    \ex[ ]{ernst → Ernst}
    \pause
    \ex[ ]{schwarz → Schwarz}
    \pause
    \ex[ ]{gestrichen → gestrichen}
    \pause
    \ex[!]{schwarz → schwärzen}
    \pause
    \ex[!]{schieß-en → Schuss}
    \pause
    \ex[?]{stech-en → Stich}
  \end{exe}
\end{frame}

\begin{frame}
  {Stammkonversion}
  \pause
  \begin{itemize}[<+->]
    \item Ausgangswort: Stamm
    \item → Zielwort: Stamm \alert{(mit Wortklassenwechsel)}
      \Halbzeile
    \item also \textit{Einkauf}, \textit{Schwarz}, \textit{Ernst}
      \Halbzeile
    \item Zielwort: andere Flexion, gemäß Zielwortklasse
      \begin{itemize}[<+->]
        \item \textit{einkaufst}; \textit{des Einkaufs}
        \item \textit{dem schwarzen Schal}; \textit{dem Schwarz der Nacht}
      \end{itemize}
  \end{itemize}
\end{frame}

\begin{frame}
  {Wortformenkonversion}
  \pause
  \begin{itemize}[<+->]
    \item Ausgangswort: \alert{flektierte Wortform}
    \item → Zielwort: Stamm \alert{(mit Wortklassenwechsel)}
      \Halbzeile
    \item also (\textit{das}) \textit{Einkaufen}, (\textit{das}) \textit{Gemahlene} usw.
      \Halbzeile
    \item bildungsferne Konversion: \textit{"`Wir brauchen noch Fleisch fürs Gehacktes."'}\\
      (ca.\ 2007 im Real Weende, Göttingen)
  \end{itemize}
\end{frame}

\section{Derivation}

\begin{frame}
  {Beispiele für Derivation}
  \pause
  \begin{exe}
    \ex
    \begin{xlist}
      \ex Scherz → scherzhaft
      \pause
      \ex brenn-en → brennbar
      \pause
      \ex grün → grünlich
    \end{xlist}
    \pause
    \Halbzeile
    \ex
    \begin{xlist}
      \ex doof → Doofheit
      \pause
      \ex Fahrer → Fahrerin
      \pause
      \ex Kunde → Kundschaft
      \pause
      \ex Hund → Hündchen
    \end{xlist}
    \pause
    \Halbzeile
    \ex
    \begin{xlist}
      \ex Schlange → schlängeln
      \pause
      \ex Ruck → ruckeln
    \end{xlist}
  \end{exe}
\end{frame}

\begin{frame}
  {Mit und ohne Wortklassenwechsel}
  \pause
  \begin{itemize}[<+->]
    \item mit Wortklassenwechsel: Wortart ändert sich (\textit{Hand} → \textit{händ:isch})
    \item ohen Wortklassenwechsel: Wortart bleibt gleich (\textit{rot} → röt:lich)
      \Zeile
    \item ohne Wortklassenwechsel: geänderte statische Merkmale?
      \begin{itemize}[<+->]
        \item in jedem Fall \alert{Bedeutung}
        \item prototypisch: \textit{Tiefe → Un:tiefe}, \textit{bedeutend → un:bedeutend}
      \end{itemize}
  \end{itemize}
\end{frame}

\begin{frame}
  {Etwas schwierigere Fälle}
  \pause
  \begin{exe}
    \ex
    \begin{xlist}
      \ex{bebeispielen, bestuhlen, bevölkern}
      \ex{entvölkern, entgräten, entwanzen}
      \ex{verholzen, vernageln, verwanzen, verzinnen}
    \end{xlist}
    \pause
    \ex
    \begin{xlist}
      \ex{ergrauen, ermüden, erneuern}
      \ex{befreien, beengen, begrünen}
    \end{xlist}
  \end{exe}
  \pause
  \begin{itemize}[<+->]
    \item entweder Stammkonversion + Präfigierung
    \item oder wortartenverändernde Präfixe
  \end{itemize}
\end{frame}

\begin{frame}
  {In welchem Bereich wird vor allem suffigiert?}
  \pause
  \begin{center}
    \scalebox{0.5}{
      \begin{tabular}{llll}
        \toprule
        \textbf{Ausgangsklasse} & \textbf{Substantiv-Affix} & \textbf{Adjektiv-Affix} & \textbf{Verb-Affix} \\
       \midrule
       \multirow{8}{*}{\textbf{Substantiv}} & \~:chen & :haft & \\
       & \textit{Äst:chen} & \textit{schreck:haft} & \\
       \cmidrule{2-4}
       
       & :in & :ig & \\
       & \textit{Arbeiter:in} & \textit{fisch:ig} & \\
       \cmidrule{2-4}
       
       & :ler & \~:isch & \\
       & \textit{Volkskund:ler} & \textit{händ:isch} & \\
       \cmidrule{2-4}
       
       & :schaft & \~:lich & \\
       & \textit{Wissen:schaft} & \textit{häus:lich} & \\
       
       \midrule
       \multirow{6}{*}{\textbf{Adjektiv}} & :heit & \~:lich & \\
       & \textit{Schön:heit} & \textit{röt:lich} & \\
       \cmidrule{2-4}
       
       & :keit && \\
       & \textit{Heiter:keit} & & \\
       \cmidrule{2-4}
       
       & :igkeit && \\
       & \textit{Neu:igkeit} & & \\
       
       \midrule
       \multirow{6}{*}{\textbf{Verb}} & :er & :bar & \~:el \\
       & \textit{Arbeit:er} & \textit{bieg:bar} & \textit{kreis:el-n} \\
       \cmidrule{2-4}
       
       & :erei && \\
       & \textit{Arbeit:erei} & & \\
       \cmidrule{2-4}
       
       & :ung && \\
       & \textit{Les:ung} & & \\
       
       \bottomrule
      \end{tabular}
    }\\
    \Zeile
    \alert{\large \ldots zum Nomen hin, vor allem zum, Substantiv.}\\
    \rot{\large Und wo wird prototypisch präfigiert?}
  \end{center}
\end{frame}

\begin{frame}
  {Notationskonvention im Buch}
  \pause
  \begin{itemize}[<+->]
    \item \alert{Flexion (und Fuge)} mit Bindestrich: \textit{Tisch-es}, \textit{Fäng-e}
    \item \alert{Komposition} mit Punkt: \textit{Tasche-n.tuch}
    \item \alert{Derivation} mit Doppelpunkt: \textit{Läuf:er}, \textit{be:äugen}
    \item \alert{Verbpartikeln} mit Gleichheitszeichen: \textit{ab=trenn-en}, \textit{um=renn-en}
    \Halbzeile
    \item bei Angabe der einzelnen Affixe, wenn sie Umlaut auslösen:
      \begin{itemize}[<+->]
        \item \char`~ bei Flexion (Plural \textit{\char`~er})
        \item \~: bei Derivation (wie bei \textit{\~:lich})
      \end{itemize}
    \Halbzeile
  \item keine allgemeine Konvention
  \end{itemize}
\end{frame}

\section{Vorschau}

\begin{frame}
  {Die Flexionssysteme}
  \pause
  \begin{itemize}[<+->]
    \item \alert{Nominalflexion}
      \begin{itemize}[<+->]
        \item An welchen Formen erkennen wir die vier Kasus?
        \item Welche Klassen von Substantiven gibt es?
        \item Was unterscheidet Artikel und Pronomina?
        \item Wie sind die \alert{vier} verschiedenen Flexionsmuster\\
          der Artikel und Pronomina beschaffen?
        \item Gibt es wirklich 48 verschiedene Formen des Adjektivs?
      \end{itemize}
      \Zeile
    \item \alert{Verbalflexion}
      \begin{itemize}[<+->]
        \item Wie funktioniert reduzierte Person\slash Numerus-Flexionssystem?
        \item Es gibt nur zwei Tempus- und zwei Modusbildungen!
        \item Was sind infinite und finite Formen?
        \item Was für Verbklassen gibt es (inkl.\ Modal- und Hilfsverben)?
      \end{itemize}
  \end{itemize}
  \pause
  \begin{center}
    Bitte lesen Sie bis nächste Woche:\\
    \alert{Abschnitt 9.2--9.4 9, S.~257--284, Abschnitt 10.2, S.~300--315}
  \end{center}
\end{frame}

\begin{frame}
  {Literatur}
  \renewcommand*{\bibfont}{\footnotesize}
  \setbeamertemplate{bibliography item}{}
  \printbibliography
\end{frame}


\end{document}
